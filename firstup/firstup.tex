\documentclass{beamer}

\mode<presentation>
{
  \usetheme{Warsaw}

  \setbeamercovered{transparent}
}


\usepackage[english]{babel}
\usepackage[utf8x]{inputenc}
%\usepackage[latin1]{inputenc}
\usepackage{ amssymb }

\usepackage{times}
\usepackage[T1]{fontenc}
% Note that the encoding and the font should match. If T1
% does not look nice, try deleting the line with the fontenc.


\title
{...Machine Learning...}

\author[Artur, Ezequiel, Nelson] 
{Artur \and Ezequiel \and Nelson}

\institute
{Universidade do Minho}

\date{26 de Abril} 

\subject{Talks}

% If you have a file called "university-logo-filename.xxx", where xxx
% is a graphic format that can be processed by latex or pdflatex,
% resp., then you can add a logo as follows:

% \pgfdeclareimage[height=0.5cm]{university-logo}{university-logo-filename}
% \logo{\pgfuseimage{university-logo}}



% Delete this, if you do not want the table of contents to pop up at
% the beginning of each subsection:

\AtBeginSubsection[]
{
  \begin{frame}<beamer>{Outline}
    \tableofcontents[currentsection,currentsubsection]
  \end{frame}
}


% If you wish to uncover everything in a step-wise fashion, uncomment
% the following command: 

%\beamerdefaultoverlayspecification{<+->}


\begin{document}

\begin{frame}
  \titlepage
\end{frame}

\begin{frame}{Indice}
  \tableofcontents
\end{frame}

% - Exactly two or three sections (other than the summary).
% - At *most* three subsections per section.
% - Talk about 30s to 2min per frame. So there should be between about
%   15 and 30 frames, all told.

\section{Nelson}

\begin{frame}{titulo}
\end{frame}





\section{Categorias}

\begin{frame}{Uma curta introdução}
\begin{itemize}
 \item<1-> Queremos calcular $\mathcal{D}^{+}$.
 \item<2-> Problema: $\mathcal{D}$ não é computável.
 \item<3-> Solução: observar corolários apresentados e implementar recorrendo a categorias.
\end{itemize}

\end{frame}



\begin{frame}{Uma curta introdução}

    \begin{block}{Corolário 1.1}
    NOTA: adicionar definição do corolário 1.1 aqui
    \end{block}
    
    \begin{block}{Corolário 2.1}
    NOTA: adicionar definição do corolário 2.1 aqui
    \end{block}
    
    \begin{block}{Corolário 3.1}
    NOTA: adicionar definição do corolário 3.1 aqui
    \end{block}
 
\end{frame}



\begin{frame}{Categorias clássicas}

Uma categoria é um conjunto de objetos(conjuntos e tipos) e de morfismos(operações entre objetos).
Uma categoria tem definidas 2 operações básicas, identidade e composição de morfismos, e 2 leis:

\begin{itemize}
    \item<1-> $id \circ f = id \circ f = f$ ---- (C.1)
    \item<2-> $f \circ (g \circ h) = (f \circ g) \circ h$ ---- (C.2)
\end{itemize}

\end{frame}

\begin{frame}{Categorias clássicas}

\begin{columns}
 
\column{0.5\textwidth}
class \textit{Category k} where

\hspace{0.2cm}id :: (a'k'a)
    
\hspace{0.2cm}($\circ$) :: (b'k'c) → (a'k'b) → (a'k'c)
 
\column{0.5\textwidth}
instance \textit{Category (→)} where

\hspace{0.2cm}id = $\lambda$a → a 

\hspace{0.2cm}$g \circ f = \lambda$a → g (f a)  

\end{columns}

\begin{block}{Nota}
Funções lineares e funções diferenciáveis também formam uma categoria própria
\end{block}

\begin{block}{Nota}
Para os efeitos deste papel, objetos são tipos de dados e morfismos são funções
\end{block}

\end{frame}



\begin{frame}{Functores clássicos}

Para converter entre categorias necessitamos de uma nova estrutura: o Functor

Um functor \textit{F} entre categorias $\mathcal{U}$ e $\mathcal{V}$ é tal que:
\begin{itemize}
    \item para qualquer objeto t $\in \mathcal{U}$ temos que \textit{F} t $\in \mathcal{V}$
    \item para qualquer morfismo m :: a → b $\in \mathcal{U}$ temos que \textit{F} m :: \textit{F} a → \textit{F} b $\in \mathcal{V}$
    \item \textit{F} id ($\in \mathcal{U}$) = id ($\in \mathcal{V}$)
    \item \textit{F} ($f \circ g$) = \textit{F} f $\circ$ \textit{F} g
\end{itemize}

Devido à definição de categoria deste papel(objetos são tipos de dados) os functores mapeiam tipos neles próprios.

\end{frame}


\begin{frame}{Objetivo}

Observando as definições dos corolários podemos ver que não só as funções diferenciáveis formam uma categoria mas que é possível criar uma categoria para o qual $\mathcal{D}^{+}$ é functor.

Esta categoria é o tipo de dados produzidos por $\mathcal{D}^{+}$: $a → b \times (a \multimap b)$

Para tornar mais explicita a categoria começamos por definir um novo tipo de dados:
newtype $\mathcal{D}$ a b = $\mathcal{D}$($a → b \times (a \multimap b)$)

\end{frame}

\begin{frame}{Objetivo}

Depois adaptamos $\mathcal{D}^{+}$ para usar este tipo de dados:

\begin{block}{Definição adaptada}

$\mathcal{\hat{D}}$ :: (a → b) → $\mathcal{D}$ a b

$\mathcal{\hat{D}}$ f = $\mathcal{D}$($\mathcal{D}^{+}$ f)

\end{block}

O nosso objetivo é assim deduzir uma instância de categoria para $\mathcal{D}$ onde $\mathcal{\hat{D}}$ seja functor.


\end{frame}



\begin{frame}{Dedução da instância}

Recordando os corolários 3.1 e 1.1 deduzimos que
\begin{itemize}
    \item $\mathcal{D}^{+} id$ = $\lambda$a → (id a,id) ---- (DP.1)
    \item $\mathcal{D}^{+}(g \circ f)$ = $\lambda a → let\{(b,f')$ = $\mathcal{D}^{+}$ f a; $(c,g') = \mathcal{D}^{+}$ g b \} in $(c,g' \circ f'$)  ---- (DP.2)
\end{itemize}

$\mathcal{\hat{D}}$ ser functor é equivalente a dizer que, para todas as funções f e g de tipos apropriados:

\begin{itemize}
    \item id = $\mathcal{\hat{D}}$ id = $\mathcal{D} (\mathcal{D}^{+} id)$
    \item $\mathcal{\hat{D}}$ g $\circ$ $\mathcal{\hat{D}}$ f = $\mathcal{\hat{D}}$  (g $\circ$ f) = $\mathcal{D} (\mathcal{D}^{+} (g \circ f))$
\end{itemize}

\end{frame}


\begin{frame}{Dedução da instância}

Com base em (DP.1) e (DP.2) podemos reescrever como sendo:
\begin{itemize}
    \item id = $\mathcal{D} (\lambda$a → (id a,id))
    \item $\mathcal{\hat{D}}$ g $\circ$ $\mathcal{\hat{D}}$ f = $\mathcal{D}$ ( $\lambda a → let\{(b,f')$ = $\mathcal{D}^{+}$ f a; $(c,g') = \mathcal{D}^{+}$ g b \} in $(c,g' \circ f'$) )
\end{itemize}

Resolver a primeira equação é trivial(definir id como sendo $\mathcal{D} (\lambda$a → (id a,id))).


\end{frame}

\begin{frame}{Dedução da instância}

A segunda equação será resolvida resolvendo uma condição mais geral:

$\mathcal{D} g \circ \mathcal{D} f$ = $\mathcal{D}$($\lambda a → let\{(b,f')$ = f a; $(c,g')$ = g b \} in $(c,g' \circ f'$)).

A resolução desta equação é imediata, levando à seguinte definição da instância:
\end{frame}


\begin{frame}{Dedução da instância}

\begin{block}{Definição de $\mathcal{\hat{D}}$ para funções lineares}
linearD :: (a → b) → $\mathcal{D}$ a b

linearD f = $\mathcal{D}$($\lambda$a → (f a,f))
\end{block}

\begin{block}{Instância da categoria que deduzimos}

instance \textit{Category $\mathcal{D}$} where

\hspace{0.2cm}id = linearD id

\hspace{0.2cm}$\mathcal{D} g \circ \mathcal{D} f$ = $\mathcal{D}$($\lambda a → let\{(b,f')$ = f a; $(c,g')$ = g b \} in $(c,g' \circ f'$))

\end{block}

\end{frame}


\begin{frame}{Prova da instância}

Antes de continuarmos devemos verificar se esta instância obedece às leis (C.1) e (C.2).

Se considerarmos apenas morfismos $\hat{f}$ :: $\mathcal{D}$ a b tal que $\hat{f}$ = $\mathcal{D}^{+}$ f para f :: a → b(o que podemos garantir se transformarmos $\mathcal{D}$ a b em tipo abstrato) podemos garantir que $\mathcal{D}^{+}$ é functor.

\end{frame}


\begin{frame}{Prova da instância}

Prova de (C.1):

id $\circ \mathcal{\hat{D}}$ 

=  $\mathcal{\hat{D}} id \circ \mathcal{\hat{D}}$ f -lei functor de id (especificação de $\mathcal{\hat{D}}$)

= $\mathcal{\hat{D}}$ (id $\circ$ f) - lei functor para ($\circ$)

= $\mathcal{\hat{D}}$ f - lei de categoria


\end{frame}


\begin{frame}{Prova da instância}

Prova de (C.2):

$\mathcal{\hat{D}}$ h $\circ$ ($\mathcal{\hat{D}}$ g $\circ$ $\mathcal{\hat{D}}$ f)

= $\mathcal{\hat{D}}$ h $\circ$ $\mathcal{\hat{D}}$ (g $\circ$ f) - lei functor para ($\circ$)

= $\mathcal{\hat{D}}$ (h $\circ$ (g $\circ$ f)) - lei functor para ($\circ$)

= $\mathcal{\hat{D}}$ ((h $\circ$ g) $\circ$ f) - lei de categoria

= $\mathcal{\hat{D}}$ (h $\circ$ g) $\circ$ $\mathcal{\hat{D}}$  f - lei functor para ($\circ$)

= ($\mathcal{\hat{D}}$ h $\circ$ $\mathcal{\hat{D}}$ g) $\circ$ $\mathcal{\hat{D}}$ f - lei functor para ($\circ$)

\end{frame}

\begin{frame}{Prova da instância}

\begin{alertblock}{Nota}
Estas provas não requerem nada de $\mathcal{D}$ e $\mathcal{\hat{D}}$ para além das leis do functor.

Isto é importante porque sabendo isto não precisaremos mais de voltar a realizar estas provas para instâncias deduzidas a partir de um functor.

\end{alertblock}

\end{frame}




\begin{frame}{Categorias monoidais}

Anteriormente definimos composição paralela, identificando-a como importante para o nosso problema.Definiremos a versão generalizada desta através de uma categoria monoidal:

\begin{columns}
 
\column{0.6\textwidth}
class \textit{Category k} $\Rightarrow$ \textit{Monoidal k} where

\hspace{0.2cm}($\times$)::(a'k'c)→(b'k'd)→((a$\times$b)'k'(c$\times$d))
 
\column{0.4\textwidth}
instance \textit{Monoidal (→)} where

\hspace{0.2cm}$f \times g= \lambda$(a,b)→(f a,g b)  

\end{columns}

\begin{block}{Nota}
Note-se que é possível uma categoria ser monoidal sobre outras operações, mas que para este papel a definição anterior é suficiente.
\end{block}

\end{frame}


\begin{frame}{Functores monoidais}

Do mesmo modo que relacionamos 2 categorias com um functor relacionamos duas categorias monoidais com um functor monoidal.

Um functor \textit{F} monoidal entre categorias $\mathcal{U}$ e $\mathcal{V}$ é tal que:
\begin{itemize}
    \item \textit{F} é functor clássico
    \item \textit{F} (f $\times$ g) = \textit{F} f $\times$ \textit{F} g
\end{itemize}

\end{frame}


\begin{frame}{Dedução da instância}

A dedução da instância passará agora pelo corolário 2.1, de onde deduzimos:

$\mathcal{D}^{+}$ (f $\times$ g) = $\lambda$(a,b) → let\{(c,f' )= $\mathcal{D}^{+}$ f a; (d,g') = $\mathcal{D}^{+}$ g b \} in ((c,d),f'$\times$g')


Seja \textit{F} equivalente a $\mathcal{\hat{D}}$ sob sua definição expandida e invertamos a segunda condição do functor monoidal. Então ficamos com:

$\mathcal{D}$($\mathcal{D}^{+}$ f) $\times$ $\mathcal{D}$($\mathcal{D}^{+}$ g) = $\mathcal{D}$($\mathcal{D}^{+}$ (f $\times$ g))

\end{frame}


\begin{frame}{Dedução da instância}

Substituindo e fortalecendo a condição que tínhamos anteriormente obtemos:

$\mathcal{D}$ f $\times$ $\mathcal{D}$ g = $\mathcal{D}$($\lambda$(a,b) → let\{(c,f') = f a; (d,g') =  g b \} in ((c,d),f'$\times$g'))

e esta condição é suficiente para a nossa instância:

\begin{block}{Instância da categoria que deduzimos}

instance \textit{Monoidal $\mathcal{D}$} where

\hspace{0.2cm}$\mathcal{D}$ f $\times$ $\mathcal{D}$ g = $\mathcal{D}$($\lambda$(a,b) → let\{(c,f') = f a; (d,g') =  g b \} in ((c,d),f'$\times$g'))

\end{block}

\end{frame}





\begin{frame}{Categorias cartesianas}

As categoria monoidais permitem-nos combinar funções mas não separar os dados obtidos.
Para termos tal funcionalidade necessitamos de uma nova classe de categorias: a categoria cartesiana.

\begin{columns}
 
\column{0.6\textwidth}
class \textit{Monoidal k} $\Rightarrow$ \textit{Cartesean k} where

\hspace{0.2cm}exl :: (a$\times$b)'k'a

\hspace{0.2cm}exr :: (a$\times$b)'k'b

\hspace{0.2cm}dup :: a'k'(a$\times$a)
 
\column{0.4\textwidth}
instance \textit{Cartesean (→)} where

\hspace{0.2cm}exl = $\lambda$(a,b) → a

\hspace{0.2cm}exr = $\lambda$(a,b) → b

\hspace{0.2cm}dup = $\lambda$a → (a,a)

\end{columns}

\end{frame}


\begin{frame}{Functores cartesianos}

Do mesmo modo que relacionamos 2 categorias com um functor relacionamos duas categorias cartesianas com um functor cartesiano.

Um functor \textit{F} cartesiano entre categorias $\mathcal{U}$ e $\mathcal{V}$ é tal que:
\begin{itemize}
    \item \textit{F} é functor monoidal
    \item \textit{F} exl = exl
    \item \textit{F} exp = exp
    \item \textit{F} dup = dup
\end{itemize}

\end{frame}


\begin{frame}{Dedução da instância}

A dedução da instância requer primeiro que pelo corolário 3.1 deduzamos(note-se que exl,exr e dup são lineares) que:

$\mathcal{D}^{+}$ exl $\lambda$p → (exp p, exl)

$\mathcal{D}^{+}$ exr $\lambda$p → (exr p, exr)

$\mathcal{D}^{+}$ dup $\lambda$a → (dup a, dup)

Após esta dedução podemos continuar a determinar a instância:

exl = $\mathcal{D}$($\mathcal{D}^{+}$ exl)

exr = $\mathcal{D}$($\mathcal{D}^{+}$ exr)

dup = $\mathcal{D}$($\mathcal{D}^{+}$ dup)

\end{frame}



\begin{frame}{Dedução da instância} 

Substituindo e usando a definição de linearD obtemos:

exl = linearD exl

exr = linearD exr

dup = linearD dup

E podemos converter a dedução acima diretamente em instância:

\begin{block}{Instância da categoria que deduzimos}

instance \textit{Cartesian $\mathcal{D}$} where

\hspace{0.2cm}exl = linearD exl

\hspace{0.2cm}exr = linearD exr

\hspace{0.2cm}dup = linearD dup

\end{block}

\end{frame}



\begin{frame}{Categorias cocartesianas}

São o dual das categorias cartesianas.

Normalmente todas as categorias têm a noção de coproduto definida(somatório de tipos para a categoria (→)). No entanto para este papel os coprodutos coincidem com os produtos de categorias, i.e., estamos a usar categorias de biprodutos.

class \textit{Category k} $\Rightarrow$ \textit{Cocartesian k} where:

\hspace{0.2cm}inl :: a'k'(a$\times$b)

\hspace{0.2cm}inlr:: b'k'(a$\times$b)

\hspace{0.2cm}jam :: (a$\times$a)'k'a

\end{frame}


\begin{frame}{Categorias cocartesianas}

\begin{block}{Nota}
Devido ás limitações que definimos para este tipo de categorias não conseguimos definir uma instância para a categoria  (→). 

Em vez disso definiremos mais tarde uma instância para  ($\to^+$) de funções aditivas que terá uma instância para este tipo de categorias.
\end{block}

\end{frame}


\begin{frame}{Functores cocartesianos}

Do mesmo modo que relacionamos 2 categorias com um functor relacionamos duas categorias cocartesianas com um functor cocartesiano.

Um functor \textit{F} cartesiano entre categorias $\mathcal{U}$ e $\mathcal{V}$ é tal que:
\begin{itemize}
    \item \textit{F} é functor 
    \item \textit{F} inl = inl
    \item \textit{F} inr = inr
    \item \textit{F} jam = jam
\end{itemize}


\end{frame}










\section{Fork e Join}
\begin{frame}{Fork e Join}
    \begin{itemize}
        \item
            $(\triangle)$ :: Cartesian k $\Rightarrow$ (a 'k' c) $\to$ (a 'k' d) $\to$ (a 'k' (c $\times$ d))
        \item
            $(\bigtriangledown)$ :: Cartesian k $\Rightarrow$ (c 'k' a) $\to$ (d 'k' a) $\to$ ((c $\times$ d) 'k' a)
            \pause
        \item
            \textbf{instance} Cocartesian ($\to^+$) \textbf{where}\\
                \hspace{1cm} inl = AddFun inlF\\
                \hspace{1cm} inr = AddFun inrF\\
                \hspace{1cm} jam = AddFun jamF\\
            \vspace{2mm} 
            inlF :: Additive b ⇒ a → a × b\\ 
            inrF :: Additive a ⇒ b → a × b\\
            jamF :: Additive a ⇒ a × a → a\\
            \vspace{2mm} 
            inlF = $\lambda$a → (a, 0)    \\ 
            inrF = $\lambda$b → (0, b)    \\
            jamF = $\lambda$(a, b) → a + b\\ 
    \end{itemize}
\end{frame}

\section{Operacoes Numericas}
\begin{frame}{Operações Numéricas}
    \begin{itemize}
        \item 
            ola
    \end{itemize}
\end{frame}

\section{Exemplos}
\begin{frame}{Exemplos}
\end{frame}

\section{Generalizar}
\begin{frame}{Generalizar}
\end{frame}

%============================EXEMPLO==========================
%\section{Introduction}
%
%\subsection[Short First Subsection Name]{First Subsection Name}
%
%\begin{frame}{Make Titles Informative. Use Uppercase Letters.}{Subtitles are optional.}
%  % - A title should summarize the slide in an understandable fashion
%  %   for anyone how does not follow everything on the slide itself.
%
%  \begin{itemize}
%  \item
%    Use \texttt{itemize} a lot.
%  \item
%    Use very short sentences or short phrases.
%  \end{itemize}
%\end{frame}
%
%\begin{frame}{Make Titles Informative.}
%
%  You can create overlays\dots
%  \begin{itemize}
%  \item using the \texttt{pause} command:
%    \begin{itemize}
%    \item
%      First item.
%      \pause
%    \item    
%      Second item.
%    \end{itemize}
%  \item
%    using overlay specifications:
%    \begin{itemize}
%    \item<3->
%      First item.
%    \item<4->
%      Second item.
%    \end{itemize}
%  \item
%    using the general \texttt{uncover} command:
%    \begin{itemize}
%      \uncover<5->{\item
%        First item.}
%      \uncover<6->{\item
%        Second item.}
%    \end{itemize}
%  \end{itemize}
%\end{frame}
%
%\begin{frame}{}
%\end{frame}
%
%\subsection{Second Subsection}
%
%\begin{frame}{Make Titles Informative.}
%\end{frame}
%
%
%
%\section{Summary}
%
%\subsection{coisas1}
%
%\begin{frame}{Summary}
%
%  % Keep the summary *very short*.
%  \begin{itemize}
%  \item
%    The \alert{first main message} of your talk in one or two lines.
%  \item
%    The \alert{second main message} of your talk in one or two lines.
%  \item
%    Perhaps a \alert{third message}, but not more than that.
%  \end{itemize}
%  
%  % The following outlook is optional.
%  \vskip0pt plus.5fill
%  \begin{itemize}
%  \item
%    Outlook
%    \begin{itemize}
%    \item
%      Something you haven't solved.
%    \item
%      Something else you haven't solved.
%    \end{itemize}
%  \end{itemize}
%\end{frame}
%
%\subsection{coisas2}
%\begin{frame}{coisas2}
%
%
%\end{frame}
%=================================================================

\end{document}



