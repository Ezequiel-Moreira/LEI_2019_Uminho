\documentclass{beamer}

\mode<presentation>
{
  \usetheme{Warsaw}

  \setbeamercovered{transparent}
}


\usepackage[english]{babel}
\usepackage[utf8x]{inputenc}
%\usepackage[latin1]{inputenc}

\usepackage{times}
\usepackage[T1]{fontenc}
% Note that the encoding and the font should match. If T1
% does not look nice, try deleting the line with the fontenc.

%---------- lhs2tex ---------------------------------
%% ODER: format ==         = "\mathrel{==}"
%% ODER: format /=         = "\neq "
%
%
\makeatletter
\@ifundefined{lhs2tex.lhs2tex.sty.read}%
  {\@namedef{lhs2tex.lhs2tex.sty.read}{}%
   \newcommand\SkipToFmtEnd{}%
   \newcommand\EndFmtInput{}%
   \long\def\SkipToFmtEnd#1\EndFmtInput{}%
  }\SkipToFmtEnd

\newcommand\ReadOnlyOnce[1]{\@ifundefined{#1}{\@namedef{#1}{}}\SkipToFmtEnd}
\usepackage{amstext}
\usepackage{amssymb}
\usepackage{stmaryrd}
\DeclareFontFamily{OT1}{cmtex}{}
\DeclareFontShape{OT1}{cmtex}{m}{n}
  {<5><6><7><8>cmtex8
   <9>cmtex9
   <10><10.95><12><14.4><17.28><20.74><24.88>cmtex10}{}
\DeclareFontShape{OT1}{cmtex}{m}{it}
  {<-> ssub * cmtt/m/it}{}
\newcommand{\texfamily}{\fontfamily{cmtex}\selectfont}
\DeclareFontShape{OT1}{cmtt}{bx}{n}
  {<5><6><7><8>cmtt8
   <9>cmbtt9
   <10><10.95><12><14.4><17.28><20.74><24.88>cmbtt10}{}
\DeclareFontShape{OT1}{cmtex}{bx}{n}
  {<-> ssub * cmtt/bx/n}{}
\newcommand{\tex}[1]{\text{\texfamily#1}}	% NEU

\newcommand{\Sp}{\hskip.33334em\relax}


\newcommand{\Conid}[1]{\mathit{#1}}
\newcommand{\Varid}[1]{\mathit{#1}}
\newcommand{\anonymous}{\kern0.06em \vbox{\hrule\@width.5em}}
\newcommand{\plus}{\mathbin{+\!\!\!+}}
\newcommand{\bind}{\mathbin{>\!\!\!>\mkern-6.7mu=}}
\newcommand{\rbind}{\mathbin{=\mkern-6.7mu<\!\!\!<}}% suggested by Neil Mitchell
\newcommand{\sequ}{\mathbin{>\!\!\!>}}
\renewcommand{\leq}{\leqslant}
\renewcommand{\geq}{\geqslant}
\usepackage{polytable}

%mathindent has to be defined
\@ifundefined{mathindent}%
  {\newdimen\mathindent\mathindent\leftmargini}%
  {}%

\def\resethooks{%
  \global\let\SaveRestoreHook\empty
  \global\let\ColumnHook\empty}
\newcommand*{\savecolumns}[1][default]%
  {\g@addto@macro\SaveRestoreHook{\savecolumns[#1]}}
\newcommand*{\restorecolumns}[1][default]%
  {\g@addto@macro\SaveRestoreHook{\restorecolumns[#1]}}
\newcommand*{\aligncolumn}[2]%
  {\g@addto@macro\ColumnHook{\column{#1}{#2}}}

\resethooks

\newcommand{\onelinecommentchars}{\quad-{}- }
\newcommand{\commentbeginchars}{\enskip\{-}
\newcommand{\commentendchars}{-\}\enskip}

\newcommand{\visiblecomments}{%
  \let\onelinecomment=\onelinecommentchars
  \let\commentbegin=\commentbeginchars
  \let\commentend=\commentendchars}

\newcommand{\invisiblecomments}{%
  \let\onelinecomment=\empty
  \let\commentbegin=\empty
  \let\commentend=\empty}

\visiblecomments

\newlength{\blanklineskip}
\setlength{\blanklineskip}{0.66084ex}

\newcommand{\hsindent}[1]{\quad}% default is fixed indentation
\let\hspre\empty
\let\hspost\empty
\newcommand{\NB}{\textbf{NB}}
\newcommand{\Todo}[1]{$\langle$\textbf{To do:}~#1$\rangle$}

\EndFmtInput
\makeatother
%
%
%
%
%
%
% This package provides two environments suitable to take the place
% of hscode, called "plainhscode" and "arrayhscode". 
%
% The plain environment surrounds each code block by vertical space,
% and it uses \abovedisplayskip and \belowdisplayskip to get spacing
% similar to formulas. Note that if these dimensions are changed,
% the spacing around displayed math formulas changes as well.
% All code is indented using \leftskip.
%
% Changed 19.08.2004 to reflect changes in colorcode. Should work with
% CodeGroup.sty.
%
\ReadOnlyOnce{polycode.fmt}%
\makeatletter

\newcommand{\hsnewpar}[1]%
  {{\parskip=0pt\parindent=0pt\par\vskip #1\noindent}}

% can be used, for instance, to redefine the code size, by setting the
% command to \small or something alike
\newcommand{\hscodestyle}{}

% The command \sethscode can be used to switch the code formatting
% behaviour by mapping the hscode environment in the subst directive
% to a new LaTeX environment.

\newcommand{\sethscode}[1]%
  {\expandafter\let\expandafter\hscode\csname #1\endcsname
   \expandafter\let\expandafter\endhscode\csname end#1\endcsname}

% "compatibility" mode restores the non-polycode.fmt layout.

\newenvironment{compathscode}%
  {\par\noindent
   \advance\leftskip\mathindent
   \hscodestyle
   \let\\=\@normalcr
   \let\hspre\(\let\hspost\)%
   \pboxed}%
  {\endpboxed\)%
   \par\noindent
   \ignorespacesafterend}

\newcommand{\compaths}{\sethscode{compathscode}}

% "plain" mode is the proposed default.
% It should now work with \centering.
% This required some changes. The old version
% is still available for reference as oldplainhscode.

\newenvironment{plainhscode}%
  {\hsnewpar\abovedisplayskip
   \advance\leftskip\mathindent
   \hscodestyle
   \let\hspre\(\let\hspost\)%
   \pboxed}%
  {\endpboxed%
   \hsnewpar\belowdisplayskip
   \ignorespacesafterend}

\newenvironment{oldplainhscode}%
  {\hsnewpar\abovedisplayskip
   \advance\leftskip\mathindent
   \hscodestyle
   \let\\=\@normalcr
   \(\pboxed}%
  {\endpboxed\)%
   \hsnewpar\belowdisplayskip
   \ignorespacesafterend}

% Here, we make plainhscode the default environment.

\newcommand{\plainhs}{\sethscode{plainhscode}}
\newcommand{\oldplainhs}{\sethscode{oldplainhscode}}
\plainhs

% The arrayhscode is like plain, but makes use of polytable's
% parray environment which disallows page breaks in code blocks.

\newenvironment{arrayhscode}%
  {\hsnewpar\abovedisplayskip
   \advance\leftskip\mathindent
   \hscodestyle
   \let\\=\@normalcr
   \(\parray}%
  {\endparray\)%
   \hsnewpar\belowdisplayskip
   \ignorespacesafterend}

\newcommand{\arrayhs}{\sethscode{arrayhscode}}

% The mathhscode environment also makes use of polytable's parray 
% environment. It is supposed to be used only inside math mode 
% (I used it to typeset the type rules in my thesis).

\newenvironment{mathhscode}%
  {\parray}{\endparray}

\newcommand{\mathhs}{\sethscode{mathhscode}}

% texths is similar to mathhs, but works in text mode.

\newenvironment{texthscode}%
  {\(\parray}{\endparray\)}

\newcommand{\texths}{\sethscode{texthscode}}

% The framed environment places code in a framed box.

\def\codeframewidth{\arrayrulewidth}
\RequirePackage{calc}

\newenvironment{framedhscode}%
  {\parskip=\abovedisplayskip\par\noindent
   \hscodestyle
   \arrayrulewidth=\codeframewidth
   \tabular{@{}|p{\linewidth-2\arraycolsep-2\arrayrulewidth-2pt}|@{}}%
   \hline\framedhslinecorrect\\{-1.5ex}%
   \let\endoflinesave=\\
   \let\\=\@normalcr
   \(\pboxed}%
  {\endpboxed\)%
   \framedhslinecorrect\endoflinesave{.5ex}\hline
   \endtabular
   \parskip=\belowdisplayskip\par\noindent
   \ignorespacesafterend}

\newcommand{\framedhslinecorrect}[2]%
  {#1[#2]}

\newcommand{\framedhs}{\sethscode{framedhscode}}

% The inlinehscode environment is an experimental environment
% that can be used to typeset displayed code inline.

\newenvironment{inlinehscode}%
  {\(\def\column##1##2{}%
   \let\>\undefined\let\<\undefined\let\\\undefined
   \newcommand\>[1][]{}\newcommand\<[1][]{}\newcommand\\[1][]{}%
   \def\fromto##1##2##3{##3}%
   \def\nextline{}}{\) }%

\newcommand{\inlinehs}{\sethscode{inlinehscode}}

% The joincode environment is a separate environment that
% can be used to surround and thereby connect multiple code
% blocks.

\newenvironment{joincode}%
  {\let\orighscode=\hscode
   \let\origendhscode=\endhscode
   \def\endhscode{\def\hscode{\endgroup\def\@currenvir{hscode}\\}\begingroup}
   %\let\SaveRestoreHook=\empty
   %\let\ColumnHook=\empty
   %\let\resethooks=\empty
   \orighscode\def\hscode{\endgroup\def\@currenvir{hscode}}}%
  {\origendhscode
   \global\let\hscode=\orighscode
   \global\let\endhscode=\origendhscode}%

\makeatother
\EndFmtInput
%
%----------------------------------------------------

\newenvironment{slide}[1]{\begin{frame}\frametitle{#1}}{\end{frame}}

\title
{...Machine Learning...}

\author[Artur, Ezequiel, Nelson] 
{Artur \and Ezequiel \and Nelson}

\institute
{Universidade do Minho}

\date
{26 de Abril}

\subject{Talks}

% If you have a file called "university-logo-filename.xxx", where xxx
% is a graphic format that can be processed by latex or pdflatex,
% resp., then you can add a logo as follows:

% \pgfdeclareimage[height=0.5cm]{university-logo}{university-logo-filename}
% \logo{\pgfuseimage{university-logo}}

% Delete this, if you do not want the table of contents to pop up at
% the beginning of each subsection:

\AtBeginSubsection[]
{
  \begin{frame}<beamer>{Outline}
    \tableofcontents[currentsection,currentsubsection]
  \end{frame}
}


% If you wish to uncover everything in a step-wise fashion, uncomment
% the following command: 

%\beamerdefaultoverlayspecification{<+->}


\begin{document}

\begin{frame}
  \titlepage
\end{frame}

\begin{frame}{Indice}
  \tableofcontents
\end{frame}

% - Exactly two or three sections (other than the summary).
% - At *most* three subsections per section.
% - Talk about 30s to 2min per frame. So there should be between about
%   15 and 30 frames, all told.

\section{Nelson}

\begin{frame}{titulo}
\end{frame}



\section{Categories}

\begin{frame}{Uma curta introdução}
\begin{itemize}
 \item<1-> Queremos calcular $\mathcal{D}^{+}$.
 \item<2-> Problema: $\mathcal{D}$ não é computável.
 \item<3-> Solução: observar corolários apresentados e implementar recorrendo a categorias.
\end{itemize}

\end{frame}



\begin{frame}{Uma curta introdução}

    \begin{block}{Corolário 1.1}
    NOTA: adicionar definição do corolário 1.1 aqui
    \end{block}
    
    \begin{block}{Corolário 2.1}
    NOTA: adicionar definição do corolário 2.1 aqui
    \end{block}
    
    \begin{block}{Corolário 3.1}
    NOTA: adicionar definição do corolário 3.1 aqui
    \end{block}
 
\end{frame}



\begin{frame}{Categorias clássicas}

Uma categoria é um conjunto de objetos(conjuntos e tipos) e de morfismos(operações entre objetos), tendo definidas 2 operações básicas, identidade e composição de morfismos, e 2 leis:

\begin{itemize}
    \item<1-> (C.1) ---- $id \circ f = id \circ f = f$ 
    \item<2-> (C.2) ---- $f \circ (g \circ h) = (f \circ g) \circ h$
\end{itemize}


\begin{block}

Para os efeitos deste papel, objetos são tipos de dados e morfismos são funções

\end{block}

\begin{block}

\begin{columns}
 
\column{0.5\textwidth}
class \textit{Category k} where

\hspace{0.2cm}id :: (a'k'a)
    
\hspace{0.2cm}($\circ$) :: (b'k'c) → (a'k'b) → (a'k'c)
 
\column{0.5\textwidth}
instance \textit{Category (→)} where

\hspace{0.2cm}id = $\lambda$a → a 

\hspace{0.2cm}$g \circ f = \lambda$a → g (f a)  

\end{columns}

\end{block}

\end{frame}




\begin{frame}{Functores clássicos}


Um functor \textit{F} entre categorias $\mathcal{U}$ e $\mathcal{V}$ é tal que:
\begin{itemize}
    \item para qualquer objeto t $\in \mathcal{U}$ temos que \textit{F} t $\in \mathcal{V}$
    \item para qualquer morfismo m :: a → b $\in \mathcal{U}$ temos que \textit{F} m :: \textit{F} a → \textit{F} b $\in \mathcal{V}$
    \item \textit{F} id ($\in \mathcal{U}$) = id ($\in \mathcal{V}$)
    \item \textit{F} ($f \circ g$) = \textit{F} f $\circ$ \textit{F} g
\end{itemize}


\begin{block}{Nota}
Devido à definição de categoria deste papel(objetos são tipos de dados) os functores mapeiam tipos neles próprios.
\end{block}

\end{frame}

\begin{frame}{Objetivo}

Começamos por definir um novo tipo de dados:

newtype $\mathcal{D}$ a b = $\mathcal{D}$($a → b \times (a \multimap b)$)

Depois adaptamos $\mathcal{D}^{+}$ para usar este tipo de dados:

\begin{block}{Definição adaptada}

$\mathcal{\hat{D}}$ :: (a → b) → $\mathcal{D}$ a b

$\mathcal{\hat{D}}$ f = $\mathcal{D}$($\mathcal{D}^{+}$ f)

\end{block}

O nosso objetivo é a dedução de uma instância de categoria para $\mathcal{D}$ onde $\mathcal{\hat{D}}$ seja functor.


\end{frame}



\begin{frame}{Dedução da instância}

Recordando os corolários 3.1 e 1.1 deduzimos que
\begin{itemize}
    \item (DP.1) ---- $\mathcal{D}^{+} id$ = $\lambda$a → (id a,id)
    \item (DP.2) ----
    
    $\mathcal{D}^{+}(g \circ f)$ = $\lambda a → let\{(b,f')$ = $\mathcal{D}^{+}$ f a; $(c,g') = \mathcal{D}^{+}$ g b \} in $(c,g' \circ f'$)   
\end{itemize}

$\mathcal{\hat{D}}$ ser functor é equivalente a dizer que, para todas as funções f e g de tipos apropriados:

\begin{itemize}
    \item id = $\mathcal{\hat{D}}$ id = $\mathcal{D} (\mathcal{D}^{+} id)$
    \item $\mathcal{\hat{D}}$ g $\circ$ $\mathcal{\hat{D}}$ f = $\mathcal{\hat{D}}$  (g $\circ$ f) = $\mathcal{D} (\mathcal{D}^{+} (g \circ f))$
\end{itemize}

\end{frame}


\begin{frame}{Dedução da instância}

Com base em (DP.1) e (DP.2) podemos reescrever como sendo:
\begin{itemize}
    \item id = $\mathcal{D} (\lambda$a → (id a,id))
    \item $\mathcal{\hat{D}}$ g $\circ$ $\mathcal{\hat{D}}$ f = $\mathcal{D}$ ( $\lambda a → let\{(b,f')$ = $\mathcal{D}^{+}$ f a; $(c,g') = \mathcal{D}^{+}$ g b \} in $(c,g' \circ f'$) )
\end{itemize}

Resolver a primeira equação é trivial(definir id da instância como sendo $\mathcal{D} (\lambda$a → (id a,id))).


A segunda equação será resolvida resolvendo uma condição mais geral:
$\mathcal{D} g \circ \mathcal{D} f$ = $\mathcal{D}$($\lambda a → let\{(b,f')$ = f a; $(c,g')$ = g b \} in $(c,g' \circ f'$)), cuja solução é igualmente trivial.


\end{frame}

\begin{frame}{Dedução da instância}


\begin{block}{Definição de $\mathcal{\hat{D}}$ para funções lineares}
linearD :: (a → b) → $\mathcal{D}$ a b

linearD f = $\mathcal{D}$($\lambda$a → (f a,f))
\end{block}

\begin{block}{Instância da categoria que deduzimos}

instance \textit{Category $\mathcal{D}$} where

\hspace{0.2cm}id = linearD id

\hspace{0.2cm}$\mathcal{D} g \circ \mathcal{D} f$ = $\mathcal{D}$($\lambda a → let\{(b,f')$ = f a; $(c,g')$ = g b \} in $(c,g' \circ f'$))

\end{block}


\end{frame}



\begin{frame}{Prova da instância}

Antes de continuarmos devemos verificar se esta instância obedece às leis (C.1) e (C.2).

Se considerarmos apenas morfismos $\hat{f}$ :: $\mathcal{D}$ a b tal que $\hat{f}$ = $\mathcal{D}^{+}$ f para f :: a → b(o que podemos garantir se transformarmos $\mathcal{D}$ a b em tipo abstrato) podemos garantir que $\mathcal{D}^{+}$ é functor.


\begin{block}{Prova de (C.1)}

id $\circ \mathcal{\hat{D}}$ 

=  $\mathcal{\hat{D}} id \circ \mathcal{\hat{D}}$ f -lei functor de id (especificação de $\mathcal{\hat{D}}$)

= $\mathcal{\hat{D}}$ (id $\circ$ f) - lei functor para ($\circ$)

= $\mathcal{\hat{D}}$ f - lei de categoria
\end{block}


\end{frame}

\begin{frame}{Prova da instância}

\begin{block}{Prova de (C.2)}

$\mathcal{\hat{D}}$ h $\circ$ ($\mathcal{\hat{D}}$ g $\circ$ $\mathcal{\hat{D}}$ f)

= $\mathcal{\hat{D}}$ h $\circ$ $\mathcal{\hat{D}}$ (g $\circ$ f) - lei functor para ($\circ$)

= $\mathcal{\hat{D}}$ (h $\circ$ (g $\circ$ f)) - lei functor para ($\circ$)

= $\mathcal{\hat{D}}$ ((h $\circ$ g) $\circ$ f) - lei de categoria

= $\mathcal{\hat{D}}$ (h $\circ$ g) $\circ$ $\mathcal{\hat{D}}$  f - lei functor para ($\circ$)

= ($\mathcal{\hat{D}}$ h $\circ$ $\mathcal{\hat{D}}$ g) $\circ$ $\mathcal{\hat{D}}$ f - lei functor para ($\circ$)

\end{block}

\begin{alertblock}{Nota}
Estas provas não requerem nada de $\mathcal{D}$ e $\mathcal{\hat{D}}$ para além das leis do functor, logo nas próximas instâncias deduzidas de um functor não precisamos de voltar a realizar estas provas.

\end{alertblock}

\end{frame}




\begin{frame}{Categorias e functores monoidais}

A versão generalizada da composição paralela será definida através de uma categoria monoidal:

\begin{block}


\begin{columns}
 
\column{0.5\textwidth}
class \textit{Category k} $\Rightarrow$ \textit{Monoidal k} where

\hspace{0.2cm}($\times$)::(a'k'c)→(b'k'd)→((a$\times$b)'k'(c$\times$d))
 
\column{0.4\textwidth}
instance \textit{Monoidal (→)} where

\hspace{0.2cm}$f \times g= \lambda$(a,b)→(f a,g b)  

\end{columns}

\end{block}


\begin{block}{Definição de functor monoidal}

Um functor \textit{F} monoidal entre categorias $\mathcal{U}$ e $\mathcal{V}$ é tal que:
\begin{itemize}
    \item \textit{F} é functor clássico
    \item \textit{F} (f $\times$ g) = \textit{F} f $\times$ \textit{F} g
\end{itemize}

\end{block}

\end{frame}



\begin{frame}{Dedução da instância}

A partir do corolário 2.1 deduzimos que:

$\mathcal{D}^{+}$ (f $\times$ g) = $\lambda$(a,b) → let\{(c,f' )= $\mathcal{D}^{+}$ f a; (d,g') = $\mathcal{D}^{+}$ g b \} in ((c,d),f'$\times$g')

Se definirmos o functor F a partir de $\mathcal{\hat{D}}$ chegamos à seguinte condição:

$\mathcal{D}$($\mathcal{D}^{+}$ f) $\times$ $\mathcal{D}$($\mathcal{D}^{+}$ g) = $\mathcal{D}$($\mathcal{D}^{+}$ (f $\times$ g))

Substituindo e fortalecendo-a obtemos:

$\mathcal{D}$ f $\times$ $\mathcal{D}$ g = $\mathcal{D}$($\lambda$(a,b) → let\{(c,f') = f a; (d,g') =  g b \} in ((c,d),f'$\times$g'))

e esta condição é suficiente para obtermos a nossa instância.

\end{frame}


\begin{frame}{Dedução da instância}

\begin{block}{Instância da categoria que deduzimos}

instance \textit{Monoidal $\mathcal{D}$} where

\hspace{0.2cm}$\mathcal{D}$ f $\times$ $\mathcal{D}$ g = $\mathcal{D}$($\lambda$(a,b) → let\{(c,f') = f a; (d,g') =  g b \} in ((c,d),f'$\times$g'))

\end{block}

\end{frame}



\begin{frame}{Categorias e funtores cartesianas}

\begin{block}

\begin{columns}
 
\column{0.5\textwidth}
class \textit{Monoidal k} $\Rightarrow$ \textit{Cartesean k} where

\hspace{0.2cm}exl :: (a$\times$b)'k'a

\hspace{0.2cm}exr :: (a$\times$b)'k'b

\hspace{0.2cm}dup :: a'k'(a$\times$a)
 
\column{0.4\textwidth}
instance \textit{Cartesean (→)} where

\hspace{0.2cm}exl = $\lambda$(a,b) → a

\hspace{0.2cm}exr = $\lambda$(a,b) → b

\hspace{0.2cm}dup = $\lambda$a → (a,a)

\end{columns}

\end{block}


\begin{block}


Um functor \textit{F} cartesiano entre categorias $\mathcal{U}$ e $\mathcal{V}$ é tal que:
\begin{itemize}
    \item \textit{F} é functor monoidal
    \item \textit{F} exl = exl
    \item \textit{F} exp = exp
    \item \textit{F} dup = dup
\end{itemize}


\end{block}


\end{frame}




\begin{frame}{Dedução da instância}

Pelo corolário 3.1 e pelo facto que exl,exr e dup são linerares deduzimos que:

$\mathcal{D}^{+}$ exl $\lambda$p → (exp p, exl)

$\mathcal{D}^{+}$ exr $\lambda$p → (exr p, exr)

$\mathcal{D}^{+}$ dup $\lambda$a → (dup a, dup)

Após esta dedução podemos continuar a determinar a instância:

exl = $\mathcal{D}$($\mathcal{D}^{+}$ exl)

exr = $\mathcal{D}$($\mathcal{D}^{+}$ exr)

dup = $\mathcal{D}$($\mathcal{D}^{+}$ dup)

\end{frame}



\begin{frame}{Dedução da instância} 

Substituindo e usando a definição de linearD obtemos:

exl = linearD exl

exr = linearD exr

dup = linearD dup

E podemos converter a dedução acima diretamente em instância:

\begin{block}{Instância da categoria que deduzimos}

instance \textit{Cartesian $\mathcal{D}$} where

\hspace{0.2cm}exl = linearD exl

\hspace{0.2cm}exr = linearD exr

\hspace{0.2cm}dup = linearD dup

\end{block}

\end{frame}



\begin{frame}{Categorias cocartesianas}

São o dual das categorias cartesianas.
\begin{block}{Nota}
Neste papel os coprodutos correspondem aos produtos das categorias, i.e., categorias de biprodutos.
\end{block}

\begin{block}

class \textit{Category k} $\Rightarrow$ \textit{Cocartesian k} where:

\hspace{0.2cm}inl :: a'k'(a$\times$b)

\hspace{0.2cm}inlr:: b'k'(a$\times$b)

\hspace{0.2cm}jam :: (a$\times$a)'k'a

\end{block}


\end{frame}


\begin{frame}{Functores cocartesianos}

\begin{block}{Definição de functor cocartesiano}


Um functor \textit{F} cartesiano entre categorias $\mathcal{U}$ e $\mathcal{V}$ é tal que:
\begin{itemize}
    \item \textit{F} é functor 
    \item \textit{F} inl = inl
    \item \textit{F} inr = inr
    \item \textit{F} jam = jam
\end{itemize}


\end{block}


\end{frame}

\section{Fork and Join}
\begin{slide}{Fork and Join}
    \begin{itemize}
        \item
            \ensuremath{ \Delta \mathbin{::}\Conid{Cartesian}\;\Varid{k}\Rightarrow(\Varid{a}\;\text{\tt 'k'}\;\Varid{c})\rightarrow (\Varid{a}\;\text{\tt 'k'}\;\Varid{d})\rightarrow (\Varid{a}\;\text{\tt 'k'}\;(\Varid{c} \times \Varid{d}))}
        \item
            \ensuremath{ \nabla \mathbin{::}\Conid{Cartesian}\;\Varid{k}\Rightarrow(\Varid{c}\;\text{\tt 'k'}\;\Varid{a})\rightarrow (\Varid{d}\;\text{\tt 'k'}\;\Varid{a})\rightarrow ((\Varid{c} \times \Varid{d})\;\text{\tt 'k'}\;\Varid{a})}
    \end{itemize}
\end{slide}

\begin{slide}{Instance of \ensuremath{\rightarrow^+ }}
\begin{hscode}\SaveRestoreHook
\column{B}{@{}>{\hspre}l<{\hspost}@{}}%
\column{4}{@{}>{\hspre}l<{\hspost}@{}}%
\column{E}{@{}>{\hspre}l<{\hspost}@{}}%
\>[B]{}\mathbf{newtype}\;\Varid{a}\rightarrow^+ \Varid{b}\mathrel{=}\Conid{AddFun}\;(\Varid{a}\rightarrow \Varid{b}){}\<[E]%
\\[\blanklineskip]%
\>[B]{}\mathbf{instance}\;\Conid{Category}\;(\rightarrow^+ )\;\mathbf{where}{}\<[E]%
\\
\>[B]{}\hsindent{4}{}\<[4]%
\>[4]{}\mathbf{type}\;\Conid{Obj}\;(\rightarrow^+ )\mathrel{=}\Conid{Additive}{}\<[E]%
\\
\>[B]{}\hsindent{4}{}\<[4]%
\>[4]{}\Varid{id}\mathrel{=}\Conid{AddFun}\;\Varid{id}{}\<[E]%
\\
\>[B]{}\hsindent{4}{}\<[4]%
\>[4]{}\Conid{AddFun}\;\Varid{g}\mathbin{\circ}\Conid{AddFun}\;\Varid{f}\mathrel{=}\Conid{AddFun}\;(\Varid{g}\mathbin{\circ}\Varid{f}){}\<[E]%
\\[\blanklineskip]%
\>[B]{}\mathbf{instance}\;\Conid{Monoidal}\;(\rightarrow^+ )\;\mathbf{where}{}\<[E]%
\\
\>[B]{}\hsindent{4}{}\<[4]%
\>[4]{}\Conid{AddFun}\;\Varid{f} \times \Conid{AddFun}\;\Varid{g}\mathrel{=}\Conid{AddFun}\;(\Varid{f} \times \Varid{g}){}\<[E]%
\\[\blanklineskip]%
\>[B]{}\mathbf{instance}\;\Conid{Cartesian}\;(\rightarrow^+ )\;\mathbf{where}{}\<[E]%
\\
\>[B]{}\hsindent{4}{}\<[4]%
\>[4]{}\Varid{exl}\mathrel{=}\Conid{AddFun}\;\Varid{exl}{}\<[E]%
\\
\>[B]{}\hsindent{4}{}\<[4]%
\>[4]{}\Varid{exr}\mathrel{=}\Conid{AddFun}\;\Varid{exr}{}\<[E]%
\\
\>[B]{}\hsindent{4}{}\<[4]%
\>[4]{}\Varid{dup}\mathrel{=}\Conid{AddFun}\;\Varid{dup}{}\<[E]%
\ColumnHook
\end{hscode}\resethooks
\end{slide}

\begin{slide}{Instance of \ensuremath{\rightarrow^+ }}
\begin{hscode}\SaveRestoreHook
\column{B}{@{}>{\hspre}l<{\hspost}@{}}%
\column{5}{@{}>{\hspre}l<{\hspost}@{}}%
\column{E}{@{}>{\hspre}l<{\hspost}@{}}%
\>[B]{}\mathbf{instance}\;\Conid{Cocartesian}\;(\rightarrow^+ )\;\mathbf{where}{}\<[E]%
\\
\>[B]{}\hsindent{5}{}\<[5]%
\>[5]{}\Varid{inl}\mathrel{=}\Conid{AddFun}\;\Varid{inlF}{}\<[E]%
\\
\>[B]{}\hsindent{5}{}\<[5]%
\>[5]{}\Varid{inr}\mathrel{=}\Conid{AddFun}\;\Varid{inrF}{}\<[E]%
\\
\>[B]{}\hsindent{5}{}\<[5]%
\>[5]{}\Varid{jam}\mathrel{=}\Conid{AddFun}\;\Varid{jamF}{}\<[E]%
\\[\blanklineskip]%
\>[B]{}\Varid{inlF}\mathbin{::}\Conid{Additive}\;\Varid{b}\Rightarrow\Varid{a}\rightarrow \Varid{a} \times \Varid{b}{}\<[E]%
\\
\>[B]{}\Varid{inrF}\mathbin{::}\Conid{Additive}\;\Varid{a}\Rightarrow\Varid{b}\rightarrow \Varid{a} \times \Varid{b}{}\<[E]%
\\
\>[B]{}\Varid{jamF}\mathbin{::}\Conid{Additive}\;\Varid{a}\Rightarrow\Varid{a} \times \Varid{a}\rightarrow \Varid{a}{}\<[E]%
\\[\blanklineskip]%
\>[B]{}\Varid{inlF}\mathrel{=}\lambda \Varid{a}\rightarrow (\Varid{a},\mathrm{0}){}\<[E]%
\\
\>[B]{}\Varid{inrF}\mathrel{=}\lambda \Varid{b}\rightarrow (\mathrm{0},\Varid{b}){}\<[E]%
\\
\>[B]{}\Varid{jamF}\mathrel{=}\lambda (\Varid{a},\Varid{b})\rightarrow \Varid{a}\mathbin{+}\Varid{b}{}\<[E]%
\ColumnHook
\end{hscode}\resethooks
\end{slide}

\section{Operacoes Numericas}
\begin{slide}{NumCat definition}
\begin{hscode}\SaveRestoreHook
\column{B}{@{}>{\hspre}l<{\hspost}@{}}%
\column{5}{@{}>{\hspre}l<{\hspost}@{}}%
\column{9}{@{}>{\hspre}l<{\hspost}@{}}%
\column{E}{@{}>{\hspre}l<{\hspost}@{}}%
\>[5]{}\mathbf{class}\;\Conid{NumCat}\;\Varid{k}\;\Varid{a}\;\mathbf{where}{}\<[E]%
\\
\>[5]{}\hsindent{4}{}\<[9]%
\>[9]{}\Varid{negateC}\mathbin{::}\Varid{a}\mathbin{‘}\Varid{k}\mathbin{‘}\Varid{a}{}\<[E]%
\\
\>[5]{}\hsindent{4}{}\<[9]%
\>[9]{}\Varid{addC}\mathbin{::}(\Varid{a}\mathbin{×}\Varid{a})\mathbin{‘}\Varid{k}\mathbin{‘}\Varid{a}{}\<[E]%
\\
\>[5]{}\hsindent{4}{}\<[9]%
\>[9]{}\Varid{mulC}\mathbin{::}(\Varid{a}\mathbin{×}\Varid{a})\mathbin{‘}\Varid{k}\mathbin{‘}\Varid{a}{}\<[E]%
\\
\>[5]{}\hsindent{4}{}\<[9]%
\>[9]{}\mathbin{...}{}\<[E]%
\\[\blanklineskip]%
\>[5]{}\mathbf{instance}\;\Conid{Num}\;\Varid{a}\Rightarrow\Conid{NumCat}\;(\rightarrow )\;\Varid{a}\;\mathbf{where}{}\<[E]%
\\
\>[5]{}\hsindent{4}{}\<[9]%
\>[9]{}\Varid{negateC}\mathrel{=}\Varid{negate}{}\<[E]%
\\
\>[5]{}\hsindent{4}{}\<[9]%
\>[9]{}\Varid{addC}\mathrel{=}\Varid{uncurry}\;(\mathbin{+}){}\<[E]%
\\
\>[5]{}\hsindent{4}{}\<[9]%
\>[9]{}\Varid{mulC}\mathrel{=}\Varid{uncurry}\;(\mathbin{*}){}\<[E]%
\\
\>[5]{}\hsindent{4}{}\<[9]%
\>[9]{}\mathbin{...}{}\<[E]%
\ColumnHook
\end{hscode}\resethooks
\end{slide}

\begin{frame}
    \ensuremath{\mathcal{D}\;(\Varid{negate}\;\Varid{u})\mathrel{=}\Varid{negate}\;(\mathcal{D}\;\Varid{u})}\\
    \ensuremath{\mathcal{D}\;(\Varid{u}\mathbin{+}\Varid{v})\mathrel{=}\mathcal{D}\;\Varid{u}\mathbin{+}\mathcal{D}\;\Varid{v}}\\
    \ensuremath{\mathcal{D}\;(\Varid{u}\mathbin{*}\Varid{v})\mathrel{=}\Varid{u}\mathbin{*}\mathcal{D}\;\Varid{v}\mathbin{+}\Varid{v}\mathbin{*}\mathcal{D}\;\Varid{u}}\\
    \begin{itemize}
        \item
            Imprecise on the nature of u and v.
        \item
            A precise and simpler definition would be to differentiate the operations themselves.
    \end{itemize}
\end{frame}

\begin{frame}
\begin{hscode}\SaveRestoreHook
\column{B}{@{}>{\hspre}l<{\hspost}@{}}%
\column{5}{@{}>{\hspre}l<{\hspost}@{}}%
\column{9}{@{}>{\hspre}l<{\hspost}@{}}%
\column{E}{@{}>{\hspre}l<{\hspost}@{}}%
\>[5]{}\mathbf{class}\;\Conid{Scalable}\;\Varid{k}\;\Varid{a}\;\mathbf{where}{}\<[E]%
\\
\>[5]{}\hsindent{4}{}\<[9]%
\>[9]{}\Varid{scale}\mathbin{::}\Varid{a}\rightarrow (\Varid{a}\mathbin{‘}\Varid{k}\mathbin{‘}\Varid{a}){}\<[E]%
\\[\blanklineskip]%
\>[5]{}\mathbf{instance}\;\Conid{Num}\;\Varid{a}\Rightarrow\Conid{Scalable}\;(\rightarrow^+ )\;\Varid{a}\;\mathbf{where}{}\<[E]%
\\
\>[5]{}\hsindent{4}{}\<[9]%
\>[9]{}\Varid{scale}\;\Varid{a}\mathrel{=}\Conid{AddFun}\;(\lambda \Varid{da}\rightarrow \Varid{a}\mathbin{*}\Varid{da}){}\<[E]%
\\[\blanklineskip]%
\>[5]{}\mathbf{instance}\;\Conid{NumCat}\;\Conid{D}\;\mathbf{where}{}\<[E]%
\\
\>[5]{}\hsindent{4}{}\<[9]%
\>[9]{}\Varid{negateC}\mathrel{=}\Varid{linearD}\;\Varid{negateC}{}\<[E]%
\\
\>[5]{}\hsindent{4}{}\<[9]%
\>[9]{}\Varid{addC}\mathrel{=}\Varid{linearD}\;\Varid{addC}{}\<[E]%
\\
\>[5]{}\hsindent{4}{}\<[9]%
\>[9]{}\Varid{mulC}\mathrel{=}\Conid{D}\;(\lambda (\Varid{a},\Varid{b})\rightarrow (\Varid{a}\mathbin{*}\Varid{b},\Varid{scale}\;\Varid{b}\; \nabla \;\Varid{scale}\;\Varid{a})){}\<[E]%
\ColumnHook
\end{hscode}\resethooks
\end{frame}

\section{Generalizing Automatic Differentiation}
\begin{frame}{Generalizing Automatic Differentiation}
\begin{hscode}\SaveRestoreHook
\column{B}{@{}>{\hspre}l<{\hspost}@{}}%
\column{5}{@{}>{\hspre}l<{\hspost}@{}}%
\column{9}{@{}>{\hspre}l<{\hspost}@{}}%
\column{E}{@{}>{\hspre}l<{\hspost}@{}}%
\>[5]{}\mathbf{newtype}\;\Conid{D}\;\Varid{k}\;\Varid{a}\;\Varid{b}\mathrel{=}\Conid{D}\;(\Varid{a}\rightarrow \Varid{b} \times (\Varid{a}\mathbin{‘}\Varid{k}\mathbin{‘}\Varid{b})){}\<[E]%
\\[\blanklineskip]%
\>[5]{}\Varid{linearD}\mathbin{::}(\Varid{a}\rightarrow \Varid{b})\rightarrow (\Varid{a}\mathbin{‘}\Varid{k}\mathbin{‘}\Varid{b})\rightarrow \Conid{D}\;\Varid{k}\;\Varid{a}\;\Varid{b}{}\<[E]%
\\
\>[5]{}\Varid{linearD}\;\Varid{f}\;\Varid{f'}\mathrel{=}\Conid{D}\;(\lambda \Varid{a}\rightarrow (\Varid{f}\;\Varid{a},\Varid{f'})){}\<[E]%
\\[\blanklineskip]%
\>[5]{}\mathbf{instance}\;\Conid{Category}\;\Varid{k}\Rightarrow\Conid{Category}\;(\Conid{D}\;\Varid{k})\;\mathbf{where}{}\<[E]%
\\
\>[5]{}\hsindent{4}{}\<[9]%
\>[9]{}\mathbf{type}\;\Conid{Obj}\;(\Conid{D}\;\Varid{k})\mathrel{=}\Conid{Additive}\mathbin{∧}\Conid{Obj}\;\Varid{k}\mathbin{...}{}\<[E]%
\\[\blanklineskip]%
\>[5]{}\mathbf{instance}\;\Conid{Monoidal}\;\Varid{k}\Rightarrow\Conid{Monoidal}\;(\Conid{D}\;\Varid{k})\;\mathbf{where}\mathbin{...}{}\<[E]%
\\[\blanklineskip]%
\>[5]{}\mathbf{instance}\;\Conid{Cartesian}\;\Varid{k}\Rightarrow\Conid{Cartesian}\;(\Conid{D}\;\Varid{k})\;\mathbf{where}\mathbin{...}{}\<[E]%
\\[\blanklineskip]%
\>[5]{}\mathbf{instance}\;\Conid{Cocartesian}\;\Varid{k}\Rightarrow\Conid{Cocartesian}\;(\Conid{D}\;\Varid{k})\;\mathbf{where}{}\<[E]%
\\
\>[5]{}\hsindent{4}{}\<[9]%
\>[9]{}\Varid{inl}\mathrel{=}\Varid{linearD}\;\Varid{inlF}\;\Varid{inl}{}\<[E]%
\\
\>[5]{}\hsindent{4}{}\<[9]%
\>[9]{}\Varid{inr}\mathrel{=}\Varid{linearD}\;\Varid{inrF}\;\Varid{inr}{}\<[E]%
\\
\>[5]{}\hsindent{4}{}\<[9]%
\>[9]{}\Varid{jam}\mathrel{=}\Varid{linearD}\;\Varid{jamF}\;\Varid{jam}{}\<[E]%
\ColumnHook
\end{hscode}\resethooks
\end{frame}

\begin{frame}
\begin{hscode}\SaveRestoreHook
\column{B}{@{}>{\hspre}l<{\hspost}@{}}%
\column{5}{@{}>{\hspre}l<{\hspost}@{}}%
\column{9}{@{}>{\hspre}l<{\hspost}@{}}%
\column{E}{@{}>{\hspre}l<{\hspost}@{}}%
\>[5]{}\mathbf{instance}\;\Conid{Scalable}\;\Varid{k}\;\Varid{s}\mathbin{⇒}\Conid{NumCat}\;(\Conid{D}\;\Varid{k})\;\Varid{s}\;\mathbf{where}{}\<[E]%
\\
\>[5]{}\hsindent{4}{}\<[9]%
\>[9]{}\Varid{negateC}\mathrel{=}\Varid{linearD}\;\Varid{negateC}\;\Varid{negateC}{}\<[E]%
\\
\>[5]{}\hsindent{4}{}\<[9]%
\>[9]{}\Varid{addC}\mathrel{=}\Varid{linearD}\;\Varid{addC}\;\Varid{addC}{}\<[E]%
\\
\>[5]{}\hsindent{4}{}\<[9]%
\>[9]{}\Varid{mulC}\mathrel{=}\Conid{D}\;(\lambda (\Varid{a},\Varid{b})\rightarrow (\Varid{a}\mathbin{*}\Varid{b},\Varid{scale}\;\Varid{b}\; \nabla \;\Varid{scale}\;\Varid{a})){}\<[E]%
\ColumnHook
\end{hscode}\resethooks
\end{frame}

\section{Exemplos}
\begin{frame}{Exemplos}
\end{frame}

\section{Generalizar}
\begin{frame}{Generalizar}
\end{frame}


%============================EXEMPLO==========================
%\section{Introduction}
%
%\subsection[Short First Subsection Name]{First Subsection Name}
%
%\begin{frame}{Make Titles Informative. Use Uppercase Letters.}{Subtitles are optional.}
%  % - A title should summarize the slide in an understandable fashion
%  %   for anyone how does not follow everything on the slide itself.
%
%  \begin{itemize}
%  \item
%    Use \texttt{itemize} a lot.
%  \item
%    Use very short sentences or short phrases.
%  \end{itemize}
%\end{frame}
%
%\begin{frame}{Make Titles Informative.}
%
%  You can create overlays\dots
%  \begin{itemize}
%  \item using the \texttt{pause} command:
%    \begin{itemize}
%    \item
%      First item.
%      \pause
%    \item    
%      Second item.
%    \end{itemize}
%  \item
%    using overlay specifications:
%    \begin{itemize}
%    \item<3->
%      First item.
%    \item<4->
%      Second item.
%    \end{itemize}
%  \item
%    using the general \texttt{uncover} command:
%    \begin{itemize}
%      \uncover<5->{\item
%        First item.}
%      \uncover<6->{\item
%        Second item.}
%    \end{itemize}
%  \end{itemize}
%\end{frame}
%
%\begin{frame}{}
%\end{frame}
%
%\subsection{Second Subsection}
%
%\begin{frame}{Make Titles Informative.}
%\end{frame}
%
%
%
%\section{Summary}
%
%\subsection{coisas1}
%
%\begin{frame}{Summary}
%
%  % Keep the summary *very short*.
%  \begin{itemize}
%  \item
%    The \alert{first main message} of your talk in one or two lines.
%  \item
%    The \alert{second main message} of your talk in one or two lines.
%  \item
%    Perhaps a \alert{third message}, but not more than that.
%  \end{itemize}
%  
%  % The following outlook is optional.
%  \vskip0pt plus.5fill
%  \begin{itemize}
%  \item
%    Outlook
%    \begin{itemize}
%    \item
%      Something you haven't solved.
%    \item
%      Something else you haven't solved.
%    \end{itemize}
%  \end{itemize}
%\end{frame}
%
%\subsection{coisas2}
%\begin{frame}{coisas2}
%
%
%\end{frame}
%=================================================================

\end{document}


