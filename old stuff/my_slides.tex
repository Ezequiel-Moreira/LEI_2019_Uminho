\documentclass{beamer}

\mode<presentation>
{
  \usetheme{Warsaw}

  \setbeamercovered{transparent}
}


\usepackage[english]{babel}
\usepackage[utf8x]{inputenc}
%\usepackage[latin1]{inputenc}

\usepackage{times}
\usepackage[T1]{fontenc}
% Note that the encoding and the font should match. If T1
% does not look nice, try deleting the line with the fontenc.

%---------- lhs2tex ---------------------------------
%% ODER: format ==         = "\mathrel{==}"
%% ODER: format /=         = "\neq "
%
%
\makeatletter
\@ifundefined{lhs2tex.lhs2tex.sty.read}%
  {\@namedef{lhs2tex.lhs2tex.sty.read}{}%
   \newcommand\SkipToFmtEnd{}%
   \newcommand\EndFmtInput{}%
   \long\def\SkipToFmtEnd#1\EndFmtInput{}%
  }\SkipToFmtEnd

\newcommand\ReadOnlyOnce[1]{\@ifundefined{#1}{\@namedef{#1}{}}\SkipToFmtEnd}
\usepackage{amstext}
\usepackage{amssymb}
\usepackage{stmaryrd}
\DeclareFontFamily{OT1}{cmtex}{}
\DeclareFontShape{OT1}{cmtex}{m}{n}
  {<5><6><7><8>cmtex8
   <9>cmtex9
   <10><10.95><12><14.4><17.28><20.74><24.88>cmtex10}{}
\DeclareFontShape{OT1}{cmtex}{m}{it}
  {<-> ssub * cmtt/m/it}{}
\newcommand{\texfamily}{\fontfamily{cmtex}\selectfont}
\DeclareFontShape{OT1}{cmtt}{bx}{n}
  {<5><6><7><8>cmtt8
   <9>cmbtt9
   <10><10.95><12><14.4><17.28><20.74><24.88>cmbtt10}{}
\DeclareFontShape{OT1}{cmtex}{bx}{n}
  {<-> ssub * cmtt/bx/n}{}
\newcommand{\tex}[1]{\text{\texfamily#1}}	% NEU

\newcommand{\Sp}{\hskip.33334em\relax}


\newcommand{\Conid}[1]{\mathit{#1}}
\newcommand{\Varid}[1]{\mathit{#1}}
\newcommand{\anonymous}{\kern0.06em \vbox{\hrule\@width.5em}}
\newcommand{\plus}{\mathbin{+\!\!\!+}}
\newcommand{\bind}{\mathbin{>\!\!\!>\mkern-6.7mu=}}
\newcommand{\rbind}{\mathbin{=\mkern-6.7mu<\!\!\!<}}% suggested by Neil Mitchell
\newcommand{\sequ}{\mathbin{>\!\!\!>}}
\renewcommand{\leq}{\leqslant}
\renewcommand{\geq}{\geqslant}
\usepackage{polytable}

%mathindent has to be defined
\@ifundefined{mathindent}%
  {\newdimen\mathindent\mathindent\leftmargini}%
  {}%

\def\resethooks{%
  \global\let\SaveRestoreHook\empty
  \global\let\ColumnHook\empty}
\newcommand*{\savecolumns}[1][default]%
  {\g@addto@macro\SaveRestoreHook{\savecolumns[#1]}}
\newcommand*{\restorecolumns}[1][default]%
  {\g@addto@macro\SaveRestoreHook{\restorecolumns[#1]}}
\newcommand*{\aligncolumn}[2]%
  {\g@addto@macro\ColumnHook{\column{#1}{#2}}}

\resethooks

\newcommand{\onelinecommentchars}{\quad-{}- }
\newcommand{\commentbeginchars}{\enskip\{-}
\newcommand{\commentendchars}{-\}\enskip}

\newcommand{\visiblecomments}{%
  \let\onelinecomment=\onelinecommentchars
  \let\commentbegin=\commentbeginchars
  \let\commentend=\commentendchars}

\newcommand{\invisiblecomments}{%
  \let\onelinecomment=\empty
  \let\commentbegin=\empty
  \let\commentend=\empty}

\visiblecomments

\newlength{\blanklineskip}
\setlength{\blanklineskip}{0.66084ex}

\newcommand{\hsindent}[1]{\quad}% default is fixed indentation
\let\hspre\empty
\let\hspost\empty
\newcommand{\NB}{\textbf{NB}}
\newcommand{\Todo}[1]{$\langle$\textbf{To do:}~#1$\rangle$}

\EndFmtInput
\makeatother
%
%
%
%
%
%
% This package provides two environments suitable to take the place
% of hscode, called "plainhscode" and "arrayhscode". 
%
% The plain environment surrounds each code block by vertical space,
% and it uses \abovedisplayskip and \belowdisplayskip to get spacing
% similar to formulas. Note that if these dimensions are changed,
% the spacing around displayed math formulas changes as well.
% All code is indented using \leftskip.
%
% Changed 19.08.2004 to reflect changes in colorcode. Should work with
% CodeGroup.sty.
%
\ReadOnlyOnce{polycode.fmt}%
\makeatletter

\newcommand{\hsnewpar}[1]%
  {{\parskip=0pt\parindent=0pt\par\vskip #1\noindent}}

% can be used, for instance, to redefine the code size, by setting the
% command to \small or something alike
\newcommand{\hscodestyle}{}

% The command \sethscode can be used to switch the code formatting
% behaviour by mapping the hscode environment in the subst directive
% to a new LaTeX environment.

\newcommand{\sethscode}[1]%
  {\expandafter\let\expandafter\hscode\csname #1\endcsname
   \expandafter\let\expandafter\endhscode\csname end#1\endcsname}

% "compatibility" mode restores the non-polycode.fmt layout.

\newenvironment{compathscode}%
  {\par\noindent
   \advance\leftskip\mathindent
   \hscodestyle
   \let\\=\@normalcr
   \let\hspre\(\let\hspost\)%
   \pboxed}%
  {\endpboxed\)%
   \par\noindent
   \ignorespacesafterend}

\newcommand{\compaths}{\sethscode{compathscode}}

% "plain" mode is the proposed default.
% It should now work with \centering.
% This required some changes. The old version
% is still available for reference as oldplainhscode.

\newenvironment{plainhscode}%
  {\hsnewpar\abovedisplayskip
   \advance\leftskip\mathindent
   \hscodestyle
   \let\hspre\(\let\hspost\)%
   \pboxed}%
  {\endpboxed%
   \hsnewpar\belowdisplayskip
   \ignorespacesafterend}

\newenvironment{oldplainhscode}%
  {\hsnewpar\abovedisplayskip
   \advance\leftskip\mathindent
   \hscodestyle
   \let\\=\@normalcr
   \(\pboxed}%
  {\endpboxed\)%
   \hsnewpar\belowdisplayskip
   \ignorespacesafterend}

% Here, we make plainhscode the default environment.

\newcommand{\plainhs}{\sethscode{plainhscode}}
\newcommand{\oldplainhs}{\sethscode{oldplainhscode}}
\plainhs

% The arrayhscode is like plain, but makes use of polytable's
% parray environment which disallows page breaks in code blocks.

\newenvironment{arrayhscode}%
  {\hsnewpar\abovedisplayskip
   \advance\leftskip\mathindent
   \hscodestyle
   \let\\=\@normalcr
   \(\parray}%
  {\endparray\)%
   \hsnewpar\belowdisplayskip
   \ignorespacesafterend}

\newcommand{\arrayhs}{\sethscode{arrayhscode}}

% The mathhscode environment also makes use of polytable's parray 
% environment. It is supposed to be used only inside math mode 
% (I used it to typeset the type rules in my thesis).

\newenvironment{mathhscode}%
  {\parray}{\endparray}

\newcommand{\mathhs}{\sethscode{mathhscode}}

% texths is similar to mathhs, but works in text mode.

\newenvironment{texthscode}%
  {\(\parray}{\endparray\)}

\newcommand{\texths}{\sethscode{texthscode}}

% The framed environment places code in a framed box.

\def\codeframewidth{\arrayrulewidth}
\RequirePackage{calc}

\newenvironment{framedhscode}%
  {\parskip=\abovedisplayskip\par\noindent
   \hscodestyle
   \arrayrulewidth=\codeframewidth
   \tabular{@{}|p{\linewidth-2\arraycolsep-2\arrayrulewidth-2pt}|@{}}%
   \hline\framedhslinecorrect\\{-1.5ex}%
   \let\endoflinesave=\\
   \let\\=\@normalcr
   \(\pboxed}%
  {\endpboxed\)%
   \framedhslinecorrect\endoflinesave{.5ex}\hline
   \endtabular
   \parskip=\belowdisplayskip\par\noindent
   \ignorespacesafterend}

\newcommand{\framedhslinecorrect}[2]%
  {#1[#2]}

\newcommand{\framedhs}{\sethscode{framedhscode}}

% The inlinehscode environment is an experimental environment
% that can be used to typeset displayed code inline.

\newenvironment{inlinehscode}%
  {\(\def\column##1##2{}%
   \let\>\undefined\let\<\undefined\let\\\undefined
   \newcommand\>[1][]{}\newcommand\<[1][]{}\newcommand\\[1][]{}%
   \def\fromto##1##2##3{##3}%
   \def\nextline{}}{\) }%

\newcommand{\inlinehs}{\sethscode{inlinehscode}}

% The joincode environment is a separate environment that
% can be used to surround and thereby connect multiple code
% blocks.

\newenvironment{joincode}%
  {\let\orighscode=\hscode
   \let\origendhscode=\endhscode
   \def\endhscode{\def\hscode{\endgroup\def\@currenvir{hscode}\\}\begingroup}
   %\let\SaveRestoreHook=\empty
   %\let\ColumnHook=\empty
   %\let\resethooks=\empty
   \orighscode\def\hscode{\endgroup\def\@currenvir{hscode}}}%
  {\origendhscode
   \global\let\hscode=\orighscode
   \global\let\endhscode=\origendhscode}%

\makeatother
\EndFmtInput
%
%----------------------------------------------------

\newenvironment{slide}[1]{\begin{frame}\frametitle{#1}}{\end{frame}}

\title
{...Machine Learning...}

\author[Artur, Ezequiel, Nelson] 
{Artur \and Ezequiel \and Nelson}

\institute
{Universidade do Minho}

\date
{26 de Abril}

\subject{Talks}

% If you have a file called "university-logo-filename.xxx", where xxx
% is a graphic format that can be processed by latex or pdflatex,
% resp., then you can add a logo as follows:

% \pgfdeclareimage[height=0.5cm]{university-logo}{university-logo-filename}
% \logo{\pgfuseimage{university-logo}}

% Delete this, if you do not want the table of contents to pop up at
% the beginning of each subsection:

\AtBeginSubsection[]
{
  \begin{frame}<beamer>{Outline}
    \tableofcontents[currentsection,currentsubsection]
  \end{frame}
}


% If you wish to uncover everything in a step-wise fashion, uncomment
% the following command: 

%\beamerdefaultoverlayspecification{<+->}


\begin{document}
\begin{slide}{Instance of \ensuremath{\rightarrow^+ }}
\begin{hscode}\SaveRestoreHook
\column{B}{@{}>{\hspre}l<{\hspost}@{}}%
\column{4}{@{}>{\hspre}l<{\hspost}@{}}%
\column{E}{@{}>{\hspre}l<{\hspost}@{}}%
\>[B]{}\mathbf{newtype}\;\Varid{a}\rightarrow^+ \Varid{b}\mathrel{=}\Conid{AddFun}\;(\Varid{a}\to \Varid{b}){}\<[E]%
\\[\blanklineskip]%
\>[B]{}\mathbf{instance}\;\Conid{Category}\;(\rightarrow^+ )\;\mathbf{where}{}\<[E]%
\\
\>[B]{}\hsindent{4}{}\<[4]%
\>[4]{}\mathbf{type}\;\Conid{Obj}\;(\rightarrow^+ )\mathrel{=}\Conid{Additive}{}\<[E]%
\\
\>[B]{}\hsindent{4}{}\<[4]%
\>[4]{}\Varid{id}\mathrel{=}\Conid{AddFun}\;\Varid{id}{}\<[E]%
\\
\>[B]{}\hsindent{4}{}\<[4]%
\>[4]{}\Conid{AddFun}\;\Varid{g}\mathbin{\circ}\Conid{AddFun}\;\Varid{f}\mathrel{=}\Conid{AddFun}\;(\Varid{g}\mathbin{\circ}\Varid{f}){}\<[E]%
\\[\blanklineskip]%
\>[B]{}\mathbf{instance}\;\Conid{Monoidal}\;(\rightarrow^+ )\;\mathbf{where}{}\<[E]%
\\
\>[B]{}\hsindent{4}{}\<[4]%
\>[4]{}\Conid{AddFun}\;\Varid{f}\mathbin{×}\Conid{AddFun}\;\Varid{g}\mathrel{=}\Conid{AddFun}\;(\Varid{f}\mathbin{×}\Varid{g}){}\<[E]%
\\[\blanklineskip]%
\>[B]{}\mathbf{instance}\;\Conid{Cartesian}\;(\rightarrow^+ )\;\mathbf{where}{}\<[E]%
\\
\>[B]{}\hsindent{4}{}\<[4]%
\>[4]{}\Varid{exl}\mathrel{=}\Conid{AddFun}\;\Varid{exl}{}\<[E]%
\\
\>[B]{}\hsindent{4}{}\<[4]%
\>[4]{}\Varid{exr}\mathrel{=}\Conid{AddFun}\;\Varid{exr}{}\<[E]%
\\
\>[B]{}\hsindent{4}{}\<[4]%
\>[4]{}\Varid{dup}\mathrel{=}\Conid{AddFun}\;\Varid{dup}{}\<[E]%
\ColumnHook
\end{hscode}\resethooks
\end{slide}

\begin{slide}{Fork and Join}
    \begin{itemize}
        \item
            \ensuremath{ \bigtriangledown \mathbin{::}\Conid{Cartesian}\;\Varid{k}\Rightarrow (\Varid{a}\;\text{\tt 'k'}\;\Varid{c})\to (\Varid{a}\;\text{\tt 'k'}\;\Varid{d})\to (\Varid{a}\;\text{\tt 'k'}\;(\Varid{c}\mathbin{×}\Varid{d}))}
        \item
            \ensuremath{ \bigtriangleup \mathbin{::}\Conid{Cartesian}\;\Varid{k}\Rightarrow (\Varid{c}\;\text{\tt 'k'}\;\Varid{a})\to (\Varid{d}\;\text{\tt 'k'}\;\Varid{a})\to ((\Varid{c}\mathbin{+}\Varid{d})\;\text{\tt 'k'}\;\Varid{a})}
    \end{itemize}
\end{slide}

\begin{frame}
  \titlepage
\end{frame}

\begin{frame}{Indice}
  \tableofcontents
\end{frame}

% - Exactly two or three sections (other than the summary).
% - At *most* three subsections per section.
% - Talk about 30s to 2min per frame. So there should be between about
%   15 and 30 frames, all told.

\section{Nelson}

\begin{frame}{titulo}
\end{frame}


\section{Ezequiel}

\begin{frame}{titulo}
\end{frame}

\section{Fork e Join}
\begin{frame}{Fork e Join}
    \begin{itemize}
        \item
            $(\bigtriangleup)$ :: Cartesian k $\Rightarrow$ (a 'k' c) $\to$ (a 'k' d) $\to$ (a 'k' (c $\times$ d))
        \item
            $(\bigtriangledown)$ :: Cartesian k $\Rightarrow$ (c 'k' a) $\to$ (d 'k' a) $\to$ ((c $\times$ d) 'k' a)
    \end{itemize}
\end{frame}
\begin{frame}{instancia de $\to^+$}
    \textbf{newtype} a $\to^+$ b = AddFun (a → b)\\
    \vspace{2mm} 
    \textbf{instance} Category ($\to^+$) \textbf{where}\\
        \hspace{1cm}type Obj ($\to^+$) = Additive\\
        \hspace{1cm}id = AddFun id\\
        \hspace{1cm}AddFun g $\circ$ AddFun f = AddFun (g $\circ$ f )\\
    \vspace{2mm} 
    \textbf{instance} Monoidal ($\to^+$) \textbf{where}\\
        \hspace{1cm}AddFun f × AddFun g = AddFun (f × g)\\
    \vspace{2mm} 
    \textbf{instance} Cartesian ($\to^+$) \textbf{where}\\
        \hspace{1cm}exl = AddFun exl\\
        \hspace{1cm}exr = AddFun exr\\
        \hspace{1cm}dup = AddFun dup\\
\end{frame}
\begin{frame}{instancia de $\to^+$}
    \textbf{instance} Cocartesian ($\to^+$) \textbf{where}\\
        \hspace{1cm} inl = AddFun inlF\\
        \hspace{1cm} inr = AddFun inrF\\
        \hspace{1cm} jam = AddFun jamF\\
    \vspace{2mm} 
    inlF :: Additive b ⇒ a → a × b\\ 
    inrF :: Additive a ⇒ b → a × b\\
    jamF :: Additive a ⇒ a × a → a\\
    \vspace{2mm} 
    inlF = $\lambda$a → (a, 0)    \\ 
    inrF = $\lambda$b → (0, b)    \\
    jamF = $\lambda$(a, b) → a + b\\ 
\end{frame}

\section{Operacoes Numericas}
\begin{frame}{definição de NumCat}
    \textbf{class} NumCat k a \textbf{where}\\
        \hspace{1cm}negateC :: a ‘k‘ a\\
        \hspace{1cm}addC :: (a × a) ‘k‘ a\\
        \hspace{1cm}mulC :: (a × a) ‘k‘ a\\
        \hspace{1cm}...\\
    \vspace{2mm} 
    \textbf{instance} Num a ⇒ NumCat (→) a \textbf{where}\\
        \hspace{1cm}negateC = negate\\
        \hspace{1cm}addC = uncurry (+)\\
        \hspace{1cm}mulC = uncurry (·)\\
        \hspace{1cm}...\\
\end{frame}
\begin{frame}
    D (negate u) = negate (D u)\\
    D (u + v) = D u + D v\\
    D (u · v) = u · D v + v · D u\\
    \begin{itemize}
        \item
            Impreciso na natureza de u e v.
        \item
            Algo mais preciso seria defenir a diferenciação das operações em si.
    \end{itemize}
\end{frame}
\begin{frame}
    \textbf{class} Scalable k a \textbf{where}\\
        \hspace{1cm}scale :: a → (a ‘k‘ a)\\
    \vspace{2mm} 
    \textbf{instance} Num a ⇒ Scalable ($\to^+$) a \textbf{where}\\
        \hspace{1cm}scale a = AddFun ($\lambda$da → a · da)\\
    \vspace{5mm} 
    \textbf{instance} NumCat D \textbf{where}\\
        \hspace{1cm}negateC = linearD negateC\\
        \hspace{1cm}addC = linearD addC\\
        \hspace{1cm}mulC = D ($\lambda$(a, b) → (a · b, scale b $\bigtriangledown$ scale a))\\
\end{frame}

\section{Generalizing Automatic Differentiation}
\begin{frame}{Generalizing Automatic Differentiation}
    newtype $D_k$ a b = D (a → b × (a ‘k‘ b))\\
    \vspace{2mm} 
    linearD :: (a → b) → (a ‘k‘ b) → $D_k$ a b\\
    linearD f f'= D ($\lambda$a → (f a, f'))\\
    \vspace{2mm} 
    \textbf{instance} Category k ⇒ Category $D_k$ \textbf{where}\\
        \hspace{1cm}type Obj $D_k$ = Additive ∧ Obj k ...\\
    \vspace{2mm} 
    \textbf{instance} Monoidal k ⇒ Monoidal $D_k$ \textbf{where} ...\\
    \vspace{2mm} 
    \textbf{instance} Cartesian k ⇒ Cartesian $D_k$ \textbf{where} ...\\
    \vspace{2mm} 
    \textbf{instance} Cocartesian k ⇒ Cocartesian $D_k$ \textbf{where}\\
        \hspace{1cm}inl = linearD inlF inl\\
        \hspace{1cm}inr = linearD inrF inr\\
        \hspace{1cm}jam = linearD jamF jam\\
    \vspace{2mm} 
\end{frame}

\begin{frame}
    \textbf{instance} Scalable k s ⇒ NumCat $D_k$ s \textbf{where}\\
        \hspace{1cm}negateC = linearD negateC negateC\\
        \hspace{1cm}addC = linearD addC addC\\
        \hspace{1cm}mulC = D ($\lambda$(a, b) → (a · b, scale b $\bigtriangledown$ scale a))\\
\end{frame}

\section{Exemplos}
\begin{frame}{Exemplos}
\end{frame}

\section{Generalizar}
\begin{frame}{Generalizar}
\end{frame}

%============================EXEMPLO==========================
%\section{Introduction}
%
%\subsection[Short First Subsection Name]{First Subsection Name}
%
%\begin{frame}{Make Titles Informative. Use Uppercase Letters.}{Subtitles are optional.}
%  % - A title should summarize the slide in an understandable fashion
%  %   for anyone how does not follow everything on the slide itself.
%
%  \begin{itemize}
%  \item
%    Use \texttt{itemize} a lot.
%  \item
%    Use very short sentences or short phrases.
%  \end{itemize}
%\end{frame}
%
%\begin{frame}{Make Titles Informative.}
%
%  You can create overlays\dots
%  \begin{itemize}
%  \item using the \texttt{pause} command:
%    \begin{itemize}
%    \item
%      First item.
%      \pause
%    \item    
%      Second item.
%    \end{itemize}
%  \item
%    using overlay specifications:
%    \begin{itemize}
%    \item<3->
%      First item.
%    \item<4->
%      Second item.
%    \end{itemize}
%  \item
%    using the general \texttt{uncover} command:
%    \begin{itemize}
%      \uncover<5->{\item
%        First item.}
%      \uncover<6->{\item
%        Second item.}
%    \end{itemize}
%  \end{itemize}
%\end{frame}
%
%\begin{frame}{}
%\end{frame}
%
%\subsection{Second Subsection}
%
%\begin{frame}{Make Titles Informative.}
%\end{frame}
%
%
%
%\section{Summary}
%
%\subsection{coisas1}
%
%\begin{frame}{Summary}
%
%  % Keep the summary *very short*.
%  \begin{itemize}
%  \item
%    The \alert{first main message} of your talk in one or two lines.
%  \item
%    The \alert{second main message} of your talk in one or two lines.
%  \item
%    Perhaps a \alert{third message}, but not more than that.
%  \end{itemize}
%  
%  % The following outlook is optional.
%  \vskip0pt plus.5fill
%  \begin{itemize}
%  \item
%    Outlook
%    \begin{itemize}
%    \item
%      Something you haven't solved.
%    \item
%      Something else you haven't solved.
%    \end{itemize}
%  \end{itemize}
%\end{frame}
%
%\subsection{coisas2}
%\begin{frame}{coisas2}
%
%
%\end{frame}
%=================================================================

\end{document}


