\documentclass{beamer}

\mode<presentation>
{
  \usetheme{Singapore}

  \setbeamercovered{transparent}
}


\usepackage[english]{babel}
\usepackage[utf8x]{inputenc}
%\usepackage[latin1]{inputenc}

\usepackage{times}
\usepackage[T1]{fontenc}
% Note that the encoding and the font should match. If T1
% does not look nice, try deleting the line with the fontenc.
\usepackage[all]{xy}

\newtheorem{teor}{Theorem}[section]
\newtheorem{coro}{Corollary}[section]
\theoremstyle{definition}
\newtheorem{defi}{Definition}[section]
\theoremstyle{definition}
\newtheorem{exemplo}{example}[section]
\theoremstyle{theorem}
\newtheorem{propo}{Proposicion}[section]
\newtheorem{nota}{Note}[section]
\def\N{\mathbb{N}}
\def\R{\mathbb{R}}
\def\C{\mathbb{C}}
\newcommand{\norm}[1]{\left\lVert#1\right\rVert}

%---------- lhs2tex ---------------------------------
%% ODER: format ==         = "\mathrel{==}"
%% ODER: format /=         = "\neq "
%
%
\makeatletter
\@ifundefined{lhs2tex.lhs2tex.sty.read}%
  {\@namedef{lhs2tex.lhs2tex.sty.read}{}%
   \newcommand\SkipToFmtEnd{}%
   \newcommand\EndFmtInput{}%
   \long\def\SkipToFmtEnd#1\EndFmtInput{}%
  }\SkipToFmtEnd

\newcommand\ReadOnlyOnce[1]{\@ifundefined{#1}{\@namedef{#1}{}}\SkipToFmtEnd}
\usepackage{amstext}
\usepackage{amssymb}
\usepackage{stmaryrd}
\DeclareFontFamily{OT1}{cmtex}{}
\DeclareFontShape{OT1}{cmtex}{m}{n}
  {<5><6><7><8>cmtex8
   <9>cmtex9
   <10><10.95><12><14.4><17.28><20.74><24.88>cmtex10}{}
\DeclareFontShape{OT1}{cmtex}{m}{it}
  {<-> ssub * cmtt/m/it}{}
\newcommand{\texfamily}{\fontfamily{cmtex}\selectfont}
\DeclareFontShape{OT1}{cmtt}{bx}{n}
  {<5><6><7><8>cmtt8
   <9>cmbtt9
   <10><10.95><12><14.4><17.28><20.74><24.88>cmbtt10}{}
\DeclareFontShape{OT1}{cmtex}{bx}{n}
  {<-> ssub * cmtt/bx/n}{}
\newcommand{\tex}[1]{\text{\texfamily#1}}	% NEU

\newcommand{\Sp}{\hskip.33334em\relax}


\newcommand{\Conid}[1]{\mathit{#1}}
\newcommand{\Varid}[1]{\mathit{#1}}
\newcommand{\anonymous}{\kern0.06em \vbox{\hrule\@width.5em}}
\newcommand{\plus}{\mathbin{+\!\!\!+}}
\newcommand{\bind}{\mathbin{>\!\!\!>\mkern-6.7mu=}}
\newcommand{\rbind}{\mathbin{=\mkern-6.7mu<\!\!\!<}}% suggested by Neil Mitchell
\newcommand{\sequ}{\mathbin{>\!\!\!>}}
\renewcommand{\leq}{\leqslant}
\renewcommand{\geq}{\geqslant}
\usepackage{polytable}

%mathindent has to be defined
\@ifundefined{mathindent}%
  {\newdimen\mathindent\mathindent\leftmargini}%
  {}%

\def\resethooks{%
  \global\let\SaveRestoreHook\empty
  \global\let\ColumnHook\empty}
\newcommand*{\savecolumns}[1][default]%
  {\g@addto@macro\SaveRestoreHook{\savecolumns[#1]}}
\newcommand*{\restorecolumns}[1][default]%
  {\g@addto@macro\SaveRestoreHook{\restorecolumns[#1]}}
\newcommand*{\aligncolumn}[2]%
  {\g@addto@macro\ColumnHook{\column{#1}{#2}}}

\resethooks

\newcommand{\onelinecommentchars}{\quad-{}- }
\newcommand{\commentbeginchars}{\enskip\{-}
\newcommand{\commentendchars}{-\}\enskip}

\newcommand{\visiblecomments}{%
  \let\onelinecomment=\onelinecommentchars
  \let\commentbegin=\commentbeginchars
  \let\commentend=\commentendchars}

\newcommand{\invisiblecomments}{%
  \let\onelinecomment=\empty
  \let\commentbegin=\empty
  \let\commentend=\empty}

\visiblecomments

\newlength{\blanklineskip}
\setlength{\blanklineskip}{0.66084ex}

\newcommand{\hsindent}[1]{\quad}% default is fixed indentation
\let\hspre\empty
\let\hspost\empty
\newcommand{\NB}{\textbf{NB}}
\newcommand{\Todo}[1]{$\langle$\textbf{To do:}~#1$\rangle$}

\EndFmtInput
\makeatother
%
%
%
%
%
%
% This package provides two environments suitable to take the place
% of hscode, called "plainhscode" and "arrayhscode". 
%
% The plain environment surrounds each code block by vertical space,
% and it uses \abovedisplayskip and \belowdisplayskip to get spacing
% similar to formulas. Note that if these dimensions are changed,
% the spacing around displayed math formulas changes as well.
% All code is indented using \leftskip.
%
% Changed 19.08.2004 to reflect changes in colorcode. Should work with
% CodeGroup.sty.
%
\ReadOnlyOnce{polycode.fmt}%
\makeatletter

\newcommand{\hsnewpar}[1]%
  {{\parskip=0pt\parindent=0pt\par\vskip #1\noindent}}

% can be used, for instance, to redefine the code size, by setting the
% command to \small or something alike
\newcommand{\hscodestyle}{}

% The command \sethscode can be used to switch the code formatting
% behaviour by mapping the hscode environment in the subst directive
% to a new LaTeX environment.

\newcommand{\sethscode}[1]%
  {\expandafter\let\expandafter\hscode\csname #1\endcsname
   \expandafter\let\expandafter\endhscode\csname end#1\endcsname}

% "compatibility" mode restores the non-polycode.fmt layout.

\newenvironment{compathscode}%
  {\par\noindent
   \advance\leftskip\mathindent
   \hscodestyle
   \let\\=\@normalcr
   \let\hspre\(\let\hspost\)%
   \pboxed}%
  {\endpboxed\)%
   \par\noindent
   \ignorespacesafterend}

\newcommand{\compaths}{\sethscode{compathscode}}

% "plain" mode is the proposed default.
% It should now work with \centering.
% This required some changes. The old version
% is still available for reference as oldplainhscode.

\newenvironment{plainhscode}%
  {\hsnewpar\abovedisplayskip
   \advance\leftskip\mathindent
   \hscodestyle
   \let\hspre\(\let\hspost\)%
   \pboxed}%
  {\endpboxed%
   \hsnewpar\belowdisplayskip
   \ignorespacesafterend}

\newenvironment{oldplainhscode}%
  {\hsnewpar\abovedisplayskip
   \advance\leftskip\mathindent
   \hscodestyle
   \let\\=\@normalcr
   \(\pboxed}%
  {\endpboxed\)%
   \hsnewpar\belowdisplayskip
   \ignorespacesafterend}

% Here, we make plainhscode the default environment.

\newcommand{\plainhs}{\sethscode{plainhscode}}
\newcommand{\oldplainhs}{\sethscode{oldplainhscode}}
\plainhs

% The arrayhscode is like plain, but makes use of polytable's
% parray environment which disallows page breaks in code blocks.

\newenvironment{arrayhscode}%
  {\hsnewpar\abovedisplayskip
   \advance\leftskip\mathindent
   \hscodestyle
   \let\\=\@normalcr
   \(\parray}%
  {\endparray\)%
   \hsnewpar\belowdisplayskip
   \ignorespacesafterend}

\newcommand{\arrayhs}{\sethscode{arrayhscode}}

% The mathhscode environment also makes use of polytable's parray 
% environment. It is supposed to be used only inside math mode 
% (I used it to typeset the type rules in my thesis).

\newenvironment{mathhscode}%
  {\parray}{\endparray}

\newcommand{\mathhs}{\sethscode{mathhscode}}

% texths is similar to mathhs, but works in text mode.

\newenvironment{texthscode}%
  {\(\parray}{\endparray\)}

\newcommand{\texths}{\sethscode{texthscode}}

% The framed environment places code in a framed box.

\def\codeframewidth{\arrayrulewidth}
\RequirePackage{calc}

\newenvironment{framedhscode}%
  {\parskip=\abovedisplayskip\par\noindent
   \hscodestyle
   \arrayrulewidth=\codeframewidth
   \tabular{@{}|p{\linewidth-2\arraycolsep-2\arrayrulewidth-2pt}|@{}}%
   \hline\framedhslinecorrect\\{-1.5ex}%
   \let\endoflinesave=\\
   \let\\=\@normalcr
   \(\pboxed}%
  {\endpboxed\)%
   \framedhslinecorrect\endoflinesave{.5ex}\hline
   \endtabular
   \parskip=\belowdisplayskip\par\noindent
   \ignorespacesafterend}

\newcommand{\framedhslinecorrect}[2]%
  {#1[#2]}

\newcommand{\framedhs}{\sethscode{framedhscode}}

% The inlinehscode environment is an experimental environment
% that can be used to typeset displayed code inline.

\newenvironment{inlinehscode}%
  {\(\def\column##1##2{}%
   \let\>\undefined\let\<\undefined\let\\\undefined
   \newcommand\>[1][]{}\newcommand\<[1][]{}\newcommand\\[1][]{}%
   \def\fromto##1##2##3{##3}%
   \def\nextline{}}{\) }%

\newcommand{\inlinehs}{\sethscode{inlinehscode}}

% The joincode environment is a separate environment that
% can be used to surround and thereby connect multiple code
% blocks.

\newenvironment{joincode}%
  {\let\orighscode=\hscode
   \let\origendhscode=\endhscode
   \def\endhscode{\def\hscode{\endgroup\def\@currenvir{hscode}\\}\begingroup}
   %\let\SaveRestoreHook=\empty
   %\let\ColumnHook=\empty
   %\let\resethooks=\empty
   \orighscode\def\hscode{\endgroup\def\@currenvir{hscode}}}%
  {\origendhscode
   \global\let\hscode=\orighscode
   \global\let\endhscode=\origendhscode}%

\makeatother
\EndFmtInput
%
%----------------------------------------------------

\newenvironment{slide}[1]{\begin{frame}\frametitle{#1}}{\end{frame}}

\title
{Simple essence of AD}

\author[Artur, Ezequiel, Nelson] 
{Artur \and Ezequiel \and Nelson}

\institute
{Universidade do Minho}

\date
{29 de Maio}

\subject{Talks}

% If you have a file called "university-logo-filename.xxx", where xxx
% is a graphic format that can be processed by latex or pdflatex,
% resp., then you can add a logo as follows:

% \pgfdeclareimage[height=0.5cm]{university-logo}{university-logo-filename}
% \logo{\pgfuseimage{university-logo}}

% Delete this, if you do not want the table of contents to pop up at
% the beginning of each subsection:

\AtBeginSubsection[]
{
  \begin{frame}<beamer>{Outline}
    \tableofcontents[currentsection,currentsubsection]
  \end{frame}
}


% If you wish to uncover everything in a step-wise fashion, uncomment
% the following command: 

%\beamerdefaultoverlayspecification{<+->}


\begin{document}

\begin{frame}
  \titlepage
\end{frame}

\begin{frame}{Index}
  \tableofcontents
\end{frame}




\section{Relembrar}

\begin{frame}{Relembrar}
Documento estudado: "The simple essence of automatic differentiation"([Elliott 2018]\cite{Elliott:2018})

Objetivo: Estudo e implementação de um algoritmo AD genérico

\end{frame}





\section{Conceitos base}

\begin{frame}{Definição de \ensuremath{\mathcal{D}} e conversão em \ensuremath{\mathcal{D}^{+}}}
\ensuremath{\mathcal{D}} - aproximação linear de uma função

\begin{block}{Definição}
  Seja $f::a \to b$ uma função onde $a$ e $b$ são espaços vetoriais sobre um corpo comum.
  A primeira definição de derivada é:
	\begin{hscode}\SaveRestoreHook
\column{B}{@{}>{\hspre}l<{\hspost}@{}}%
\column{9}{@{}>{\hspre}l<{\hspost}@{}}%
\column{E}{@{}>{\hspre}l<{\hspost}@{}}%
\>[9]{}\mathcal{D}\mathbin{::}(\Varid{a}\rightarrow \Varid{b})\rightarrow (\Varid{a}\rightarrow (\Varid{a}\;\multimap \;\Varid{b})){}\<[E]%
\ColumnHook
\end{hscode}\resethooks
\end{block}

\ensuremath{\mathcal{D}^{+}} - versão de D com composição eficiente 

\begin{hscode}\SaveRestoreHook
\column{B}{@{}>{\hspre}l<{\hspost}@{}}%
\column{E}{@{}>{\hspre}l<{\hspost}@{}}%
\>[B]{}\mathcal{D}^{+}\mathbin{::}(\Varid{a}\rightarrow \Varid{b})\rightarrow (\Varid{a}\rightarrow (\Varid{b} \times (\Varid{a}\;\multimap \;\Varid{b}))){}\<[E]%
\\
\>[B]{}\mathcal{D}^{+}\;\Varid{f}\;\Varid{a}\mathrel{=}(\Varid{f}\;\Varid{a},\mathcal{D}\;\Varid{f}\;\Varid{a}){}\<[E]%
\ColumnHook
\end{hscode}\resethooks

\end{frame}

\begin{frame}{Teoremas}

\begin{block}{Teorema 1}
  Sejam $f:: a \to b$ e $g:: b \to c$ duas funções. A derivada da composição $f$ e $g$ é:
	\begin{align*}
	\mathcal{D} \ (g \circ f ) \ a= \mathcal{D} \hspace{0.9mm} g \hspace{1.0mm}  (f \hspace{0.5mm} a)   \hspace{0.6mm} \circ \hspace{0.6mm} \mathcal{D} \hspace{0.6mm} f \ a
	\end{align*}
\end{block}


\begin{block}{Teorema 2}
  Sejam $f:: a \to b$ e $g:: b \to c$ duas funções. A regra de "cross" é a seguinte:
	\begin{align*}
	\mathcal{D} \hspace{0.5mm} (f \boldsymbol{\times} g) \hspace{0.5mm} (a,\hspace{0.2mm} b) = \mathcal{D} \hspace{0.5mm} f \hspace{0.9mm} a \boldsymbol{\times} \mathcal{D} \hspace{0.5mm} g \hspace{0.9mm} b
	\end{align*}
\end{block}


\begin{block}{Teorema 3}
  Para todas as funções lineares $f$, $\mathcal{D} \hspace{0.5mm} f \hspace{0.9mm} a = f $.
\end{block}

\end{frame}


\begin{frame}{Corolários associados a \ensuremath{\mathcal{D}^{+}}}

\begin{block}{Corolário 1.1}
\begin{hscode}\SaveRestoreHook
\column{B}{@{}>{\hspre}l<{\hspost}@{}}%
\column{9}{@{}>{\hspre}l<{\hspost}@{}}%
\column{13}{@{}>{\hspre}l<{\hspost}@{}}%
\column{E}{@{}>{\hspre}l<{\hspost}@{}}%
\>[9]{}\mathcal{D}^{+}\;(\Varid{g}\mathbin{\circ}\Varid{f})\;\Varid{a}\mathrel{=}\mathbf{let}\;\{\mskip1.5mu (\Varid{b},\Varid{f'})\mathrel{=}\mathcal{D}^{+}\;\Varid{f}\;\Varid{a};(\Varid{c},\Varid{g'})\mathrel{=}\mathcal{D}^{+}\;\Varid{g}\;\Varid{b}\mskip1.5mu\}{}\<[E]%
\\
\>[9]{}\hsindent{4}{}\<[13]%
\>[13]{}\mathbf{in}\;(\Varid{c},\Varid{g'}\mathbin{\circ}\Varid{f'}){}\<[E]%
\ColumnHook
\end{hscode}\resethooks
\end{block}
    
\begin{block}{Corolário 2.1}
\begin{hscode}\SaveRestoreHook
\column{B}{@{}>{\hspre}l<{\hspost}@{}}%
\column{9}{@{}>{\hspre}l<{\hspost}@{}}%
\column{13}{@{}>{\hspre}l<{\hspost}@{}}%
\column{E}{@{}>{\hspre}l<{\hspost}@{}}%
\>[9]{}\mathcal{D}^{+}\;(\Varid{f} \times \Varid{g})\;(\Varid{a},\Varid{b})\mathrel{=}\mathbf{let}\;\{\mskip1.5mu (\Varid{c},\Varid{f'})\mathrel{=}\mathcal{D}^{+}\;\Varid{f}\;\Varid{a};(\Varid{d},\Varid{g'})\mathrel{=}\mathcal{D}^{+}\;\Varid{g}\;\Varid{b}\mskip1.5mu\}{}\<[E]%
\\
\>[9]{}\hsindent{4}{}\<[13]%
\>[13]{}\mathbf{in}\;((\Varid{c},\Varid{d}),\Varid{f'} \times \Varid{g'}){}\<[E]%
\ColumnHook
\end{hscode}\resethooks
\end{block}
    
\begin{block}{Corolário 3.1}
	Para todas as funções lineares $f$, $\mathcal{D}^{+} \hspace{0.5mm} f = \lambda a \to (fa,\hspace{0.5mm} f)$.
\end{block}

\end{frame}





\section{Categorias e funtores}

\begin{frame}{Objetivo}
Criar uma implementação de \ensuremath{\mathcal{D}^{+}} através da transcrição dos seus corolários para teoria de categorias
de modo a obter um algoritmo generalizado para AD.
\end{frame}






\begin{frame}{Categorias e funtores}

Categoria: conjunto de objetos e morfismos com duas operações base(id e composição) e 2 regras:
\begin{itemize}
    \item $id \circ f = f \circ id = f$
    \item $f \circ (g \circ h) = (f \circ g) \circ h$
\end{itemize}

\pause

Funtor: mapeia uma categoria noutra, preservando a estrutura
\begin{itemize}
    \item Dado um objeto t $\in$ \ensuremath{\mathcal{U}} existe um objeto correspondente F t $\in$ \ensuremath{\mathcal{V}} 
    \item  Dado um morfismo m :: a \ensuremath{\rightarrow } b $\in$ \ensuremath{\mathcal{U}} existe um morfismo correspondente F m :: F a \ensuremath{\rightarrow } F b $\in$ \ensuremath{\mathcal{V}}
    \item F id ($\in$ \ensuremath{\mathcal{U}}) = id ($\in$ \ensuremath{\mathcal{V}})
    \item F (f $\circ$ g) = F f $\circ$ F g
\end{itemize}

\end{frame}


\begin{frame}{Adaptação de definições}

\begin{block}{Definição de tipo \ensuremath{\mathcal{D}}}
\ensuremath{\mathbf{newtype}\;\mathcal{D}\;\Varid{a}\;\Varid{b}\mathrel{=}\mathcal{D}\;(\Varid{a}\rightarrow \Varid{b} \times (\Varid{a}\;\multimap \;\Varid{b}))}
\end{block}
\begin{block}{Definição de \ensuremath{\mathcal{\hat{D}}}}
\begin{hscode}\SaveRestoreHook
\column{B}{@{}>{\hspre}l<{\hspost}@{}}%
\column{E}{@{}>{\hspre}l<{\hspost}@{}}%
\>[B]{}\mathcal{\hat{D}}\mathbin{::}(\Varid{a}\rightarrow \Varid{b})\rightarrow \mathcal{D}\;\Varid{a}\;\Varid{b}{}\<[E]%
\\
\>[B]{}\mathcal{\hat{D}}\;\Varid{f}\mathrel{=}\mathcal{D}\;(\mathcal{D}^{+}\;\Varid{f}){}\<[E]%
\ColumnHook
\end{hscode}\resethooks
\end{block}


\begin{block}{ Definição de \ensuremath{\mathcal{\hat{D}}} para funções lineares}
\begin{hscode}\SaveRestoreHook
\column{B}{@{}>{\hspre}l<{\hspost}@{}}%
\column{E}{@{}>{\hspre}l<{\hspost}@{}}%
\>[B]{}\Varid{linearD}\mathbin{::}(\Varid{a}\rightarrow \Varid{b})\rightarrow \mathcal{D}\;\Varid{a}\;\Varid{b}{}\<[E]%
\\
\>[B]{}\Varid{linearD}\;\Varid{f}\mathrel{=}\mathcal{D}\;(\lambda \Varid{a}\rightarrow (\Varid{f}\;\Varid{a},\Varid{f})){}\<[E]%
\ColumnHook
\end{hscode}\resethooks
\end{block}


\end{frame}






\section{Dedução de instâncias}

\begin{frame}{Passos para obter a instância a partir da definição do funtor}

\begin{itemize}
    \item Passo 1 - Assumir que \ensuremath{\mathcal{\hat{D}}} é funtor de uma instância de \ensuremath{\mathcal{D}} a determinar
    \item Passo 2 - Substituir pelo que determinamos nos corolários
    \item Passo 3 - Generalizar condições se necessário para obtermos instância
\end{itemize}

\end{frame}



\begin{frame}{Exemplo para categorias}
\begin{block}{Passo 1}

\ensuremath{\Varid{id}\mathrel{=}\mathcal{\hat{D}}\;\Varid{id}\mathrel{=}\mathcal{D}\;(\mathcal{D}^{+}\;\Varid{id})}

\ensuremath{\mathcal{\hat{D}}\;\Varid{g}\mathbin{\circ}\mathcal{\hat{D}}\;\Varid{f}\mathrel{=}\mathcal{\hat{D}}\;(\Varid{g}\mathbin{\circ}\Varid{f})\mathrel{=}\mathcal{D}\;(\mathcal{\hat{D}}\;(\Varid{g}\mathbin{\circ}\Varid{f}))}

\end{block}

\pause

\begin{block}{Passo 2}

\ensuremath{\Varid{id}\mathrel{=}\mathcal{D}\;(\lambda \Varid{a}\rightarrow (\Varid{id}\;\Varid{a},\Varid{id}))}

\ensuremath{\mathcal{\hat{D}}\;\Varid{g}\mathbin{\circ}\mathcal{\hat{D}}\;\Varid{f}\mathrel{=}\mathcal{D}\;(\lambda \Varid{a}\rightarrow \mathbf{let}\;\{\mskip1.5mu (\Varid{b},\Varid{f'})\mathrel{=}\mathcal{D}^{+}\;\Varid{f}\;\Varid{a};(\Varid{c},\Varid{g'})\mathrel{=}\mathcal{D}^{+}\;\Varid{g}\;\Varid{b}\mskip1.5mu\}\;\mathbf{in}\;(\Varid{c},\Varid{g'}\mathbin{\circ}\Varid{f'}))}

\end{block}

\pause

\begin{block}{Passo 3}

Para a nossa instância a primeira equação que determinamos serve como definição da identidade.

Para definir a composição generalizamos a condição:

\ensuremath{\mathcal{D}\;\Varid{g}\mathbin{\circ}\mathcal{D}\;\Varid{f}\mathrel{=}\mathcal{D}\;(\lambda \Varid{a}\rightarrow \mathbf{let}\;\{\mskip1.5mu (\Varid{b},\Varid{f'})\mathrel{=}\Varid{f}\;\Varid{a};(\Varid{c},\Varid{g'})\mathrel{=}\Varid{g}\;\Varid{b}\mskip1.5mu\}\;\mathbf{in}\;(\Varid{c},\Varid{g'}\mathbin{\circ}\Varid{f'}))}

\end{block}
\end{frame}


\begin{frame}
\begin{block}{Definição da classe de categoria}
\begin{hscode}\SaveRestoreHook
\column{B}{@{}>{\hspre}l<{\hspost}@{}}%
\column{5}{@{}>{\hspre}l<{\hspost}@{}}%
\column{E}{@{}>{\hspre}l<{\hspost}@{}}%
\>[B]{}\mathbf{class}\;\Conid{Category}\;\Varid{k}\;\mathbf{where}{}\<[E]%
\\
\>[B]{}\hsindent{5}{}\<[5]%
\>[5]{}\Varid{id}\mathbin{::}(\Varid{a'k'a}){}\<[E]%
\\
\>[B]{}\hsindent{5}{}\<[5]%
\>[5]{}(\mathbin{\circ})\mathbin{::}(\Varid{b'k'c})\rightarrow (\Varid{a'k'b})\rightarrow (\Varid{a'k'c}){}\<[E]%
\ColumnHook
\end{hscode}\resethooks
\end{block}

\begin{block}{Instância deduzida para a categoria}
\begin{hscode}\SaveRestoreHook
\column{B}{@{}>{\hspre}l<{\hspost}@{}}%
\column{5}{@{}>{\hspre}l<{\hspost}@{}}%
\column{E}{@{}>{\hspre}l<{\hspost}@{}}%
\>[B]{}\mathbf{instance}\;\Conid{Category}\;\mathcal{D}\;\mathbf{where}{}\<[E]%
\\
\>[B]{}\hsindent{5}{}\<[5]%
\>[5]{}\Varid{id}\mathrel{=}\Varid{linearD}\;\Varid{id}{}\<[E]%
\\
\>[B]{}\hsindent{5}{}\<[5]%
\>[5]{}\mathcal{D}\;\Varid{g}\mathbin{\circ}\mathcal{D}\;\Varid{f}\mathrel{=}{}\<[E]%
\\
\>[B]{}\hsindent{5}{}\<[5]%
\>[5]{}\mathcal{D}\;(\lambda \Varid{a}\rightarrow \mathbf{let}\;\{\mskip1.5mu (\Varid{b},\Varid{f'})\mathrel{=}\Varid{f}\;\Varid{a};(\Varid{c},\Varid{g'})\mathrel{=}\Varid{g}\;\Varid{b}\mskip1.5mu\}\;\mathbf{in}\;(\Varid{c},\Varid{g'}\mathbin{\circ}\Varid{f'})){}\<[E]%
\ColumnHook
\end{hscode}\resethooks
\end{block}
\end{frame}



\begin{frame}
\begin{block}{Definição da classe de categoria monoidal}
\begin{hscode}\SaveRestoreHook
\column{B}{@{}>{\hspre}l<{\hspost}@{}}%
\column{5}{@{}>{\hspre}l<{\hspost}@{}}%
\column{E}{@{}>{\hspre}l<{\hspost}@{}}%
\>[B]{}\mathbf{class}\;\Conid{Category}\;\Varid{k}\Rightarrow\Conid{Monoidal}\;\Varid{k}\;\mathbf{where}{}\<[E]%
\\
\>[B]{}\hsindent{5}{}\<[5]%
\>[5]{}( \times )\mathbin{::}(\Varid{a}\;\text{\tt 'k'}\;\Varid{c})\rightarrow (\Varid{b}\;\text{\tt 'k'}\;\Varid{d})\rightarrow ((\Varid{a} \times \Varid{b})\;\text{\tt 'k'}\;(\Varid{c} \times \Varid{d})){}\<[E]%
\ColumnHook
\end{hscode}\resethooks
\end{block}


\begin{block}{Instância deduzida para a categoria monoidal}
\begin{hscode}\SaveRestoreHook
\column{B}{@{}>{\hspre}l<{\hspost}@{}}%
\column{5}{@{}>{\hspre}l<{\hspost}@{}}%
\column{38}{@{}>{\hspre}l<{\hspost}@{}}%
\column{E}{@{}>{\hspre}l<{\hspost}@{}}%
\>[B]{}\mathbf{instance}\;\Conid{Monoidal}\;\mathcal{D}\;\mathbf{where}{}\<[E]%
\\
\>[B]{}\hsindent{5}{}\<[5]%
\>[5]{}\mathcal{D}\;\Varid{f} \times \mathcal{D}\;\Varid{g}\mathrel{=}\mathcal{D}\;(\lambda (\Varid{a},\Varid{b})\rightarrow \mathbf{let}\;\{\mskip1.5mu (\Varid{c},\Varid{f'})\mathrel{=}\Varid{f}\;\Varid{a};(\Varid{d},\Varid{g'})\mathrel{=}\Varid{g}\;\Varid{b}\mskip1.5mu\}{}\<[E]%
\\
\>[5]{}\hsindent{33}{}\<[38]%
\>[38]{}\mathbf{in}\;((\Varid{c},\Varid{d}),\Varid{f'} \times \Varid{g'})){}\<[E]%
\ColumnHook
\end{hscode}\resethooks
\end{block}
\end{frame}




\begin{frame}

\begin{block}{Definição da classe de categoria cartesiana}
\begin{hscode}\SaveRestoreHook
\column{B}{@{}>{\hspre}l<{\hspost}@{}}%
\column{3}{@{}>{\hspre}l<{\hspost}@{}}%
\column{E}{@{}>{\hspre}l<{\hspost}@{}}%
\>[B]{}\mathbf{class}\;\Conid{Monoidal}\;\Varid{k}\Rightarrow\Conid{Cartesian}\;\Varid{k}\;\mathbf{where}{}\<[E]%
\\
\>[B]{}\hsindent{3}{}\<[3]%
\>[3]{}\Varid{exl}\mathbin{::}(\Varid{a},\Varid{b})\;\text{\tt 'k'}\;\Varid{a}{}\<[E]%
\\
\>[B]{}\hsindent{3}{}\<[3]%
\>[3]{}\Varid{exr}\mathbin{::}(\Varid{a},\Varid{b})\;\text{\tt 'k'}\;\Varid{b}{}\<[E]%
\\
\>[B]{}\hsindent{3}{}\<[3]%
\>[3]{}\Varid{dup}\mathbin{::}\Varid{a}\;\text{\tt 'k'}\;(\Varid{a},\Varid{a}){}\<[E]%
\ColumnHook
\end{hscode}\resethooks
\end{block}

\begin{block}{Instância deduzida para a categoria cartesiana}
\begin{hscode}\SaveRestoreHook
\column{B}{@{}>{\hspre}l<{\hspost}@{}}%
\column{5}{@{}>{\hspre}l<{\hspost}@{}}%
\column{E}{@{}>{\hspre}l<{\hspost}@{}}%
\>[B]{}\mathbf{instance}\;\Conid{Cartesian}\;\Conid{D}\;\mathbf{where}{}\<[E]%
\\
\>[B]{}\hsindent{5}{}\<[5]%
\>[5]{}\Varid{exl}\mathrel{=}\Varid{linearD}\;\Varid{exl}{}\<[E]%
\\
\>[B]{}\hsindent{5}{}\<[5]%
\>[5]{}\Varid{exr}\mathrel{=}\Varid{linearD}\;\Varid{exr}{}\<[E]%
\\
\>[B]{}\hsindent{5}{}\<[5]%
\>[5]{}\Varid{dup}\mathrel{=}\Varid{linearD}\;\Varid{dup}{}\<[E]%
\ColumnHook
\end{hscode}\resethooks
\end{block}

\end{frame}

\begin{frame}
\begin{block}{Definição da classe de categoria cocartesiana}
\begin{hscode}\SaveRestoreHook
\column{B}{@{}>{\hspre}l<{\hspost}@{}}%
\column{5}{@{}>{\hspre}l<{\hspost}@{}}%
\column{E}{@{}>{\hspre}l<{\hspost}@{}}%
\>[B]{}\mathbf{class}\;\Conid{Category}\;\Varid{k}\Rightarrow\Conid{Cocartesian}\;\Varid{k}\;\mathbf{where}{}\<[E]%
\\
\>[B]{}\hsindent{5}{}\<[5]%
\>[5]{}\Varid{inl}\mathbin{::}\Varid{a}\;\text{\tt 'k'}\;(\Varid{a},\Varid{b}){}\<[E]%
\\
\>[B]{}\hsindent{5}{}\<[5]%
\>[5]{}\Varid{inr}\mathbin{::}\Varid{b}\;\text{\tt 'k'}\;(\Varid{a},\Varid{b}){}\<[E]%
\\
\>[B]{}\hsindent{5}{}\<[5]%
\>[5]{}\Varid{jam}\mathbin{::}(\Varid{a},\Varid{a})\;\text{\tt 'k'}\;\Varid{a}{}\<[E]%
\ColumnHook
\end{hscode}\resethooks
\end{block}

\pause

\begin{block}{Dedução de\ensuremath{(\rightarrow^+ )}}
\hspace*{\fill}
\xymatrix@R=2mm{
    a \ar[rd] &  \\
      & a+b?\\
    b \ar[ru] &  \\
}
\hspace*{\fill}
\begin{itemize}
\item 
    Queremos uma categoria que os seus objetos tenham adição e seja capaz de multiplicar por constantes (scaling).
\end{itemize}
\end{block}
\end{frame}


\begin{frame}{Instância deduzida para AD genérico}
\begin{block}{Dedução de \ensuremath{ D_k }}
\begin{itemize}
\item
    Queremos ter um \ensuremath{ D_k } tal que seja o \ensuremath{\mathcal{D}} mas com uma categoria
de genérica, pois nunca fizemos nada de específico com a \ensuremath{\multimap }.
\item
    Queremos definir também as operações usuais nesta categoria, como a soma, multiplicação, sin, cos, etc. 
\end{itemize}
\end{block}

\end{frame}







\section{RAD e FAD generalizados}

\begin{frame}{Generalizando RAD e FAD}

Obter RAD e FAD de algoritmo AD genérico: forçar a direção da composição de morfismos

\pause

\begin{block}{Conversão da escrita de morfismos}
\ensuremath{\Varid{f}\mathbin{::}\Varid{a}\;\text{\tt 'k'}\;\Varid{b}\Rightarrow(\mathbin{\circ}\Varid{f})\mathbin{::}(\Varid{b}\;\text{\tt 'k'}\;\Varid{r})\rightarrow (\Varid{a}\;\text{\tt 'k'}\;\Varid{r})} para r objeto de categoria k.
\end{block}


\pause

\begin{block}{Definição de novo tipo}
\begin{hscode}\SaveRestoreHook
\column{B}{@{}>{\hspre}l<{\hspost}@{}}%
\column{E}{@{}>{\hspre}l<{\hspost}@{}}%
\>[B]{}\mathbf{newtype}\;Cont^{r}_{k}\;\Varid{a}\;\Varid{b}\mathrel{=}\Conid{Cont}\;((\Varid{b}\;\text{\tt 'k'}\;\Varid{r})\rightarrow (\Varid{a}\;\text{\tt 'k'}\;\Varid{r})){}\<[E]%
\ColumnHook
\end{hscode}\resethooks
\end{block}

\begin{block}{Funtor derivado dele}
\begin{hscode}\SaveRestoreHook
\column{B}{@{}>{\hspre}l<{\hspost}@{}}%
\column{E}{@{}>{\hspre}l<{\hspost}@{}}%
\>[B]{}\Varid{cont}\mathbin{::}\Conid{Category}\;\Varid{k}\Rightarrow(\Varid{a}\;\text{\tt 'k'}\;\Varid{b})\rightarrow Cont^{r}_{k}\;\Varid{a}\;\Varid{b}{}\<[E]%
\\
\>[B]{}\Varid{cont}\;\Varid{f}\mathrel{=}\Conid{Cont}\;(\mathbin{\circ}\Varid{f}){}\<[E]%
\ColumnHook
\end{hscode}\resethooks
\end{block}

\end{frame}


\begin{frame}{Instância deduzida para RAD genérico}
\begin{hscode}\SaveRestoreHook
\column{B}{@{}>{\hspre}l<{\hspost}@{}}%
\column{3}{@{}>{\hspre}l<{\hspost}@{}}%
\column{E}{@{}>{\hspre}l<{\hspost}@{}}%
\>[B]{}\mathbf{instance}\;\Conid{Category}\;\Varid{k}\Rightarrow\Conid{Category}\;Cont^{r}_{k}\;\mathbf{where}{}\<[E]%
\\
\>[B]{}\hsindent{3}{}\<[3]%
\>[3]{}\Varid{id}\mathrel{=}\Conid{Cont}\;\Varid{id}{}\<[E]%
\\
\>[B]{}\hsindent{3}{}\<[3]%
\>[3]{}\Conid{Cont}\;\Varid{g}\mathbin{\circ}\Conid{Cont}\;\Varid{f}\mathrel{=}\Conid{Cont}\;(\Varid{f}\mathbin{\circ}\Varid{g}){}\<[E]%
\\[\blanklineskip]%
\>[B]{}\mathbf{instance}\;\Conid{Monoidal}\;\Varid{k}\Rightarrow\Conid{Monoidal}\;Cont^{r}_{k}\;\mathbf{where}{}\<[E]%
\\
\>[B]{}\hsindent{3}{}\<[3]%
\>[3]{}\Conid{Conf}\;\Varid{f} \times \Conid{Cont}\;\Varid{g}\mathrel{=}\Conid{Cont}\;(\Varid{join}\mathbin{\circ}(\Varid{f} \times \Varid{g})\mathbin{\circ}\Varid{unjoin}){}\<[E]%
\\[\blanklineskip]%
\>[B]{}\mathbf{instance}\;\Conid{Cartesian}\;\Varid{k}\Rightarrow\Conid{Cartesian}\;Cont^{r}_{k}\;\mathbf{where}{}\<[E]%
\\
\>[B]{}\hsindent{3}{}\<[3]%
\>[3]{}\Varid{exl}\mathrel{=}\Conid{Cont}\;(\Varid{join}\mathbin{\circ}\Varid{inl});\Varid{exr}\mathrel{=}\Conid{Cont}\;(\Varid{join}\mathbin{\circ}\Varid{inr}){}\<[E]%
\\
\>[B]{}\hsindent{3}{}\<[3]%
\>[3]{}\Varid{dup}\mathrel{=}\Conid{Cont}\;(\Varid{jam}\mathbin{\circ}\Varid{unjoin}){}\<[E]%
\\[\blanklineskip]%
\>[B]{}\mathbf{instance}\;\Conid{Cocartesian}\;\Varid{k}\Rightarrow\Conid{Cocartesian}\;Cont^{r}_{k}\;\mathbf{where}{}\<[E]%
\\
\>[B]{}\hsindent{3}{}\<[3]%
\>[3]{}\Varid{inl}\mathrel{=}\Conid{Cont}\;(\Varid{exl}\mathbin{\circ}\Varid{unjoin});\Varid{inr}\mathrel{=}\Conid{Cont}\;(\Varid{exr}\mathbin{\circ}\Varid{unjoin}){}\<[E]%
\\
\>[B]{}\hsindent{3}{}\<[3]%
\>[3]{}\Varid{jam}\mathrel{=}\Conid{Cont}\;(\Varid{join}\mathbin{\circ}\Varid{dup}){}\<[E]%
\\[\blanklineskip]%
\>[B]{}\mathbf{instance}\;\Conid{Scalable}\;\Varid{k}\;\Varid{a}\Rightarrow\Conid{Scalable}\;Cont^{r}_{k}\;\Varid{a}\;\mathbf{where}{}\<[E]%
\\
\>[B]{}\hsindent{3}{}\<[3]%
\>[3]{}\Varid{scale}\;\Varid{s}\mathrel{=}\Conid{Cont}\;(\Varid{scale}\;\Varid{s}){}\<[E]%
\ColumnHook
\end{hscode}\resethooks
\end{frame}








\section{Demonstração prática}

\begin{frame}{Demonstração prática}

Demonstração prática

Link do projeto git: [LEI 2019] \cite{LEI:2019}

\end{frame}


\bibliographystyle{acm}%{ACM-Reference-Format}
\bibliography{apresentacao}



\end{document}


