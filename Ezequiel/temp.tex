\documentclass{beamer}

\mode<presentation>
{
  \usetheme{Warsaw}

  \setbeamercovered{transparent}
}


\usepackage[english]{babel}
\usepackage[utf8x]{inputenc}
%\usepackage[latin1]{inputenc}

\usepackage{times}
\usepackage[T1]{fontenc}
% Note that the encoding and the font should match. If T1
% does not look nice, try deleting the line with the fontenc.

%---------- lhs2tex ---------------------------------
%% ODER: format ==         = "\mathrel{==}"
%% ODER: format /=         = "\neq "
%
%
\makeatletter
\@ifundefined{lhs2tex.lhs2tex.sty.read}%
  {\@namedef{lhs2tex.lhs2tex.sty.read}{}%
   \newcommand\SkipToFmtEnd{}%
   \newcommand\EndFmtInput{}%
   \long\def\SkipToFmtEnd#1\EndFmtInput{}%
  }\SkipToFmtEnd

\newcommand\ReadOnlyOnce[1]{\@ifundefined{#1}{\@namedef{#1}{}}\SkipToFmtEnd}
\usepackage{amstext}
\usepackage{amssymb}
\usepackage{stmaryrd}
\DeclareFontFamily{OT1}{cmtex}{}
\DeclareFontShape{OT1}{cmtex}{m}{n}
  {<5><6><7><8>cmtex8
   <9>cmtex9
   <10><10.95><12><14.4><17.28><20.74><24.88>cmtex10}{}
\DeclareFontShape{OT1}{cmtex}{m}{it}
  {<-> ssub * cmtt/m/it}{}
\newcommand{\texfamily}{\fontfamily{cmtex}\selectfont}
\DeclareFontShape{OT1}{cmtt}{bx}{n}
  {<5><6><7><8>cmtt8
   <9>cmbtt9
   <10><10.95><12><14.4><17.28><20.74><24.88>cmbtt10}{}
\DeclareFontShape{OT1}{cmtex}{bx}{n}
  {<-> ssub * cmtt/bx/n}{}
\newcommand{\tex}[1]{\text{\texfamily#1}}	% NEU

\newcommand{\Sp}{\hskip.33334em\relax}


\newcommand{\Conid}[1]{\mathit{#1}}
\newcommand{\Varid}[1]{\mathit{#1}}
\newcommand{\anonymous}{\kern0.06em \vbox{\hrule\@width.5em}}
\newcommand{\plus}{\mathbin{+\!\!\!+}}
\newcommand{\bind}{\mathbin{>\!\!\!>\mkern-6.7mu=}}
\newcommand{\rbind}{\mathbin{=\mkern-6.7mu<\!\!\!<}}% suggested by Neil Mitchell
\newcommand{\sequ}{\mathbin{>\!\!\!>}}
\renewcommand{\leq}{\leqslant}
\renewcommand{\geq}{\geqslant}
\usepackage{polytable}

%mathindent has to be defined
\@ifundefined{mathindent}%
  {\newdimen\mathindent\mathindent\leftmargini}%
  {}%

\def\resethooks{%
  \global\let\SaveRestoreHook\empty
  \global\let\ColumnHook\empty}
\newcommand*{\savecolumns}[1][default]%
  {\g@addto@macro\SaveRestoreHook{\savecolumns[#1]}}
\newcommand*{\restorecolumns}[1][default]%
  {\g@addto@macro\SaveRestoreHook{\restorecolumns[#1]}}
\newcommand*{\aligncolumn}[2]%
  {\g@addto@macro\ColumnHook{\column{#1}{#2}}}

\resethooks

\newcommand{\onelinecommentchars}{\quad-{}- }
\newcommand{\commentbeginchars}{\enskip\{-}
\newcommand{\commentendchars}{-\}\enskip}

\newcommand{\visiblecomments}{%
  \let\onelinecomment=\onelinecommentchars
  \let\commentbegin=\commentbeginchars
  \let\commentend=\commentendchars}

\newcommand{\invisiblecomments}{%
  \let\onelinecomment=\empty
  \let\commentbegin=\empty
  \let\commentend=\empty}

\visiblecomments

\newlength{\blanklineskip}
\setlength{\blanklineskip}{0.66084ex}

\newcommand{\hsindent}[1]{\quad}% default is fixed indentation
\let\hspre\empty
\let\hspost\empty
\newcommand{\NB}{\textbf{NB}}
\newcommand{\Todo}[1]{$\langle$\textbf{To do:}~#1$\rangle$}

\EndFmtInput
\makeatother
%
%
%
%
%
%
% This package provides two environments suitable to take the place
% of hscode, called "plainhscode" and "arrayhscode". 
%
% The plain environment surrounds each code block by vertical space,
% and it uses \abovedisplayskip and \belowdisplayskip to get spacing
% similar to formulas. Note that if these dimensions are changed,
% the spacing around displayed math formulas changes as well.
% All code is indented using \leftskip.
%
% Changed 19.08.2004 to reflect changes in colorcode. Should work with
% CodeGroup.sty.
%
\ReadOnlyOnce{polycode.fmt}%
\makeatletter

\newcommand{\hsnewpar}[1]%
  {{\parskip=0pt\parindent=0pt\par\vskip #1\noindent}}

% can be used, for instance, to redefine the code size, by setting the
% command to \small or something alike
\newcommand{\hscodestyle}{}

% The command \sethscode can be used to switch the code formatting
% behaviour by mapping the hscode environment in the subst directive
% to a new LaTeX environment.

\newcommand{\sethscode}[1]%
  {\expandafter\let\expandafter\hscode\csname #1\endcsname
   \expandafter\let\expandafter\endhscode\csname end#1\endcsname}

% "compatibility" mode restores the non-polycode.fmt layout.

\newenvironment{compathscode}%
  {\par\noindent
   \advance\leftskip\mathindent
   \hscodestyle
   \let\\=\@normalcr
   \let\hspre\(\let\hspost\)%
   \pboxed}%
  {\endpboxed\)%
   \par\noindent
   \ignorespacesafterend}

\newcommand{\compaths}{\sethscode{compathscode}}

% "plain" mode is the proposed default.
% It should now work with \centering.
% This required some changes. The old version
% is still available for reference as oldplainhscode.

\newenvironment{plainhscode}%
  {\hsnewpar\abovedisplayskip
   \advance\leftskip\mathindent
   \hscodestyle
   \let\hspre\(\let\hspost\)%
   \pboxed}%
  {\endpboxed%
   \hsnewpar\belowdisplayskip
   \ignorespacesafterend}

\newenvironment{oldplainhscode}%
  {\hsnewpar\abovedisplayskip
   \advance\leftskip\mathindent
   \hscodestyle
   \let\\=\@normalcr
   \(\pboxed}%
  {\endpboxed\)%
   \hsnewpar\belowdisplayskip
   \ignorespacesafterend}

% Here, we make plainhscode the default environment.

\newcommand{\plainhs}{\sethscode{plainhscode}}
\newcommand{\oldplainhs}{\sethscode{oldplainhscode}}
\plainhs

% The arrayhscode is like plain, but makes use of polytable's
% parray environment which disallows page breaks in code blocks.

\newenvironment{arrayhscode}%
  {\hsnewpar\abovedisplayskip
   \advance\leftskip\mathindent
   \hscodestyle
   \let\\=\@normalcr
   \(\parray}%
  {\endparray\)%
   \hsnewpar\belowdisplayskip
   \ignorespacesafterend}

\newcommand{\arrayhs}{\sethscode{arrayhscode}}

% The mathhscode environment also makes use of polytable's parray 
% environment. It is supposed to be used only inside math mode 
% (I used it to typeset the type rules in my thesis).

\newenvironment{mathhscode}%
  {\parray}{\endparray}

\newcommand{\mathhs}{\sethscode{mathhscode}}

% texths is similar to mathhs, but works in text mode.

\newenvironment{texthscode}%
  {\(\parray}{\endparray\)}

\newcommand{\texths}{\sethscode{texthscode}}

% The framed environment places code in a framed box.

\def\codeframewidth{\arrayrulewidth}
\RequirePackage{calc}

\newenvironment{framedhscode}%
  {\parskip=\abovedisplayskip\par\noindent
   \hscodestyle
   \arrayrulewidth=\codeframewidth
   \tabular{@{}|p{\linewidth-2\arraycolsep-2\arrayrulewidth-2pt}|@{}}%
   \hline\framedhslinecorrect\\{-1.5ex}%
   \let\endoflinesave=\\
   \let\\=\@normalcr
   \(\pboxed}%
  {\endpboxed\)%
   \framedhslinecorrect\endoflinesave{.5ex}\hline
   \endtabular
   \parskip=\belowdisplayskip\par\noindent
   \ignorespacesafterend}

\newcommand{\framedhslinecorrect}[2]%
  {#1[#2]}

\newcommand{\framedhs}{\sethscode{framedhscode}}

% The inlinehscode environment is an experimental environment
% that can be used to typeset displayed code inline.

\newenvironment{inlinehscode}%
  {\(\def\column##1##2{}%
   \let\>\undefined\let\<\undefined\let\\\undefined
   \newcommand\>[1][]{}\newcommand\<[1][]{}\newcommand\\[1][]{}%
   \def\fromto##1##2##3{##3}%
   \def\nextline{}}{\) }%

\newcommand{\inlinehs}{\sethscode{inlinehscode}}

% The joincode environment is a separate environment that
% can be used to surround and thereby connect multiple code
% blocks.

\newenvironment{joincode}%
  {\let\orighscode=\hscode
   \let\origendhscode=\endhscode
   \def\endhscode{\def\hscode{\endgroup\def\@currenvir{hscode}\\}\begingroup}
   %\let\SaveRestoreHook=\empty
   %\let\ColumnHook=\empty
   %\let\resethooks=\empty
   \orighscode\def\hscode{\endgroup\def\@currenvir{hscode}}}%
  {\origendhscode
   \global\let\hscode=\orighscode
   \global\let\endhscode=\origendhscode}%

\makeatother
\EndFmtInput
%
%----------------------------------------------------

\newenvironment{slide}[1]{\begin{frame}\frametitle{#1}}{\end{frame}}

\title
{...Machine Learning...}

\author[Artur, Ezequiel, Nelson] 
{Artur \and Ezequiel \and Nelson}

\institute
{Universidade do Minho}

\date
{26 de Abril}

\subject{Talks}

% If you have a file called "university-logo-filename.xxx", where xxx
% is a graphic format that can be processed by latex or pdflatex,
% resp., then you can add a logo as follows:

% \pgfdeclareimage[height=0.5cm]{university-logo}{university-logo-filename}
% \logo{\pgfuseimage{university-logo}}

% Delete this, if you do not want the table of contents to pop up at
% the beginning of each subsection:

\AtBeginSubsection[]
{
  \begin{frame}<beamer>{Outline}
    \tableofcontents[currentsection,currentsubsection]
  \end{frame}
}


% If you wish to uncover everything in a step-wise fashion, uncomment
% the following command: 

%\beamerdefaultoverlayspecification{<+->}


\begin{document}


\section{Reverse-Mode Automatic Differentiation(RAD)}
\begin{frame}{Esboço}

O algoritmo AD que derivamos nas categorias(capitulo 4) e cuja instância se encontra generalizada na
figura 6 do documento pode ser visto como familia de algoritmos que pode tomar forma de FAD caso usemos
apenas composição à direita ou de RAD caso usemos apenas composição à esquerda(podendo tomar outras formas
se não usarmos exclusivamente uma composição, mas não nos interressa muito isso para este papel).

Sabendo que queremos definir RAD e sabendo o que temos do parágrafo anterior uma pergunta aparece:
como forçar a composição à esquerda?

Dada uma qualquer categoria k podemos representar os seus morfismos de forma a indicar
que queremos compor à esquerda com um morfismo h que ainda iremos definir.

Para tal um morfismo f :: a 'k' b é representado pela função \ensuremath{(\lambda \Varid{circ}\;\Varid{f})\mathbin{::}(\Varid{b}\;\text{\tt 'k'}\;\Varid{r})\rightarrow (\Varid{a}\;\text{\tt 'k'}\;\Varid{r})},
com r objeto em k.

Ao morfismo h chamaremos continuação de f, e a construção de uma categoria à volta desta ideia
permite-nos garantir a propriedade que queremos: todas as composições são feitas à esquerda.

Nesta instância de categoria as composições na computação transformam-se em computações na continuação.
Por exemplo, \ensuremath{\Varid{g}\lambda \Varid{circ}\;\Varid{f}} com continuação k (\ensuremath{\Varid{k}\lambda \Varid{circ}\;(\Varid{g}\lambda \Varid{circ}\;\Varid{f})}) transforma-se em f com continuação
\ensuremath{\Varid{g}\lambda \Varid{circ}\;\Varid{g}\;((\Varid{k}\lambda \Varid{circ}\;\Varid{g})\lambda \Varid{circ}\;\Varid{f})}.

Nota: a composição inicial para qualquer morfismo é id (\ensuremath{\Varid{id}\lambda \Varid{circ}\;\Varid{f}\mathrel{=}\Varid{id}})

Usando a mesma ideia estabelecida no capitulo 4 criamos um novo tipo de dados e geramos a categoria associada:

\ensuremath{\mathbf{newtype}\;\Conid{Cont\char95 }\;\{\mskip1.5mu \Varid{k}\mskip1.5mu\}\mathbin{\uparrow}\{\mskip1.5mu \Varid{r}\mskip1.5mu\}\;\Varid{a}\;\Varid{b}\mathrel{=}\Conid{Cont}\;((\Varid{b}\;\text{\tt 'k'}\;\Varid{r})\rightarrow (\Varid{a}\;\text{\tt 'k'}\;\Varid{r}))}

\begin{hscode}\SaveRestoreHook
\column{B}{@{}>{\hspre}l<{\hspost}@{}}%
\column{E}{@{}>{\hspre}l<{\hspost}@{}}%
\>[B]{}\Varid{cont}\mathbin{::}\Conid{Category}\;\Varid{k}\Rightarrow(\Varid{a}\;\text{\tt 'k'}\;\Varid{b})\rightarrow \Conid{Cont\char95 }\;\{\mskip1.5mu \Varid{k}\mskip1.5mu\}\mathbin{\uparrow}\{\mskip1.5mu \Varid{r}\mskip1.5mu\}\;\Varid{a}\;\Varid{b}{}\<[E]%
\\
\>[B]{}\Varid{cont}\;\Varid{f}\mathrel{=}\Conid{Cont}\;(\mathbin{\circ}\Varid{f}){}\<[E]%
\ColumnHook
\end{hscode}\resethooks

e derivamos uma instância(ver figura 7), provando que cont é homomorfismo nessa derivação.

\end{frame}




\begin{frame}{A short introdution}


\begin{itemize}
    \item<1-> In chapter 4 we've derived an AD algorithm that was generalized in figure 6 of the document
    \item<2-> With fully right-associated compositions this algoritm becomes a foward-mode AD and with fully left-associated becomes a reverse-mode AD
    \item<3-> We want to obtain generalized FAD and RAD algoritms 
    \item<4-> How do we describe this in Categorical notation?
\end{itemize}

\end{frame}




\begin{frame}{Converting morfisms}

Given a category k we can represent its morfisms using the intent to left-compose with another morfism:

\ensuremath{\Varid{f}\mathbin{::}\Varid{a'k'b}\;\Varid{becomes}\;(\lambda \Varid{circ}\;\Varid{f})\mathbin{::}(\Varid{b'k'r})\rightarrow (\Varid{a'k'r})} where r is any object of k.
If h is the morfism we'll compose with f then h is the continuation of f.

With this idea in mind we can derive a category based on it, creating a generalization of the RAD algoritm.

\end{frame}




\begin{frame}{Instance deduction}

We'll begin by creating a new data type:

\ensuremath{\mathbf{newtype}\;\Conid{Cont\char95 }\;\{\mskip1.5mu \Varid{k}\mskip1.5mu\}\mathbin{\uparrow}\{\mskip1.5mu \Varid{r}\mskip1.5mu\}\;\Varid{a}\;\Varid{b}\mathrel{=}\Conid{Cont}\;((\Varid{b'k'r})\rightarrow (\Varid{a'k'r}))}

And then defining a functor from it:
\begin{hscode}\SaveRestoreHook
\column{B}{@{}>{\hspre}l<{\hspost}@{}}%
\column{E}{@{}>{\hspre}l<{\hspost}@{}}%
\>[B]{}\Varid{cont}\mathbin{::}\Conid{Category}\;\Varid{k}\Rightarrow(\Varid{a}\;\text{\tt 'k'}\;\Varid{b})\rightarrow \Conid{Cont\char95 }\;\{\mskip1.5mu \Varid{k}\mskip1.5mu\}\mathbin{\uparrow}\{\mskip1.5mu \Varid{r}\mskip1.5mu\}\;\Varid{a}\;\Varid{b}{}\<[E]%
\\
\>[B]{}\Varid{cont}\;\Varid{f}\mathrel{=}\Conid{Cont}\;(\mathbin{\circ}\Varid{f}){}\<[E]%
\ColumnHook
\end{hscode}\resethooks


With this we can derive new categorical isntances:

\end{frame}



\begin{frame}{Instance deduction}
\begin{hscode}\SaveRestoreHook
\column{B}{@{}>{\hspre}l<{\hspost}@{}}%
\column{3}{@{}>{\hspre}l<{\hspost}@{}}%
\column{10}{@{}>{\hspre}l<{\hspost}@{}}%
\column{E}{@{}>{\hspre}l<{\hspost}@{}}%
\>[B]{}\mathbf{newtype}\;{}\<[10]%
\>[10]{}\Conid{Cont\char95 }\;\{\mskip1.5mu \Varid{k}\mskip1.5mu\}\mathbin{\uparrow}\{\mskip1.5mu \Varid{r}\mskip1.5mu\}\;\Varid{a}\;\Varid{b}\mathrel{=}\Conid{Cont}\;((\Varid{b'k'r})\rightarrow (\Varid{a'k'r})){}\<[E]%
\\[\blanklineskip]%
\>[B]{}\mathbf{instance}\;\Conid{Category}\;\Varid{k}\Rightarrow\Conid{Category}\;\Conid{Cont\char95 }\;\{\mskip1.5mu \Varid{k}\mskip1.5mu\}\mathbin{\uparrow}\{\mskip1.5mu \Varid{r}\mskip1.5mu\}\;\mathbf{where}{}\<[E]%
\\
\>[B]{}\hsindent{3}{}\<[3]%
\>[3]{}\Varid{id}\mathrel{=}\Conid{Cont}\;\Varid{id}{}\<[E]%
\\
\>[B]{}\hsindent{3}{}\<[3]%
\>[3]{}\Conid{Cont}\;\Varid{g}\mathbin{\circ}\Conid{Cont}\;\Varid{f}\mathrel{=}\Conid{Cont}\;(\Varid{f}\mathbin{\circ}\Varid{g}){}\<[E]%
\\[\blanklineskip]%
\>[B]{}\mathbf{instance}\;\Conid{Monoidal}\;\Varid{k}\Rightarrow\Conid{Monoidal}\;\Conid{Cont\char95 }\;\{\mskip1.5mu \Varid{k}\mskip1.5mu\}\mathbin{\uparrow}\{\mskip1.5mu \Varid{r}\mskip1.5mu\}\;\mathbf{where}{}\<[E]%
\\
\>[B]{}\hsindent{3}{}\<[3]%
\>[3]{}\Conid{Conf}\;\Varid{f}\;\Varid{x}\;\Conid{Cont}\;\Varid{g}\mathrel{=}\Conid{Cont}\;( \bigtriangleup \mathbin{\circ}(\Varid{f}\;\Varid{x}\;\Varid{g})\mathbin{\circ}\Varid{unjoin}){}\<[E]%
\\[\blanklineskip]%
\>[B]{}\mathbf{instance}\;\Conid{Cartesian}\;\Varid{k}\Rightarrow\Conid{Cartesian}\;\Conid{Cont\char95 }\;\{\mskip1.5mu \Varid{k}\mskip1.5mu\}\mathbin{\uparrow}\{\mskip1.5mu \Varid{r}\mskip1.5mu\}\;\mathbf{where}{}\<[E]%
\\
\>[B]{}\hsindent{3}{}\<[3]%
\>[3]{}\Varid{exl}\mathrel{=}\Conid{Cont}\;( \bigtriangleup \mathbin{\circ}\Varid{inl}){}\<[E]%
\\
\>[B]{}\hsindent{3}{}\<[3]%
\>[3]{}\Varid{exr}\mathrel{=}\Conid{Cont}\;( \bigtriangleup \mathbin{\circ}\Varid{inr}){}\<[E]%
\\
\>[B]{}\hsindent{3}{}\<[3]%
\>[3]{}\Varid{dup}\mathrel{=}\Conid{Cont}\;(\Varid{jam}\mathbin{\circ}\Varid{unjoin}){}\<[E]%
\\[\blanklineskip]%
\>[B]{}\mathbf{instance}\;\Conid{Cocartesian}\;\Varid{k}\Rightarrow\Conid{Cocartesian}\;\Conid{Cont\char95 }\;\{\mskip1.5mu \Varid{k}\mskip1.5mu\}\mathbin{\uparrow}\{\mskip1.5mu \Varid{r}\mskip1.5mu\}\;\mathbf{where}{}\<[E]%
\\
\>[B]{}\hsindent{3}{}\<[3]%
\>[3]{}\Varid{inl}\mathrel{=}\Conid{Cont}\;(\Varid{exl}\mathbin{\circ}\Varid{unjoin}){}\<[E]%
\\
\>[B]{}\hsindent{3}{}\<[3]%
\>[3]{}\Varid{inr}\mathrel{=}\Conid{Cont}\;(\Varid{exr}\mathbin{\circ}\Varid{unjoin}){}\<[E]%
\\
\>[B]{}\hsindent{3}{}\<[3]%
\>[3]{}\Varid{jam}\mathrel{=}\Conid{Cont}\;( \bigtriangleup \mathbin{\circ}\Varid{dup}){}\<[E]%
\\[\blanklineskip]%
\>[B]{}\mathbf{instance}\;\Conid{Scalable}\;\Varid{k}\;\Varid{a}\Rightarrow\Conid{Scalable}\;\Conid{Cont\char95 }\;\{\mskip1.5mu \Varid{k}\mskip1.5mu\}\mathbin{\uparrow}\{\mskip1.5mu \Varid{r}\mskip1.5mu\}\;\Varid{a}\;\mathbf{where}{}\<[E]%
\\
\>[B]{}\hsindent{3}{}\<[3]%
\>[3]{}\Varid{scale}\;\Varid{s}\mathrel{=}\Conid{Cont}\;(\Varid{scale}\;\Varid{s}){}\<[E]%
\ColumnHook
\end{hscode}\resethooks
\end{frame}







\section{Gradient and Duality}
\begin{frame}{Esboço}

Continuando da secção anterior vamos falar de um caso especial de RAD: computação de gradientes.

Dado um espeço vetorial A sobre um campo escalar s o dual de A é A \multimap s(mapas lineares para o campo s).
Como este espaço dual também é vetorial e A é finito concluimos que tanto o dual como o primal são isomorfos.

Em particular, cada mapa de A \multimap s pode ser exprimido na forma dot u para u :: A onde:

\ensuremath{\mathbf{class}\;\Conid{HasDot}\mathbin{\uparrow}\{\mskip1.5mu \Conid{S}\mskip1.5mu\}\;\Varid{u}\;\mathbf{where}\;\Varid{dot}\mathbin{::}\Varid{u}\rightarrow (\Varid{u}\lambda \Varid{multimap}\;\Varid{s})}
\ensuremath{\mathbf{instance}\;\Conid{HasDot}\mathbin{\uparrow}\{\mskip1.5mu \Conid{IR}\mskip1.5mu\}\;\mathbf{where}\;\Varid{dot}\mathrel{=}\Varid{scale}}
\ensuremath{\mathbf{instance}\;(\Conid{HasDot}\mathbin{\uparrow}\{\mskip1.5mu \Conid{S}\mskip1.5mu\}\;\Varid{a},\Conid{HasDot}\mathbin{\uparrow}\{\mskip1.5mu \Conid{S}\mskip1.5mu\}\;\Varid{b})\Rightarrow\Conid{HasDot}\mathbin{\uparrow}\{\mskip1.5mu \Conid{S}\mskip1.5mu\}\;(\Varid{a}\lambda \Varid{times}\;\Varid{b})\;\mathbf{where}\;\Varid{dot}\;(\Varid{u},\Varid{v})\mathrel{=}\Varid{dot}\;\Varid{u}\; \bigtriangledown \;\Varid{dot}\;\Varid{v}}

Como o tipo \ensuremath{\Conid{Cont\char95 }\;\{\mskip1.5mu \Varid{k}\mskip1.5mu\}\mathbin{\uparrow}\{\mskip1.5mu \Varid{r}\mskip1.5mu\}} pode tomar qualquer tipo/objeto r, usaremos o valor do campo s para r.
Assim a representação interna de \ensuremath{\Conid{Cont\char95 }\;\{\mskip1.5mu \Varid{s}\mskip1.5mu\}\mathbin{\uparrow}\{\mskip1.5mu \lambda \Varid{multimap}\mskip1.5mu\}\;\Varid{a}\;\Varid{b}} é \ensuremath{(\Varid{b}\lambda \Varid{multimap}\;\Varid{s})\rightarrow (\Varid{a}\lambda \Varid{multimap}\;\Varid{s})}, que 
,pela isomorfia de s e seu espaço dual, é isomorfo a \ensuremath{\Varid{b}\rightarrow \Varid{a}}.
A esta representação chamaremos dual(ou oposto) de k:
\ensuremath{\mathbf{newtype}\;\Conid{Dual\char95 }\;\{\mskip1.5mu \Varid{k}\mskip1.5mu\}\;\Varid{a}\;\Varid{b}\mathrel{=}\Conid{Dual}\;(\Varid{b}\;\text{\tt 'k'}\;\Varid{a})}


Para criarmos representações duais de mapas lineares generalizados basta converter de \ensuremath{\Conid{Cont\char95 }\;\{\mskip1.5mu \Varid{k}\mskip1.5mu\}\mathbin{\uparrow}\{\mskip1.5mu \Varid{s}\mskip1.5mu\}} para
\ensuremath{\Conid{Dual\char95 }\;\{\mskip1.5mu \Varid{k}\mskip1.5mu\}} através de um funtor que iremos derivar a seguir, sendo que a composição deste com a função
\ensuremath{\Varid{cont}\mathbin{::}(\Varid{a'k'b})\rightarrow \Conid{Cont\char95 }\;\{\mskip1.5mu \Varid{k}\mskip1.5mu\}\mathbin{\uparrow}\{\mskip1.5mu \Varid{s}\mskip1.5mu\}\;\Varid{a}\;\Varid{b}} nos dará um funtor de k para \ensuremath{\Conid{Dual\char95 }\;\{\mskip1.5mu \Varid{k}\mskip1.5mu\}}.

O funtor em questão é

\begin{hscode}\SaveRestoreHook
\column{B}{@{}>{\hspre}l<{\hspost}@{}}%
\column{E}{@{}>{\hspre}l<{\hspost}@{}}%
\>[B]{}\Varid{asDual}\mathbin{::}(\Conid{HasDot}\mathbin{\uparrow}\{\mskip1.5mu \Conid{S}\mskip1.5mu\}\;\Varid{a},\Conid{HasDot}\mathbin{\uparrow}\{\mskip1.5mu \Conid{S}\mskip1.5mu\}\;\Varid{b})\Rightarrow((\Varid{b}\lambda \Varid{multimap}\;\Varid{s})\rightarrow (\Varid{a}\lambda \Varid{multimap}\;\Varid{s}))\rightarrow (\Varid{b}\lambda \Varid{multimap}\;\Varid{a}){}\<[E]%
\\
\>[B]{}\Varid{asDual}\;(\Conid{Cont}\;\Varid{f})\mathrel{=}\Conid{Dual}\;(\Varid{onDot}\;\Varid{f}){}\<[E]%
\ColumnHook
\end{hscode}\resethooks

,onde onDot é definido por:

\begin{hscode}\SaveRestoreHook
\column{B}{@{}>{\hspre}l<{\hspost}@{}}%
\column{E}{@{}>{\hspre}l<{\hspost}@{}}%
\>[B]{}\Varid{onDot}\mathbin{::}(\Conid{HasDot}\mathbin{\uparrow}\{\mskip1.5mu \Conid{S}\mskip1.5mu\}\;\Varid{a},\Conid{HasDot}\mathbin{\uparrow}\{\mskip1.5mu \Conid{S}\mskip1.5mu\}\;\Varid{b})\Rightarrow((\Varid{b}\lambda \Varid{multimap}\;\Varid{s})\rightarrow (\Varid{a}\lambda \Varid{multimap}\;\Varid{s}))\rightarrow (\Varid{b}\lambda \Varid{multimap}\;\Varid{a}){}\<[E]%
\\
\>[B]{}\Varid{onDot}\;\Varid{f}\mathrel{=}\Varid{dot}\mathbin{\uparrow}\{\mskip1.5mu \text{\tt \char34 -1\char34}\mskip1.5mu\}\mathbin{\circ}\Varid{f}\mathbin{\circ}\Varid{dot}{}\<[E]%
\ColumnHook
\end{hscode}\resethooks

A partir deste novo funtor derivamos as isntâncias de categoria correspondentes(ver figura 10)
e provamos que asDual é homomorfismo nessas instâncias.

Adicionalmente obtemos o seguinte corolário:

\ensuremath{\Conid{Dual}\;\Varid{f}\; \bigtriangleup \;\Conid{Dual}\;\Varid{g}\mathrel{=}\Conid{Dual}\;(\Varid{f}\; \bigtriangledown \;\Varid{g})}
\ensuremath{\Conid{Dual}\;\Varid{f}\; \bigtriangledown \;\Conid{Dual}\;\Varid{g}\mathrel{=}\Conid{Dual}\;(\Varid{f}\; \bigtriangleup \;\Varid{g})}

Note-se que com base na representação do capitulo 8 de matrizes podemos
concluir que a dualidade de matrizes corresponde à transposição das mesmas,
e que \ensuremath{\Conid{Dual\char95 }\;\{\mskip1.5mu \Varid{k}\mskip1.5mu\}} não involve calculos de matrizes sem que k também os envolva.


\end{frame}







\begin{frame}{A short introdution}

Due to it's widespread use in ML we'll talk about a specific case of RAD: computing gradients(derivatives of functions with scalar codomains)

A vector space A over a scalar field has \ensuremath{\Conid{A}\lambda \Varid{multimap}\;\Varid{s}} as it's dual(i.e., the linear maps of the udnerlaying field of A are it's dual)
This dual space is also a vector space and if A is finite in dimention they are isomorfic.

Each linear map in \ensuremath{\Conid{A}\lambda \Varid{multimap}\;\Varid{s}} can be represented in the form of dot u for some u :: A where

\ensuremath{\mathbf{class}\;\Conid{HasDot}\mathbin{\uparrow}\{\mskip1.5mu \Conid{S}\mskip1.5mu\}\;\Varid{u}\;\mathbf{where}\;\Varid{dot}\mathbin{::}\Varid{u}\rightarrow (\Varid{u}\lambda \Varid{multimap}\;\Varid{s})}
\ensuremath{\mathbf{instance}\;\Conid{HasDot}\mathbin{\uparrow}\{\mskip1.5mu \Conid{IR}\mskip1.5mu\}\;\Conid{IR}\;\mathbf{where}\;\Varid{dot}\mathrel{=}\Varid{scale}}
\ensuremath{\mathbf{instance}\;(\Conid{HasDot}\mathbin{\uparrow}\{\mskip1.5mu \Conid{S}\mskip1.5mu\}\;\Varid{a},\Conid{HasDot}\mathbin{\uparrow}\{\mskip1.5mu \Conid{S}\mskip1.5mu\}\;\Varid{b})\Rightarrow\Conid{HasDot}\mathbin{\uparrow}\{\mskip1.5mu \Conid{S}\mskip1.5mu\}\;(\Varid{a}\lambda \Varid{times}\;\Varid{b})\;\mathbf{where}\;\Varid{dot}\;(\Varid{u},\Varid{v})\mathrel{=}\Varid{dot}\;\Varid{u}\; \bigtriangledown \;\Varid{dot}\;\Varid{v}}


\end{frame}



\begin{frame}{Instance deduction}

Since \ensuremath{\Conid{Cont\char95 }\;\{\mskip1.5mu \Varid{k}\mskip1.5mu\}\mathbin{\uparrow}\{\mskip1.5mu \Varid{r}\mskip1.5mu\}} works for any type/object r we can use it with the scalar field s.
The internal representation of \ensuremath{\Conid{Cont\char95 }\;\{\mskip1.5mu \lambda \Varid{multimap}\mskip1.5mu\}\mathbin{\uparrow}\{\mskip1.5mu \Varid{s}\mskip1.5mu\}\;\Varid{a}\;\Varid{b}} is \ensuremath{(\Varid{b}\lambda \Varid{multimap}\;\Varid{s})\rightarrow (\Varid{a}\lambda \Varid{multimap}\;\Varid{s})\;\Varid{and}\;\Varid{that}\;\Varid{this}\;\Varid{is}\;\Varid{isomorfic}\;\Varid{to}\;(\Varid{a}\rightarrow \Varid{b})},
With this in mind we can call this representation as the dual/opposite of k:

\ensuremath{\mathbf{newtype}\;\Conid{Dual\char95 }\;\{\mskip1.5mu \Varid{k}\mskip1.5mu\}\;\Varid{a}\;\Varid{b}\mathrel{=}\Conid{Dual}\;(\Varid{b'k'a})}

\end{frame}




\begin{frame}{Instance deduction}

With this construction all we need to do to create dual representations of linear maps is to
convert from \ensuremath{\Conid{Cont\char95 }\;\{\mskip1.5mu \Varid{k}\mskip1.5mu\}\mathbin{\uparrow}\{\mskip1.5mu \Conid{S}\mskip1.5mu\}} to \ensuremath{\Conid{Dual\char95 }\;\{\mskip1.5mu \Varid{k}\mskip1.5mu\}} using a functor that we'll now derive:

\begin{hscode}\SaveRestoreHook
\column{B}{@{}>{\hspre}l<{\hspost}@{}}%
\column{E}{@{}>{\hspre}l<{\hspost}@{}}%
\>[B]{}\Varid{asDual}\mathbin{::}(\Conid{HasDot}\mathbin{\uparrow}\{\mskip1.5mu \Conid{S}\mskip1.5mu\}\;\Varid{a},\Conid{HasDot}\mathbin{\uparrow}\{\mskip1.5mu \Conid{S}\mskip1.5mu\}\;\Varid{b})\Rightarrow((\Varid{b}\lambda \Varid{multimap}\;\Varid{s})\rightarrow (\Varid{a}\lambda \Varid{multimap}\;\Varid{s}))\rightarrow (\Varid{b}\lambda \Varid{multimap}\;\Varid{a}){}\<[E]%
\\
\>[B]{}\Varid{asDual}\;(\Conid{Cont}\;\Varid{f})\mathrel{=}\Conid{Dual}\;(\Varid{onDot}\;\Varid{f}){}\<[E]%
\ColumnHook
\end{hscode}\resethooks

where

\begin{hscode}\SaveRestoreHook
\column{B}{@{}>{\hspre}l<{\hspost}@{}}%
\column{E}{@{}>{\hspre}l<{\hspost}@{}}%
\>[B]{}\Varid{onDot}\mathbin{::}(\Conid{HasDot}\mathbin{\uparrow}\{\mskip1.5mu \Conid{S}\mskip1.5mu\}\;\Varid{a},\Conid{HasDot}\mathbin{\uparrow}\{\mskip1.5mu \Conid{S}\mskip1.5mu\}\;\Varid{b})\Rightarrow((\Varid{b}\lambda \Varid{multimap}\;\Varid{s})\rightarrow (\Varid{a}\lambda \Varid{multimap}\;\Varid{s}))\rightarrow (\Varid{b}\lambda \Varid{multimap}\;\Varid{a}){}\<[E]%
\\
\>[B]{}\Varid{onDot}\;\Varid{f}\mathrel{=}\Varid{dot}\mathbin{\uparrow}\{\mskip1.5mu \text{\tt \char34 -1\char34}\mskip1.5mu\}\mathbin{\circ}\Varid{f}\mathbin{\circ}\Varid{dot}{}\<[E]%
\ColumnHook
\end{hscode}\resethooks

Given this we can now derive our new categorical instances

\end{frame}



\begin{frame}{Instance deduction}
\begin{hscode}\SaveRestoreHook
\column{B}{@{}>{\hspre}l<{\hspost}@{}}%
\column{3}{@{}>{\hspre}l<{\hspost}@{}}%
\column{E}{@{}>{\hspre}l<{\hspost}@{}}%
\>[B]{}\mathbf{instance}\;\Conid{Category}\;\Varid{k}\Rightarrow\Conid{Category}\;\Conid{Dual\char95 }\;\{\mskip1.5mu \Varid{k}\mskip1.5mu\}\;\mathbf{where}{}\<[E]%
\\
\>[B]{}\hsindent{3}{}\<[3]%
\>[3]{}\Varid{id}\mathrel{=}\Conid{Dual}\;\Varid{id}{}\<[E]%
\\
\>[B]{}\hsindent{3}{}\<[3]%
\>[3]{}\Conid{Dual}\;\Varid{g}\mathbin{\circ}\Conid{Dual}\;\Varid{f}\mathrel{=}\Conid{Dual}\;(\Varid{f}\mathbin{\circ}\Varid{g}){}\<[E]%
\\[\blanklineskip]%
\>[B]{}\mathbf{instance}\;\Conid{Monoidal}\;\Varid{k}\Rightarrow\Conid{Monoidal}\;\Conid{Dual\char95 }\;\{\mskip1.5mu \Varid{k}\mskip1.5mu\}\;\mathbf{where}{}\<[E]%
\\
\>[B]{}\hsindent{3}{}\<[3]%
\>[3]{}\Conid{Dual}\;\Varid{f}\;\Varid{x}\;\Conid{Dual}\;\Varid{g}\mathrel{=}\Conid{Dual}\;(\Varid{f}\;\Varid{x}\;\Varid{g}){}\<[E]%
\\[\blanklineskip]%
\>[B]{}\mathbf{instance}\;\Conid{Cartesian}\;\Varid{k}\Rightarrow\Conid{Cartesian}\;\Conid{Dual\char95 }\;\{\mskip1.5mu \Varid{k}\mskip1.5mu\}\;\mathbf{where}{}\<[E]%
\\
\>[B]{}\hsindent{3}{}\<[3]%
\>[3]{}\Varid{exl}\mathrel{=}\Conid{Dual}\;\Varid{inl}{}\<[E]%
\\
\>[B]{}\hsindent{3}{}\<[3]%
\>[3]{}\Varid{exr}\mathrel{=}\Conid{Dual}\;\Varid{inr}{}\<[E]%
\\
\>[B]{}\hsindent{3}{}\<[3]%
\>[3]{}\Varid{dup}\mathrel{=}\Conid{Dual}\;\Varid{jam}{}\<[E]%
\\[\blanklineskip]%
\>[B]{}\mathbf{instance}\;\Conid{Cocartesian}\;\Varid{k}\Rightarrow\Conid{Cocartesian}\;\Conid{Dual\char95 }\;\{\mskip1.5mu \Varid{k}\mskip1.5mu\}\;\mathbf{where}{}\<[E]%
\\
\>[B]{}\hsindent{3}{}\<[3]%
\>[3]{}\Varid{inl}\mathrel{=}\Conid{Dual}\;\Varid{exl}{}\<[E]%
\\
\>[B]{}\hsindent{3}{}\<[3]%
\>[3]{}\Varid{inr}\mathrel{=}\Conid{Dual}\;\Varid{exr}{}\<[E]%
\\
\>[B]{}\hsindent{3}{}\<[3]%
\>[3]{}\Varid{jam}\mathrel{=}\Conid{Dual}\;\Varid{dup}{}\<[E]%
\\[\blanklineskip]%
\>[B]{}\mathbf{instance}\;\Conid{Scalable}\;\Varid{k}\Rightarrow\Conid{Scalable}\;\Conid{Dual\char95 }\;\{\mskip1.5mu \Varid{k}\mskip1.5mu\}\;\mathbf{where}{}\<[E]%
\\
\>[B]{}\hsindent{3}{}\<[3]%
\>[3]{}\Varid{scale}\;\Varid{s}\mathrel{=}\Conid{Dual}\;(\Varid{scale}\;\Varid{s}){}\<[E]%
\ColumnHook
\end{hscode}\resethooks
\end{frame}



\begin{frame}{Final notes}

\begin{itemize}
  \item \ensuremath{ \bigtriangleup \;\Varid{and}\; \bigtriangledown } mutually dualize \ensuremath{(\Conid{Dual}\;\Varid{f}\; \bigtriangleup \;\Conid{Dual}\;\Varid{g}\mathrel{=}\Conid{Dual}\;(\Varid{f}\; \bigtriangledown \;\Varid{g})\;\Varid{and}\;\Conid{Dual}\;\Varid{f}\; \bigtriangledown \;\Conid{Dual}\;\Varid{g}\mathrel{=}\Conid{Dual}\;(\Varid{f}\; \bigtriangleup \;\Varid{g}))}
  \item Using the definition from chapter 8 we can determine that the duality of a matrix corresponds to it's transposition
\end{itemize}

\end{frame}

















\section{Foward-Mode Automatic Differentiation(FAD)}
\begin{frame}{Esboço}

Tendo criado o Cont e o Dual podemos aplicar as mesmas técnicas para tentar criar composição
totalmente à direita de modo a definir um algoritmo generalizado para FAD que é mais apropriado
a dominios de baixa dimensão.

\ensuremath{\mathbf{newtype}\;\Conid{Begin\char95 }\;\{\mskip1.5mu \Varid{k}\mskip1.5mu\}\mathbin{\uparrow}\{\mskip1.5mu \Varid{r}\mskip1.5mu\}\;\Varid{a}\;\Varid{b}\mathrel{=}\Conid{Begin}\;((\Varid{r}\;\text{\tt 'k'}\;\Varid{a})\rightarrow (\Varid{r}\;\text{\tt 'k'}\;\Varid{b}))}

\begin{hscode}\SaveRestoreHook
\column{B}{@{}>{\hspre}l<{\hspost}@{}}%
\column{E}{@{}>{\hspre}l<{\hspost}@{}}%
\>[B]{}\Varid{begin}\mathbin{::}\Conid{Category}\;\Varid{k}\Rightarrow(\Varid{a}\;\text{\tt 'k'}\;\Varid{b})\rightarrow \Conid{Begin\char95 }\;\{\mskip1.5mu \Varid{k}\mskip1.5mu\}\mathbin{\uparrow}\{\mskip1.5mu \Varid{r}\mskip1.5mu\}\;\Varid{a}\;\Varid{b}{}\<[E]%
\\
\>[B]{}\Varid{begin}\;\Varid{f}\mathrel{=}\Conid{Begin}\;(\Varid{f}\mathbin{\circ}){}\<[E]%
\ColumnHook
\end{hscode}\resethooks


Após termos isto definido podemos aplicar o mesmo processo que feito anteriormente por especificação
homomorfica para begin, e escolhendo r como campo escalar de s notando que \ensuremath{\Varid{s}\lambda \Varid{multimap}\;\Varid{a}} é isomorfico a a



\end{frame}







\begin{frame}{Fowards-mode Automatic Differentiation(FAD)}

We can use the same deductions we've done in Cont and Dual to derive a category with full right-side association, thus creating a generized FAD algorithm.
This algorithm is far more apropriated for low dimention domains.

\ensuremath{\mathbf{newtype}\;\Conid{Begin\char95 }\;\{\mskip1.5mu \Varid{k}\mskip1.5mu\}\mathbin{\uparrow}\{\mskip1.5mu \Varid{r}\mskip1.5mu\}\;\Varid{a}\;\Varid{b}\mathrel{=}\Conid{Begin}\;((\Varid{r'k'a})\rightarrow (\Varid{r'k'b}))}

\begin{hscode}\SaveRestoreHook
\column{B}{@{}>{\hspre}l<{\hspost}@{}}%
\column{E}{@{}>{\hspre}l<{\hspost}@{}}%
\>[B]{}\Varid{begin}\mathbin{::}\Conid{Category}\;\Varid{k}\Rightarrow(\Varid{a}\;\text{\tt 'k'}\;\Varid{b})\rightarrow \Conid{Begin\char95 }\;\{\mskip1.5mu \Varid{k}\mskip1.5mu\}\mathbin{\uparrow}\{\mskip1.5mu \Varid{r}\mskip1.5mu\}\;\Varid{a}\;\Varid{b}{}\<[E]%
\\
\>[B]{}\Varid{begin}\;\Varid{f}\mathrel{=}\Conid{Begin}\;(\Varid{f}\mathbin{\circ}){}\<[E]%
\ColumnHook
\end{hscode}\resethooks

We can derive categorical instances from the functor above and we can choose r to be the scalar field s, noting that \ensuremath{\Varid{s}\lambda \Varid{multimap}\;\Varid{a}} is isomorfic to a.


\end{frame}














\end{document}


