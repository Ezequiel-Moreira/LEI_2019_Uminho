\documentclass{beamer}

\mode<presentation>
{
  \usetheme{Warsaw}

  \setbeamercovered{transparent}
}


\usepackage[english]{babel}
\usepackage[utf8x]{inputenc}
%\usepackage[latin1]{inputenc}

\usepackage{times}
\usepackage[T1]{fontenc}
% Note that the encoding and the font should match. If T1
% does not look nice, try deleting the line with the fontenc.

%---------- lhs2tex ---------------------------------
%% ODER: format ==         = "\mathrel{==}"
%% ODER: format /=         = "\neq "
%
%
\makeatletter
\@ifundefined{lhs2tex.lhs2tex.sty.read}%
  {\@namedef{lhs2tex.lhs2tex.sty.read}{}%
   \newcommand\SkipToFmtEnd{}%
   \newcommand\EndFmtInput{}%
   \long\def\SkipToFmtEnd#1\EndFmtInput{}%
  }\SkipToFmtEnd

\newcommand\ReadOnlyOnce[1]{\@ifundefined{#1}{\@namedef{#1}{}}\SkipToFmtEnd}
\usepackage{amstext}
\usepackage{amssymb}
\usepackage{stmaryrd}
\DeclareFontFamily{OT1}{cmtex}{}
\DeclareFontShape{OT1}{cmtex}{m}{n}
  {<5><6><7><8>cmtex8
   <9>cmtex9
   <10><10.95><12><14.4><17.28><20.74><24.88>cmtex10}{}
\DeclareFontShape{OT1}{cmtex}{m}{it}
  {<-> ssub * cmtt/m/it}{}
\newcommand{\texfamily}{\fontfamily{cmtex}\selectfont}
\DeclareFontShape{OT1}{cmtt}{bx}{n}
  {<5><6><7><8>cmtt8
   <9>cmbtt9
   <10><10.95><12><14.4><17.28><20.74><24.88>cmbtt10}{}
\DeclareFontShape{OT1}{cmtex}{bx}{n}
  {<-> ssub * cmtt/bx/n}{}
\newcommand{\tex}[1]{\text{\texfamily#1}}	% NEU

\newcommand{\Sp}{\hskip.33334em\relax}


\newcommand{\Conid}[1]{\mathit{#1}}
\newcommand{\Varid}[1]{\mathit{#1}}
\newcommand{\anonymous}{\kern0.06em \vbox{\hrule\@width.5em}}
\newcommand{\plus}{\mathbin{+\!\!\!+}}
\newcommand{\bind}{\mathbin{>\!\!\!>\mkern-6.7mu=}}
\newcommand{\rbind}{\mathbin{=\mkern-6.7mu<\!\!\!<}}% suggested by Neil Mitchell
\newcommand{\sequ}{\mathbin{>\!\!\!>}}
\renewcommand{\leq}{\leqslant}
\renewcommand{\geq}{\geqslant}
\usepackage{polytable}

%mathindent has to be defined
\@ifundefined{mathindent}%
  {\newdimen\mathindent\mathindent\leftmargini}%
  {}%

\def\resethooks{%
  \global\let\SaveRestoreHook\empty
  \global\let\ColumnHook\empty}
\newcommand*{\savecolumns}[1][default]%
  {\g@addto@macro\SaveRestoreHook{\savecolumns[#1]}}
\newcommand*{\restorecolumns}[1][default]%
  {\g@addto@macro\SaveRestoreHook{\restorecolumns[#1]}}
\newcommand*{\aligncolumn}[2]%
  {\g@addto@macro\ColumnHook{\column{#1}{#2}}}

\resethooks

\newcommand{\onelinecommentchars}{\quad-{}- }
\newcommand{\commentbeginchars}{\enskip\{-}
\newcommand{\commentendchars}{-\}\enskip}

\newcommand{\visiblecomments}{%
  \let\onelinecomment=\onelinecommentchars
  \let\commentbegin=\commentbeginchars
  \let\commentend=\commentendchars}

\newcommand{\invisiblecomments}{%
  \let\onelinecomment=\empty
  \let\commentbegin=\empty
  \let\commentend=\empty}

\visiblecomments

\newlength{\blanklineskip}
\setlength{\blanklineskip}{0.66084ex}

\newcommand{\hsindent}[1]{\quad}% default is fixed indentation
\let\hspre\empty
\let\hspost\empty
\newcommand{\NB}{\textbf{NB}}
\newcommand{\Todo}[1]{$\langle$\textbf{To do:}~#1$\rangle$}

\EndFmtInput
\makeatother
%
%
%
%
%
%
% This package provides two environments suitable to take the place
% of hscode, called "plainhscode" and "arrayhscode". 
%
% The plain environment surrounds each code block by vertical space,
% and it uses \abovedisplayskip and \belowdisplayskip to get spacing
% similar to formulas. Note that if these dimensions are changed,
% the spacing around displayed math formulas changes as well.
% All code is indented using \leftskip.
%
% Changed 19.08.2004 to reflect changes in colorcode. Should work with
% CodeGroup.sty.
%
\ReadOnlyOnce{polycode.fmt}%
\makeatletter

\newcommand{\hsnewpar}[1]%
  {{\parskip=0pt\parindent=0pt\par\vskip #1\noindent}}

% can be used, for instance, to redefine the code size, by setting the
% command to \small or something alike
\newcommand{\hscodestyle}{}

% The command \sethscode can be used to switch the code formatting
% behaviour by mapping the hscode environment in the subst directive
% to a new LaTeX environment.

\newcommand{\sethscode}[1]%
  {\expandafter\let\expandafter\hscode\csname #1\endcsname
   \expandafter\let\expandafter\endhscode\csname end#1\endcsname}

% "compatibility" mode restores the non-polycode.fmt layout.

\newenvironment{compathscode}%
  {\par\noindent
   \advance\leftskip\mathindent
   \hscodestyle
   \let\\=\@normalcr
   \let\hspre\(\let\hspost\)%
   \pboxed}%
  {\endpboxed\)%
   \par\noindent
   \ignorespacesafterend}

\newcommand{\compaths}{\sethscode{compathscode}}

% "plain" mode is the proposed default.
% It should now work with \centering.
% This required some changes. The old version
% is still available for reference as oldplainhscode.

\newenvironment{plainhscode}%
  {\hsnewpar\abovedisplayskip
   \advance\leftskip\mathindent
   \hscodestyle
   \let\hspre\(\let\hspost\)%
   \pboxed}%
  {\endpboxed%
   \hsnewpar\belowdisplayskip
   \ignorespacesafterend}

\newenvironment{oldplainhscode}%
  {\hsnewpar\abovedisplayskip
   \advance\leftskip\mathindent
   \hscodestyle
   \let\\=\@normalcr
   \(\pboxed}%
  {\endpboxed\)%
   \hsnewpar\belowdisplayskip
   \ignorespacesafterend}

% Here, we make plainhscode the default environment.

\newcommand{\plainhs}{\sethscode{plainhscode}}
\newcommand{\oldplainhs}{\sethscode{oldplainhscode}}
\plainhs

% The arrayhscode is like plain, but makes use of polytable's
% parray environment which disallows page breaks in code blocks.

\newenvironment{arrayhscode}%
  {\hsnewpar\abovedisplayskip
   \advance\leftskip\mathindent
   \hscodestyle
   \let\\=\@normalcr
   \(\parray}%
  {\endparray\)%
   \hsnewpar\belowdisplayskip
   \ignorespacesafterend}

\newcommand{\arrayhs}{\sethscode{arrayhscode}}

% The mathhscode environment also makes use of polytable's parray 
% environment. It is supposed to be used only inside math mode 
% (I used it to typeset the type rules in my thesis).

\newenvironment{mathhscode}%
  {\parray}{\endparray}

\newcommand{\mathhs}{\sethscode{mathhscode}}

% texths is similar to mathhs, but works in text mode.

\newenvironment{texthscode}%
  {\(\parray}{\endparray\)}

\newcommand{\texths}{\sethscode{texthscode}}

% The framed environment places code in a framed box.

\def\codeframewidth{\arrayrulewidth}
\RequirePackage{calc}

\newenvironment{framedhscode}%
  {\parskip=\abovedisplayskip\par\noindent
   \hscodestyle
   \arrayrulewidth=\codeframewidth
   \tabular{@{}|p{\linewidth-2\arraycolsep-2\arrayrulewidth-2pt}|@{}}%
   \hline\framedhslinecorrect\\{-1.5ex}%
   \let\endoflinesave=\\
   \let\\=\@normalcr
   \(\pboxed}%
  {\endpboxed\)%
   \framedhslinecorrect\endoflinesave{.5ex}\hline
   \endtabular
   \parskip=\belowdisplayskip\par\noindent
   \ignorespacesafterend}

\newcommand{\framedhslinecorrect}[2]%
  {#1[#2]}

\newcommand{\framedhs}{\sethscode{framedhscode}}

% The inlinehscode environment is an experimental environment
% that can be used to typeset displayed code inline.

\newenvironment{inlinehscode}%
  {\(\def\column##1##2{}%
   \let\>\undefined\let\<\undefined\let\\\undefined
   \newcommand\>[1][]{}\newcommand\<[1][]{}\newcommand\\[1][]{}%
   \def\fromto##1##2##3{##3}%
   \def\nextline{}}{\) }%

\newcommand{\inlinehs}{\sethscode{inlinehscode}}

% The joincode environment is a separate environment that
% can be used to surround and thereby connect multiple code
% blocks.

\newenvironment{joincode}%
  {\let\orighscode=\hscode
   \let\origendhscode=\endhscode
   \def\endhscode{\def\hscode{\endgroup\def\@currenvir{hscode}\\}\begingroup}
   %\let\SaveRestoreHook=\empty
   %\let\ColumnHook=\empty
   %\let\resethooks=\empty
   \orighscode\def\hscode{\endgroup\def\@currenvir{hscode}}}%
  {\origendhscode
   \global\let\hscode=\orighscode
   \global\let\endhscode=\origendhscode}%

\makeatother
\EndFmtInput
%
%----------------------------------------------------

\newenvironment{slide}[1]{\begin{frame}\frametitle{#1}}{\end{frame}}

\title
{...Machine Learning...}

\author[Artur, Ezequiel, Nelson] 
{Artur \and Ezequiel \and Nelson}

\institute
{Universidade do Minho}

\date
{26 de Abril}

\subject{Talks}

% If you have a file called "university-logo-filename.xxx", where xxx
% is a graphic format that can be processed by latex or pdflatex,
% resp., then you can add a logo as follows:

% \pgfdeclareimage[height=0.5cm]{university-logo}{university-logo-filename}
% \logo{\pgfuseimage{university-logo}}

% Delete this, if you do not want the table of contents to pop up at
% the beginning of each subsection:

\AtBeginSubsection[]
{
  \begin{frame}<beamer>{Outline}
    \tableofcontents[currentsection,currentsubsection]
  \end{frame}
}


% If you wish to uncover everything in a step-wise fashion, uncomment
% the following command: 

%\beamerdefaultoverlayspecification{<+->}


\begin{document}

\begin{frame}
  \titlepage
\end{frame}

\begin{frame}{Indice}
  \tableofcontents
\end{frame}




\section{Categories}

\begin{frame}{A short introdution}
\begin{itemize}
 \item<1-> We want to calculate \ensuremath{\mathcal{D}}.
 \item<2-> However, \ensuremath{\mathcal{D}} is not computable.
 \item<3-> Solution: reimplement corollaries using category theory
\end{itemize}

\end{frame}



\begin{frame}{A short introdution}

    \begin{block}{Corollary 1.1}
    NOTA: adicionar definição do corolário 1.1 aqui
    \end{block}
    
    \begin{block}{Corollary 2.1}
    NOTA: adicionar definição do corolário 2.1 aqui
    \end{block}
    
    \begin{block}{Corollary 3.1}
    NOTA: adicionar definição do corolário 3.1 aqui
    \end{block}
 
\end{frame}




\begin{frame}{Categories}

A category is a collection of objects(sets and types) and morphisms(operation between objects),
with 2 basic operations(identity and composition) of morfisms, and 2 laws:

\begin{itemize}
    \item (C.1)  $id \circ f = id \circ f = f$
    \item (C.2)  $f \circ (g \circ h) = (f \circ g) \circ h$
\end{itemize}

\begin{block}{Note}
For this paper, objects are data types and morfisms are functions
\end{block}

\begin{columns}
\begin{column}{0.5\textwidth}
\begin{hscode}\SaveRestoreHook
\column{B}{@{}>{\hspre}l<{\hspost}@{}}%
\column{5}{@{}>{\hspre}l<{\hspost}@{}}%
\column{E}{@{}>{\hspre}l<{\hspost}@{}}%
\>[B]{}\mathbf{class}\;\Conid{Category}\;\Varid{k}\;\mathbf{where}{}\<[E]%
\\
\>[B]{}\hsindent{5}{}\<[5]%
\>[5]{}\Varid{id}\mathbin{::}(\Varid{a'k'a}){}\<[E]%
\\
\>[B]{}\hsindent{5}{}\<[5]%
\>[5]{}(\mathbin{\circ})\mathbin{::}(\Varid{b'k'c})\rightarrow (\Varid{a'k'b})\rightarrow (\Varid{a'k'c}){}\<[E]%
\ColumnHook
\end{hscode}\resethooks
\end{column}

\begin{column}{0.5\textwidth}
\begin{hscode}\SaveRestoreHook
\column{B}{@{}>{\hspre}l<{\hspost}@{}}%
\column{5}{@{}>{\hspre}l<{\hspost}@{}}%
\column{E}{@{}>{\hspre}l<{\hspost}@{}}%
\>[B]{}\mathbf{instance}\;\Conid{Category}\;(\rightarrow )\;\mathbf{where}{}\<[E]%
\\
\>[B]{}\hsindent{5}{}\<[5]%
\>[5]{}\Varid{id}\mathrel{=}\lambda \Varid{a}\rightarrow \Varid{a}{}\<[E]%
\\
\>[B]{}\hsindent{5}{}\<[5]%
\>[5]{}\Varid{g}\mathbin{\circ}\Varid{f}\mathrel{=}\lambda \Varid{a}\rightarrow \Varid{g}\;(\Varid{f}\;\Varid{a}){}\<[E]%
\ColumnHook
\end{hscode}\resethooks
\end{column}
\end{columns}
\end{frame}




\begin{frame}{Functors}

A functor F between 2 categories \ensuremath{\mathcal{U}\;\Varid{and}\;\mathcal{V}} is such that:
\begin{itemize}
    \item given any object t $\in$ \ensuremath{\mathcal{U}} there exists an object F t $\in$ \ensuremath{\mathcal{V}}
    \item given any morphism m :: a \ensuremath{\rightarrow } b $\in$ \ensuremath{\mathcal{U}} there exists a morphism F m :: F a \ensuremath{\rightarrow } F b $\in$ \ensuremath{\mathcal{V}}
    \item F id ($\in$ \ensuremath{\mathcal{U}}) = id ($\in$ \ensuremath{\mathcal{V}})
    \item F (f $\circ$ g) = F f $\circ$ F g
\end{itemize}

\begin{block}{Note}
Given this papers category properties(objects are data types) functors map types to themselves
\end{block}

\end{frame}



\begin{frame}{Objective}

Let's start by defining a new data type:

newtype \ensuremath{\mathcal{D}} a b = \ensuremath{\mathcal{D}} (a \ensuremath{\rightarrow } b $\times$ (a $\multimap$ b))

,and adapting \ensuremath{\mathcal{D}^{+}} to use it:

\begin{block}{Adapted definition}
\begin{hscode}\SaveRestoreHook
\column{B}{@{}>{\hspre}l<{\hspost}@{}}%
\column{E}{@{}>{\hspre}l<{\hspost}@{}}%
\>[B]{}\mathcal{\hat{D}}\mathbin{::}(\Varid{a}\rightarrow \Varid{b})\rightarrow \mathcal{D}\;\Varid{a}\;\Varid{b}{}\<[E]%
\\
\>[B]{}\mathcal{\hat{D}}\;\Varid{f}\mathrel{=}\mathcal{D}\;(\mathcal{D}^{+}\;\Varid{f}){}\<[E]%
\ColumnHook
\end{hscode}\resethooks
\end{block}

Our objective is to deduce an instance of a Category for \ensuremath{\mathcal{D}} where \ensuremath{\mathcal{\hat{D}}} is a functor.

\end{frame}




\begin{frame}{Instance deduction}

Before deducing our instance we must first note that using corollaries 3.1 and 1.1 we can determine that

\begin{itemize}
    \item (DP.1) \ensuremath{\mathcal{D}^{+}} id = $\lambda$ a -> (id a,id)
    \item (DP.2)    
    \ensuremath{\mathcal{D}^{+}}(g $\circ$ f) = $\lambda$ a -> let{(b,f') = \ensuremath{\mathcal{D}^{+}} f a; (c,g') = \ensuremath{\mathcal{D}^{+}} g b } 
        in (c,g' $\circ$ f')
\end{itemize}

Saying that \ensuremath{\mathcal{\hat{D}}} is a functor is equivalent to, for all f and g functions of apropriate types:

    \ensuremath{\Varid{id}\mathrel{=}\mathcal{\hat{D}}\;\Varid{id}\mathrel{=}\mathcal{D}\;(\mathcal{D}^{+}\;\Varid{id})}

    \ensuremath{\mathcal{\hat{D}}} g $\circ$ \ensuremath{\mathcal{\hat{D}}} f = \ensuremath{\mathcal{\hat{D}}}  (g $\circ$ f) = \ensuremath{\mathcal{D}} (\ensuremath{\mathcal{\hat{D}}} (g $\circ$ f))


\end{frame}



\begin{frame}{Instance deduction}

Based on  (DP.1) and (DP.2) we'll rewrite the above into the following definition:

id = \ensuremath{\mathcal{D}} ($\lambda$ a -> (id a,id))

\ensuremath{\mathcal{\hat{D}}} g $\circ$ \ensuremath{\mathcal{\hat{D}}} f = \ensuremath{\mathcal{D}} ($\lambda$ a -> let{(b,f') = \ensuremath{\mathcal{D}^{+}} f a; (c,g') = \ensuremath{\mathcal{D}^{+}} g b} in (c,g' $\circ$ f'))


The first equation shown above has a trivial solution(define id of instance as \ensuremath{\mathcal{D}}($\lambda$ a -> (id a,id)))

To solve the second we'll first solve a more general one:

\ensuremath{\mathcal{D}} g $\circ$ \ensuremath{\mathcal{D}} f = \ensuremath{\mathcal{D}}($\lambda$ a -> let{(b,f') = f a; (c,g') = g b } in(c,g' $\circ$ f'))

This condition also leads us to a trivial solution inside our instance.

\end{frame}





\begin{frame}{Instance deduction}

\begin{block}{ \ensuremath{\mathcal{\hat{D}}} definition for linear functions}
\begin{hscode}\SaveRestoreHook
\column{B}{@{}>{\hspre}l<{\hspost}@{}}%
\column{E}{@{}>{\hspre}l<{\hspost}@{}}%
\>[B]{}\Varid{linearD}\mathbin{::}(\Varid{a}\rightarrow \Varid{b})\rightarrow \mathcal{D}\;\Varid{a}\;\Varid{b}{}\<[E]%
\\
\>[B]{}\Varid{linearD}\;\Varid{f}\mathrel{=}\mathcal{D}\;(\lambda \Varid{a}\rightarrow (\Varid{f}\;\Varid{a},\Varid{f})){}\<[E]%
\ColumnHook
\end{hscode}\resethooks
\end{block}


\begin{block}{Categorical instance we've deduced}
\begin{hscode}\SaveRestoreHook
\column{B}{@{}>{\hspre}l<{\hspost}@{}}%
\column{5}{@{}>{\hspre}l<{\hspost}@{}}%
\column{E}{@{}>{\hspre}l<{\hspost}@{}}%
\>[B]{}\mathbf{instance}\;\Conid{Category}\;\mathcal{D}\;\mathbf{where}{}\<[E]%
\\
\>[B]{}\hsindent{5}{}\<[5]%
\>[5]{}\Varid{id}\mathrel{=}\Varid{linearD}\;\Varid{id}{}\<[E]%
\\
\>[B]{}\hsindent{5}{}\<[5]%
\>[5]{}\mathcal{D}\;\Varid{g}\mathbin{\circ}\mathcal{D}\;\Varid{f}\mathrel{=}{}\<[E]%
\\
\>[B]{}\hsindent{5}{}\<[5]%
\>[5]{}\mathcal{D}\;(\lambda \Varid{a}\rightarrow \mathbf{let}\;\{\mskip1.5mu (\Varid{b},\Varid{f'})\mathrel{=}\Varid{f}\;\Varid{a};(\Varid{c},\Varid{g'})\mathrel{=}\Varid{g}\;\Varid{b}\mskip1.5mu\}\;\mathbf{in}\;(\Varid{c},\Varid{g'}\mathbin{\circ}\Varid{f'})){}\<[E]%
\ColumnHook
\end{hscode}\resethooks
\end{block}
\end{frame}




\begin{frame}{Instance proof}

In order to prove that the instance is correct we must check if it follows laws (C.1) and (C.2).

First we must make a concession: that we only use morfisms arising from \ensuremath{\mathcal{D}^{+}}(we can force this by transforming \ensuremath{\mathcal{D}} into an abstract type).
If we do, then \ensuremath{\mathcal{D}^{+}} is a functor.


\begin{block}{(C.1) proof}

id $\circ$ \ensuremath{\mathcal{\hat{D}}}

= \ensuremath{\mathcal{\hat{D}}} id $\circ$ \ensuremath{\mathcal{\hat{D}}} f - functor law for id (specification of \ensuremath{\mathcal{\hat{D}}})

= \ensuremath{\mathcal{\hat{D}}} (id $\circ$ f) - functor law for ($\circ$)

= \ensuremath{\mathcal{\hat{D}}} f - categorical law
\end{block}

\end{frame}


\begin{frame}{Instance proof}

\begin{block}{(C.2) proof}

\ensuremath{\mathcal{\hat{D}}} h $\circ$ (\ensuremath{\mathcal{\hat{D}}} g $\circ$ \ensuremath{\mathcal{\hat{D}}} f)

= \ensuremath{\mathcal{\hat{D}}} h $\circ$ \ensuremath{\mathcal{\hat{D}}} (g $\circ$ f) - functor law for ($\circ$)

= \ensuremath{\mathcal{\hat{D}}} (h $\circ$ (g $\circ$ f)) - functor law for ($\circ$)

= \ensuremath{\mathcal{\hat{D}}} ((h $\circ$ g) $\circ$ f) - categorical law

= \ensuremath{\mathcal{\hat{D}}} (h $\circ$ g) $\circ$ \ensuremath{\mathcal{\hat{D}}} f - functor law for ($\circ$)

= (\ensuremath{\mathcal{\hat{D}}} h $\circ$ \ensuremath{\mathcal{\hat{D}}} g) $\circ$ \ensuremath{\mathcal{\hat{D}}} f - functor law for ($\circ$)

\end{block}

\begin{alertblock}{Note}
This proofs don't require anything from \ensuremath{\mathcal{D}\;\Varid{and}\;\mathcal{\hat{D}}} aside from functor laws.
As such, all other instances of categories created from a functor won't require further proving like this onr did.

\end{alertblock}
\end{frame}





\begin{frame}{Monoidal categories and functors}

Generalized parallel composition shall be defined using a monoidal category:


\begin{hscode}\SaveRestoreHook
\column{B}{@{}>{\hspre}l<{\hspost}@{}}%
\column{5}{@{}>{\hspre}l<{\hspost}@{}}%
\column{E}{@{}>{\hspre}l<{\hspost}@{}}%
\>[B]{}\mathbf{class}\;\Conid{Category}\;\Varid{k}\Rightarrow\Conid{Monoidal}\;\Varid{k}\;\mathbf{where}{}\<[E]%
\\
\>[B]{}\hsindent{5}{}\<[5]%
\>[5]{}(\Varid{x})\mathbin{::}(\Varid{a}\;\text{\tt 'k'}\;\Varid{c})\rightarrow (\Varid{b}\;\text{\tt 'k'}\;\Varid{d})\rightarrow ((\Varid{a}\;\Varid{x}\;\Varid{b})\;\text{\tt 'k'}\;(\Varid{c}\;\Varid{x}\;\Varid{d})){}\<[E]%
\ColumnHook
\end{hscode}\resethooks

\begin{hscode}\SaveRestoreHook
\column{B}{@{}>{\hspre}l<{\hspost}@{}}%
\column{5}{@{}>{\hspre}l<{\hspost}@{}}%
\column{E}{@{}>{\hspre}l<{\hspost}@{}}%
\>[B]{}\mathbf{instance}\;\Conid{Monoidal}\;(\rightarrow )\;\mathbf{where}{}\<[E]%
\\
\>[B]{}\hsindent{5}{}\<[5]%
\>[5]{}\Varid{f}\;\Varid{x}\;\Varid{g}\mathrel{=}\lambda (\Varid{a},\Varid{b})\rightarrow (\Varid{f}\;\Varid{a},\Varid{g}\;\Varid{b}){}\<[E]%
\ColumnHook
\end{hscode}\resethooks



\begin{block}{Monoidal Functor definition}

A monoidal functor F between categories \ensuremath{\mathcal{U}\;\Varid{and}\;\mathcal{V}} is such that:
\begin{itemize}
    \item F is a functor
    \item F (f $\times$ g) = F f $\times$ F g
\end{itemize}
\end{block}
\end{frame}


\begin{frame}{Instance deduction}

From corollary 2.1 we can deduce that:

\ensuremath{\mathcal{D}^{+}} (f $\times$ g) = $\lambda$ (a,b) -> let{(c,f')=\ensuremath{\mathcal{D}^{+}} f a;(d,g')= \ensuremath{\mathcal{D}^{+}} g b} 

    in ((c,d),f' $\times$ g')

Deriving F from \ensuremath{\mathcal{\hat{D}}} leaves us with the following definition:

\ensuremath{\mathcal{D}} (\ensuremath{\mathcal{D}^{+}} f) $\times$ \ensuremath{\mathcal{D}} (\ensuremath{\mathcal{D}^{+}} g) = \ensuremath{\mathcal{D}} (\ensuremath{\mathcal{D}^{+}} (f $\times$ g))

Using the same method as before, we replace \ensuremath{\mathcal{D}^{+}} with it's definition and generalize the condition:

\ensuremath{\mathcal{D}} f $\times$ \ensuremath{\mathcal{D}} g =

\ensuremath{\mathcal{D}} ($\lambda$ (a,b) -> let{(c,f') = f a; (d,g') = g b} in ((c,d),f' $\times$ g'))

and this is enough for our new instance.
\end{frame}



\begin{frame}{Instance deduction}
\begin{block}{Categorical instance we've deduced}
\begin{hscode}\SaveRestoreHook
\column{B}{@{}>{\hspre}l<{\hspost}@{}}%
\column{5}{@{}>{\hspre}l<{\hspost}@{}}%
\column{38}{@{}>{\hspre}l<{\hspost}@{}}%
\column{E}{@{}>{\hspre}l<{\hspost}@{}}%
\>[B]{}\mathbf{instance}\;\Conid{Monoidal}\;\mathcal{D}\;\mathbf{where}{}\<[E]%
\\
\>[B]{}\hsindent{5}{}\<[5]%
\>[5]{}\mathcal{D}\;\Varid{f}\;\Varid{x}\;\mathcal{D}\;\Varid{g}\mathrel{=}\mathcal{D}\;(\lambda (\Varid{a},\Varid{b})\rightarrow \mathbf{let}\;\{\mskip1.5mu (\Varid{c},\Varid{f'})\mathrel{=}\Varid{f}\;\Varid{a};(\Varid{d},\Varid{g'})\mathrel{=}\Varid{g}\;\Varid{b}\mskip1.5mu\}{}\<[E]%
\\
\>[5]{}\hsindent{33}{}\<[38]%
\>[38]{}\mathbf{in}\;((\Varid{c},\Varid{d}),\Varid{f'}\;\Varid{x}\;\Varid{g'})){}\<[E]%
\ColumnHook
\end{hscode}\resethooks
\end{block}
\end{frame}




\begin{frame}{Cartesian categories and functors}
\begin{hscode}\SaveRestoreHook
\column{B}{@{}>{\hspre}l<{\hspost}@{}}%
\column{5}{@{}>{\hspre}l<{\hspost}@{}}%
\column{E}{@{}>{\hspre}l<{\hspost}@{}}%
\>[B]{}\mathbf{class}\;\Conid{Monoidal}\;\Varid{k}\Rightarrow\Conid{Cartesian}\;\Varid{k}\;\mathbf{where}{}\<[E]%
\\
\>[B]{}\hsindent{5}{}\<[5]%
\>[5]{}\Varid{exl}\mathbin{::}(\Varid{a},\Varid{b})\;\text{\tt 'k'}\;\Varid{a}{}\<[E]%
\\
\>[B]{}\hsindent{5}{}\<[5]%
\>[5]{}\Varid{exr}\mathbin{::}(\Varid{a},\Varid{b})\;\text{\tt 'k'}\;\Varid{b}{}\<[E]%
\\
\>[B]{}\hsindent{5}{}\<[5]%
\>[5]{}\Varid{dup}\mathbin{::}\Varid{a'k'}\;(\Varid{a},\Varid{a}){}\<[E]%
\ColumnHook
\end{hscode}\resethooks

\begin{hscode}\SaveRestoreHook
\column{B}{@{}>{\hspre}l<{\hspost}@{}}%
\column{5}{@{}>{\hspre}l<{\hspost}@{}}%
\column{E}{@{}>{\hspre}l<{\hspost}@{}}%
\>[B]{}\mathbf{instance}\;\Conid{Cartesian}\;(\rightarrow )\;\mathbf{where}{}\<[E]%
\\
\>[B]{}\hsindent{5}{}\<[5]%
\>[5]{}\Varid{exl}\mathrel{=}\lambda (\Varid{a},\Varid{b})\rightarrow \Varid{a}{}\<[E]%
\\
\>[B]{}\hsindent{5}{}\<[5]%
\>[5]{}\Varid{exr}\mathrel{=}\lambda (\Varid{a},\Varid{b})\rightarrow \Varid{b}{}\<[E]%
\\
\>[B]{}\hsindent{5}{}\<[5]%
\>[5]{}\Varid{dup}\mathrel{=}\lambda \Varid{a}\rightarrow (\Varid{a},\Varid{a}){}\<[E]%
\ColumnHook
\end{hscode}\resethooks



\begin{block}

A cartesian functor F between categories \ensuremath{\mathcal{U}\;\Varid{and}\;\mathcal{V}} is such that:

\begin{itemize}
    \item F is a monoidal functor
    \item F exl = exl
    \item F exp = exp
    \item F dup = dup
\end{itemize}
\end{block}
\end{frame}




\begin{frame}{Instance deduction}

From corollary 3.1 and from exl,exr and dup being linear functions we can deduce that:

\ensuremath{\mathcal{D}^{+}} exl = \ensuremath{\lambda \Varid{p}\rightarrow } (exl p,exl)

\ensuremath{\mathcal{D}^{+}} exr = \ensuremath{\lambda \Varid{p}\rightarrow } (exr p,exr)

\ensuremath{\mathcal{D}^{+}} dup = \ensuremath{\lambda \Varid{p}\rightarrow } (dup a,dup)


With this in mind we can arrive at our instance:

exl = \ensuremath{\mathcal{D}}(\ensuremath{\mathcal{D}^{+}} exl)

exr = \ensuremath{\mathcal{D}}(\ensuremath{\mathcal{D}^{+}} exr)

dup = \ensuremath{\mathcal{D}}(\ensuremath{\mathcal{D}^{+}} dup)

\end{frame}



\begin{frame}{Instance deduction} 

Replacing \ensuremath{\mathcal{D}^{+}} with it's definition and remembering linearD's definition we can obtain:

exl = linearD exl

exr = linearD exr

dup = linearD dup


and  convert this directly into a new instance:

\begin{block}{Categorical instance we've deduced}
\begin{hscode}\SaveRestoreHook
\column{B}{@{}>{\hspre}l<{\hspost}@{}}%
\column{5}{@{}>{\hspre}l<{\hspost}@{}}%
\column{E}{@{}>{\hspre}l<{\hspost}@{}}%
\>[B]{}\mathbf{instance}\;\Conid{Cartesian}\;\Conid{D}\;\mathbf{where}{}\<[E]%
\\
\>[B]{}\hsindent{5}{}\<[5]%
\>[5]{}\Varid{exl}\mathrel{=}\Varid{linearD}\;\Varid{exl}{}\<[E]%
\\
\>[B]{}\hsindent{5}{}\<[5]%
\>[5]{}\Varid{exr}\mathrel{=}\Varid{linearD}\;\Varid{exr}{}\<[E]%
\\
\>[B]{}\hsindent{5}{}\<[5]%
\>[5]{}\Varid{dup}\mathrel{=}\Varid{linearD}\;\Varid{dup}{}\<[E]%
\ColumnHook
\end{hscode}\resethooks
\end{block}
\end{frame}



\begin{frame}{Cocartesian category}

This type of categories is the dual of the cartesian type of categories.

\begin{block}{Note}
In this paper coproducts are categorical products, i.e., biproducts
\end{block}

\begin{block}{Definition}
\begin{hscode}\SaveRestoreHook
\column{B}{@{}>{\hspre}l<{\hspost}@{}}%
\column{5}{@{}>{\hspre}l<{\hspost}@{}}%
\column{E}{@{}>{\hspre}l<{\hspost}@{}}%
\>[B]{}\mathbf{class}\;\Conid{Category}\;\Varid{k}\Rightarrow\Conid{Cocartesian}\;\Varid{k}\;\mathbf{where}\mathbin{:}{}\<[E]%
\\
\>[B]{}\hsindent{5}{}\<[5]%
\>[5]{}\Varid{inl}\mathbin{::}\Varid{a'k'}\;(\Varid{a},\Varid{b}){}\<[E]%
\\
\>[B]{}\hsindent{5}{}\<[5]%
\>[5]{}\Varid{inr}\mathbin{::}\Varid{b'k'}\;(\Varid{a},\Varid{b}){}\<[E]%
\\
\>[B]{}\hsindent{5}{}\<[5]%
\>[5]{}\Varid{jam}\mathbin{::}(\Varid{a},\Varid{a})\;\text{\tt 'k'}\;\Varid{a}{}\<[E]%
\ColumnHook
\end{hscode}\resethooks
\end{block}
\end{frame}



\begin{frame}{Cocartesian functors}

\begin{block}{Cocartesian functor definition}

A cocartesian functor F between categories \ensuremath{\mathcal{U}\;\Varid{and}\;\mathcal{V}} is such that:
\begin{itemize}
    \item F is a functor
    \item F inl = inl
    \item F inr = inr
    \item F jam = jam
\end{itemize}
\end{block}
\end{frame}



\end{document}


