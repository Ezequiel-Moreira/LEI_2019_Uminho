\documentclass{beamer}

\mode<presentation>
{
  \usetheme{Warsaw}

  \setbeamercovered{transparent}
}


\usepackage[english]{babel}

\usepackage[latin1]{inputenc}

\usepackage{times}
\usepackage[T1]{fontenc}
% Note that the encoding and the font should match. If T1
% does not look nice, try deleting the line with the fontenc.


\title
{...Machine Learning...}

\author[Artur, Ezequiel, Nelson] 
{Artur \and Ezequiel \and Nelson}

\institute
{Universidade do Minho}

\date
{26 de Abril}

\subject{Talks}

% If you have a file called "university-logo-filename.xxx", where xxx
% is a graphic format that can be processed by latex or pdflatex,
% resp., then you can add a logo as follows:

% \pgfdeclareimage[height=0.5cm]{university-logo}{university-logo-filename}
% \logo{\pgfuseimage{university-logo}}



% Delete this, if you do not want the table of contents to pop up at
% the beginning of each subsection:

\AtBeginSubsection[]
{
  \begin{frame}<beamer>{Outline}
    \tableofcontents[currentsection,currentsubsection]
  \end{frame}
}


% If you wish to uncover everything in a step-wise fashion, uncomment
% the following command: 

%\beamerdefaultoverlayspecification{<+->}


\begin{document}

\begin{frame}
  \titlepage
\end{frame}

\begin{frame}{Indice}
  \tableofcontents
\end{frame}

% - Exactly two or three sections (other than the summary).
% - At *most* three subsections per section.
% - Talk about 30s to 2min per frame. So there should be between about
%   15 and 30 frames, all told.

\section{Nelson}

\begin{frame}{titulo}
\end{frame}


\section{Ezequiel}

\begin{frame}{titulo}
\end{frame}

\section{Fork e Join}
\begin{frame}{Fork e Join}
\end{frame}

\section{Operacoes Numericas}
\begin{frame}{Operacoes Numericas}
\end{frame}

\section{Exemplos}
\begin{frame}{Exemplos}
\end{frame}

\section{Generalizar}
\begin{frame}{Generalizar}
\end{frame}

%============================EXEMPLO==========================
%\section{Introduction}
%
%\subsection[Short First Subsection Name]{First Subsection Name}
%
%\begin{frame}{Make Titles Informative. Use Uppercase Letters.}{Subtitles are optional.}
%  % - A title should summarize the slide in an understandable fashion
%  %   for anyone how does not follow everything on the slide itself.
%
%  \begin{itemize}
%  \item
%    Use \texttt{itemize} a lot.
%  \item
%    Use very short sentences or short phrases.
%  \end{itemize}
%\end{frame}
%
%\begin{frame}{Make Titles Informative.}
%
%  You can create overlays\dots
%  \begin{itemize}
%  \item using the \texttt{pause} command:
%    \begin{itemize}
%    \item
%      First item.
%      \pause
%    \item    
%      Second item.
%    \end{itemize}
%  \item
%    using overlay specifications:
%    \begin{itemize}
%    \item<3->
%      First item.
%    \item<4->
%      Second item.
%    \end{itemize}
%  \item
%    using the general \texttt{uncover} command:
%    \begin{itemize}
%      \uncover<5->{\item
%        First item.}
%      \uncover<6->{\item
%        Second item.}
%    \end{itemize}
%  \end{itemize}
%\end{frame}
%
%\begin{frame}{}
%\end{frame}
%
%\subsection{Second Subsection}
%
%\begin{frame}{Make Titles Informative.}
%\end{frame}
%
%
%
%\section{Summary}
%
%\subsection{coisas1}
%
%\begin{frame}{Summary}
%
%  % Keep the summary *very short*.
%  \begin{itemize}
%  \item
%    The \alert{first main message} of your talk in one or two lines.
%  \item
%    The \alert{second main message} of your talk in one or two lines.
%  \item
%    Perhaps a \alert{third message}, but not more than that.
%  \end{itemize}
%  
%  % The following outlook is optional.
%  \vskip0pt plus.5fill
%  \begin{itemize}
%  \item
%    Outlook
%    \begin{itemize}
%    \item
%      Something you haven't solved.
%    \item
%      Something else you haven't solved.
%    \end{itemize}
%  \end{itemize}
%\end{frame}
%
%\subsection{coisas2}
%\begin{frame}{coisas2}
%
%
%\end{frame}
%=================================================================

\end{document}


